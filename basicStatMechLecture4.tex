%
% Copyright � 2013 Peeter Joot.  All Rights Reserved.
% Licenced as described in the file LICENSE under the root directory of this GIT repository.
%
%\input{../blogpost.tex}
%\renewcommand{\basename}{basicStatMechLecture4}
%\renewcommand{\dirname}{notes/phy452/}
%\newcommand{\keywords}{Statistical mechanics, PHY452H1S, Maxwell-Boltzmann distribution, Fluctuation-Dissipation theorem, Fick's law, continuity equation, ideal gas law, random walk, phase space, dissipation, diffusion, Boltzmann constant}
%\input{../peeter_prologue_print2.tex}

%\beginArtNoToc
%\generatetitle{PHY452H1S Basic Statistical Mechanics.  Lecture 4: Maxwell distribution for gases.  Taught by Prof.\ Arun Paramekanti}
%\chapter{Maxwell distribution for gases}
\label{chap:basicStatMechLecture4}

%\section{Disclaimer}
%
%Peeter's lecture notes from class.  May not be entirely coherent.

\section{Review: Lead up to Maxwell distribution for gases}

For the random walk, after a number of collisions \(N_{\txtc}\), we found that a particle (labeled the \(i\)th) will have a velocity
\begin{equation}\label{eqn:basicStatMechLecture4:20}
\Bv_i(N_{\txtc}) = \Bv_i(0) + \sum_{l = 1}^{N_{\txtc}} \Delta \Bv_i(l).
\end{equation}

We argued that the probability distribution for finding a velocity \(\Bv\) was as in \cref{fig:basicStatMechLecture4:basicStatMechLecture4Fig1}, and
\begin{equation}\label{eqn:basicStatMechLecture4:40}
\calP_{N_{\txtc}}(\Bv_i) \propto \exp
\left(
-\frac{(\Bv_i - \Bv_i(0))^2}{ 2 N_{\txtc}}
\right).
\end{equation}

\imageFigure{../figures/phy452-basicstatmech/basicStatMechLecture4Fig1}{Velocity distribution found without considering kinetic energy conservation.}{fig:basicStatMechLecture4:basicStatMechLecture4Fig1}{0.3}

\section{What went wrong?}

However, we know that this must be wrong, since we require
\begin{equation}\label{eqn:basicStatMechLecture4:60}
T = \inv{2} \sum_{i = 1}^{n} \Bv_i^2 = \mbox{conserved}.
\end{equation}

Where our argument went wrong is that when the particle has a greater than average velocity, the effect of a collision will be to slow it down.  We have to account for

\begin{itemize}
\item Fluctuations \(\rightarrow\) ``random walk''
\item Dissipation \(\rightarrow\) ``slowing down''
\end{itemize}

There were two ingredients to diffusion (the random walk), these were
\begin{itemize}
\item
Conservation of particles
\begin{equation}\label{eqn:basicStatMechLecture4:80}
\PD{t}{c} + \PD{x}{j} = 0.
\end{equation}

We can also think about a conservation of a particles in a velocity space
\begin{equation}\label{eqn:basicStatMechLecture4:100}
\PD{t}{c}(v, t) + \PD{v}{j_v} = 0
\end{equation}

where \(j_v\) is a probability current in this velocity space.
\item Fick's law in velocity space takes the form

\begin{equation}\label{eqn:basicStatMechLecture4:120}
j_v = -D \PD{v}{c}(v, t)
\end{equation}
\end{itemize}

The diffusion results in an ``attempt'' to flatten the distribution of the concentration as in \cref{fig:basicStatMechLecture4:basicStatMechLecture4Fig2}.

\imageFigure{../figures/phy452-basicstatmech/basicStatMechLecture4Fig2}{A friction like term is require to oppose the diffusion pressure.}{fig:basicStatMechLecture4:basicStatMechLecture4Fig2}{0.3}

We'd like to add to the diffusion current an extra frictional like term
\begin{equation}\label{eqn:basicStatMechLecture4:140}
j_v = -D \PD{v}{c}(v, t) - \eta v c(v)
\end{equation}

We want something directed opposite to the velocity and the concentration
\begin{subequations}
\begin{equation}\label{eqn:basicStatMechLecture4:180}
\text{Diffusion current} \equiv -D \PD{v}{c}(v, t)
\end{equation}
\begin{equation}\label{eqn:basicStatMechLecture4:200}
\text{Dissipation current} \equiv - \eta v c(v, t).
\end{equation}
\end{subequations}

This gives
\begin{dmath}\label{eqn:basicStatMechLecture4:160}
\PD{t}{c}(v, t)
= -\PD{v}{j_v}
= -\PD{v}{}
\left(
-D \PD{v}{c}(v, t) - \eta v c(v)
\right)
= D \PDSq{v}{c}(v, t) + \eta \PD{v}{}
\left(
v c(v, t)
\right).
\end{dmath}

Can we find a steady state solution to this equation when \(t \rightarrow \infty\)?  For such a steady state we have
\begin{equation}\label{eqn:basicStatMechLecture4:220}
0 = \frac{d^2 c}{dv^2} + \frac{\eta}{D} \frac{d}{dv}
\left(
v c
\right).
\end{equation}

Integrating once we have
\begin{equation}\label{eqn:basicStatMechLecture4:240}
\frac{d c}{dv} = -\frac{\eta}{D} v c + \text{constant},
\end{equation}
and supposing that \(dc/dv = 0\) at \(v = 0\), integrating once more we have
\boxedEquation{eqn:basicStatMechLecture4:260}{
c(v) \propto \exp
\left(
- \frac{\eta v^2}{2 D}
\right).
}

This is the \underlineAndIndex{Maxwell-Boltzmann} distribution, illustrated in \cref{fig:basicStatMechLecture4:basicStatMechLecture4Fig3}.

\imageFigure{../figures/phy452-basicstatmech/basicStatMechLecture4Fig3}{Maxwell-Boltzmann distribution.}{fig:basicStatMechLecture4:basicStatMechLecture4Fig3}{0.3}

The concentration \(c(v)\) has a probability distribution.

Calculating \(\expectation{v^2}\) from this distribution we can identify the \(D/\eta\) factor.
\begin{dmath}\label{eqn:basicStatMechLecture4:520}
\expectation{v^2}
= \frac{\int v^2 e^{- \eta v^2/2D} dv}{
\int e^{- \eta v^2/2D} dv
}
=
\frac{
\frac{D}{\eta}
\int v \frac{d}{dv} \left(
-e^{- \eta v^2/2D} \right) dv
}{
\int e^{- \eta v^2/2D} dv
}
= -
\frac{D}{\eta}
\frac{
\int -e^{- \eta v^2/2D} dv
}{
\int e^{- \eta v^2/2D} dv
}
=
\frac{D}{\eta}.
\end{dmath}

This also happens to be the energy in terms of temperature (we can view this as a definition of the temperature for now), writing
\begin{equation}\label{eqn:basicStatMechLecture4:280}
\inv{2} m \expectation{\Bv^2} = \inv{2} m \left( \frac{D}{\eta} \right) = \inv{2} \kB T.
\end{equation}

Here
\begin{subequations}
\begin{equation}\label{eqn:basicStatMechLecture4:300}
\kB = \mbox{Boltzmann constant}
\end{equation}
\begin{equation}\label{eqn:basicStatMechLecture4:320}
T = \mbox{absolute temperature}.
\end{equation}
\end{subequations}

\section{Equilibrium steady states}

Fluctuations \(\leftrightarrow\) Dissipation
\boxedEquation{eqn:basicStatMechLecture4:340}{
\frac{D}{\eta} = \frac{ \kB T}{m}.
}

This is a specific example of the more general \underlineAndIndex{fluctuation-dissipation theorem}.

\paragraph{Generalizing to 3D}

Fick's law and the continuity equation in 3D are respectively
\begin{subequations}
\begin{equation}\label{eqn:basicStatMechLecture4:360}
\Bj = -D \spacegrad_\Bv c(\Bv, t) - \eta \Bv c(\Bv, t)
\end{equation}
\begin{equation}\label{eqn:basicStatMechLecture4:380}
\PD{t}{} c(\Bv, t) + \spacegrad_\Bv \cdot \Bj(\Bv, t) = 0.
\end{equation}
\end{subequations}

As above we have for the steady state
\begin{dmath}\label{eqn:basicStatMechLecture4:540}
0 = \PD{t}{} c(\Bv, t) = \spacegrad_\Bv \cdot
\left(
 -D \spacegrad_\Bv c(\Bv, t) - \eta \Bv c(\Bv, t)
\right)
= -D \spacegrad^2_\Bv c - \eta \spacegrad_\Bv \cdot (\Bv c).
\end{dmath}

Integrating once over all space
\begin{equation}\label{eqn:basicStatMechLecture4:560}
D \spacegrad_\Bv c = -\eta \Bv c + \text{vector constant, assumed zero}
\end{equation}

This is three sets of equations, one for each component \(v_\alpha\) of \(\Bv\)
\begin{equation}\label{eqn:basicStatMechLecture4:580}
\PD{v_\alpha}{c} = -\frac{\eta}{D} v_\alpha c.
\end{equation}

So that our steady state equation is
\begin{equation}\label{eqn:basicStatMechLecture4:400}
c(\Bv, t \rightarrow \infty) \propto
\exp
\left(
-\frac{
v_x^2
+
v_y^2
+
v_z^2
}{ 2 (D/\eta) }
\right).
\end{equation}

Computing the average 3D squared velocity for this distribution, we have
\begin{dmath}\label{eqn:basicStatMechLecture4:420}
\inv{2} m \expectation{\Bv \cdot \Bv}
=
\frac
{
   \int dv_x dv_y dv_z
   \left(
   v_x^2
   +
   v_y^2
   +
   v_z^2
   \right)
   \exp
   \left(
   -\frac{
   v_x^2
   +
   v_y^2
   +
   v_z^2
   }{ 2 (D/\eta) }
   \right)
}
{
   \int dv_x dv_y dv_z
   \exp
   \left(
   -\frac{
   v_x^2
   +
   v_y^2
   +
   v_z^2
   }{ 2 (D/\eta) }
   \right)
}
=
\frac
{
   \int dv_x
   v_x^2
   \exp
   \left(
   -\frac{
   v_x^2
   }{ 2 (D/\eta) }
   \right)
}
{
   \int dv_x
   \exp
   \left(
   -\frac{
   v_x^2
   }{ 2 (D/\eta) }
   \right)
}
+
\frac
{
   \int dv_y
   v_y^2
   \exp
   \left(
   -\frac{
   v_y^2
   }{ 2 (D/\eta) }
   \right)
}
{
   \int dv_y
   \exp
   \left(
   -\frac{
   v_y^2
   }{ 2 (D/\eta) }
   \right)
}
+
\frac
{
   \int dv_z
   v_z^2
   \exp
   \left(
   -\frac{
   v_z^2
   }{ 2 (D/\eta) }
   \right)
}
{
   \int dv_z
   \exp
   \left(
   -\frac{
   v_z^2
   }{ 2 (D/\eta) }
   \right)
}
=
3
\left(
   \frac{1}{2}
   \kB T
\right)
= \frac{3}{2}
\kB T.
\end{dmath}

For each component \(v_\alpha\) the normalization kills off all the contributions for the other components, leaving us with the usual \(3 \kB T/2\) ideal gas law kinetic energy.

