%
% Copyright � 2013 Peeter Joot.  All Rights Reserved.
% Licenced as described in the file LICENSE under the root directory of this GIT repository.
%
\makeoproblem{Polymer stretching - entropic force.}{basicStatMech:problemSet5:1}{2013 ps5, p1}{
%(3 points)
Consider a toy model of a polymer in one dimension which is made of \(N\) steps (amino acids) of unit length, going left or right like a random walk. Let one end of this polymer be at the origin and the other end be at a point \(X = \sqrt{N}\) (viz. the rms size of the polymer) , so \(1 \ll X \ll N\). We have previously calculated the number of configurations corresponding to this condition (approximate the binomial distribution by a Gaussian).

\makesubproblem{}{pr:basicStatMechProblemSet5Problem1:a}
Using this, find the entropy of this polymer as \(S = \kB \ln \Omega\). The free energy of this polymer, even in the absence of any other interactions, thus has an entropic contribution, \(F = -T S\). If we stretch this polymer, we expect to have fewer available configurations, and thus a smaller entropy and a higher free energy.

\makesubproblem{}{pr:basicStatMechProblemSet5Problem1:b}
Find the change in free energy of this polymer if we stretch this polymer from its end being at \(X\) to a larger distance \(X + \Delta X\).

\makesubproblem{}{pr:basicStatMechProblemSet5Problem1:c}
%Show that the free energy is quadradic in the displacement for small \(\Delta X\), and hence find the temperature dependent ``entropic spring constant'' of this polymer.  (This entropic force is important to overcome for packing DNA into the nucleus, and in many biological processes.)
Show that the change in free energy is linear in the displacement for small \(\Delta X\), and hence find the temperature dependent ``entropic spring constant'' of this polymer.  (This ``entropic force'' is important to overcome for packing DNA into the nucleus, and in many biological processes.)
%
\paragraph{Typo correction (via email): }
You need to show that the change in free energy is quadratic in the displacement \(\Delta X\), not linear in \(\Delta X\). The force is linear in \(\Delta X\).  (Exactly as for a ``spring''.)
} % makeoproblem

\makeanswer{basicStatMech:problemSet5:1}{

\makeSubAnswer{Entropy.}{pr:basicStatMechProblemSet5Problem1:a}

In lecture 2 probabilities for the sums of fair coin tosses were considered.  Assigning \(\pm 1\) to the events \(Y_k\) for heads and tails coin tosses respectively, a random variable \(Y = \sum_k Y_k\) for the total of \(N\) such events was found to have the form

%\eqnref{eqn:basicStatMechLecture2:460}
\begin{equation}\label{eqn:basicStatMechProblemSet5Problem1:20}
P_N(Y)
=
\left\{
\begin{array}{l l}
\left(\inv{2}\right)^N
\frac{N!}{
\left(\frac{N-Y}{2}\right)!
\left(\frac{N+Y}{2}\right)!
}
& \quad \mbox{if \(Y\) and \(N\) have same parity} \\
0& \quad \mbox{otherwise}
\end{array}
\right.
\end{equation}

For an individual coin tosses we have averages \(\expectation{Y_1} = 0\), and \(\expectation{Y_1^2} = 1\), so the central limit theorem provides us with a large \(N\) Gaussian approximation for this distribution
\begin{equation}\label{eqn:basicStatMechProblemSet5Problem1:40}
P_N(Y)
\approx
\frac{2}{\sqrt{2 \pi N}} \exp\left( -\frac{Y^2}{2N} \right).
\end{equation}

This fair coin toss problem can also be thought of as describing the coordinate of the end point of a one dimensional polymer with the beginning point of the polymer is fixed at the origin.  Writing \(\Omega(N, Y)\) for the total number of configurations that have an end point at coordinate \(Y\) we have
\begin{equation}\label{eqn:basicStatMechProblemSet5Problem1:60}
P_N(Y) = \frac{\Omega(N, Y)}{2^N},
\end{equation}

From this, the total number of configurations that have, say, length \(X = \Abs{Y}\), in the large \(N\) Gaussian approximation, is
\begin{equation}\label{eqn:basicStatMechProblemSet5Problem1:80}
\Omega(N, X) = 2^N
\lr{ P_N(+X) +P_N(-X) }
=
\frac{2^{N + 2}}{\sqrt{2 \pi N}} \exp\left( -\frac{X^2}{2N} \right).
\end{equation}

The entropy associated with a one dimensional polymer of length \(X\) is therefore
\begin{equation}\label{eqn:basicStatMechProblemSet5Problem1:100}
S_N(X)
=
- \kB \frac{X^2}{2N} + \kB \ln \frac{2^{N + 2}}{\sqrt{2 \pi N}}
=
- \kB \frac{X^2}{2N} + \text{constant}.
\end{equation}

Writing \(S_0\) for this constant the free energy is
\boxedEquation{eqn:basicStatMechProblemSet5Problem1:120}{
F = U - T S = U + \kB T \frac{X^2}{2N} + S_0 T.
}

\makeSubAnswer{Change in free energy.}{pr:basicStatMechProblemSet5Problem1:b}

At constant temperature, stretching the polymer from its end being at \(X\) to a larger distance \(X + \Delta X\), results in a free energy change of
\begin{dmath}\label{eqn:basicStatMechProblemSet5Problem1:260}
\Delta F
= F( X + \Delta X ) - F(X)
=
 \frac{\kB T}{2N} \lr{(X + \Delta X)^2 - X^2}
=
 \frac{\kB T}{2N} \lr{ 2 X \Delta X + (\Delta X)^2 }.
\end{dmath}

If \(\Delta X\) is assumed small, our constant temperature change in free energy \(\Delta F \approx (\partial F/\partial X)_T \Delta X\) is
\boxedEquation{eqn:basicStatMechProblemSet5Problem1:280}{
\Delta F =
 \frac{\kB T}{N} X \Delta X.
}

\makeSubAnswer{Temperature dependent spring constant.}{pr:basicStatMechProblemSet5Problem1:c}

I found the statement and subsequent correction of the problem statement somewhat confusing.  To figure this all out, I thought it was reasonable to step back and relate free energy to the entropic force explicitly.

Consider temporarily a general thermodynamic system, for which we have by definition free energy and thermodynamic identity respectively
\begin{subequations}
\begin{equation}\label{eqn:basicStatMechProblemSet5Problem1:140}
F = U - T S,
\end{equation}
\begin{equation}\label{eqn:basicStatMechProblemSet5Problem1:160}
dU = T dS - P dV.
\end{equation}
\end{subequations}

The differential of the free energy is
\begin{dmath}\label{eqn:basicStatMechProblemSet5Problem1:180}
dF
= dU - T dS - S dT
= -P dV - S dT
=
\lr{\PD{T}{F}}_V dT
+
\lr{\PD{V}{F}}_T dV.
\end{dmath}

Forming the \textAndIndex{wedge product} with \(dT\), we arrive at the two form
\begin{dmath}\label{eqn:basicStatMechProblemSet5Problem1:200}
0 =
\lr{
\lr{ P +
\lr{\PD{V}{F}}_T
} dV
+ \lr{ S +
\lr{\PD{T}{F}}_V
} dT } \wedge dT
=
\lr{ P +
\lr{\PD{V}{F}}_T
} dV \wedge dT,
\end{dmath}

This provides the relation between free energy and the ``pressure'' for the system
\begin{equation}\label{eqn:basicStatMechProblemSet5Problem1:220}
P = - \lr{\PD{V}{F}}_T.
\end{equation}

For a system with a constant cross section \(\Delta A\), \(dV = \Delta A dX\), so the force associated with the system is
\begin{equation}\label{eqn:basicStatMechProblemSet5Problem1:240}
f = P \Delta A
= - \inv{\Delta A} \lr{\PD{X}{F}}_T \Delta A,
\end{equation}
or
\begin{equation}\label{eqn:basicStatMechProblemSet5Problem1:300}
f = - \lr{\PD{X}{F}}_T.
\end{equation}

Okay, now we have a relation between the force and the rate of change of the free energy
\begin{equation}\label{eqn:basicStatMechProblemSet5Problem1:320}
f(X) = -\frac{\kB T}{N} X.
\end{equation}

Our temperature dependent ``entropic spring constant'' in analogy with \(f = -k X\), is therefore
\boxedEquation{eqn:basicStatMechProblemSet5Problem1:340}{
k = \frac{\kB T}{N}.
}
}
