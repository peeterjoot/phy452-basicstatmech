%
% Copyright � 2013 Peeter Joot.  All Rights Reserved.
% Licenced as described in the file LICENSE under the root directory of this GIT repository.
%
%\input{../blogpost.tex}
%\renewcommand{\basename}{pathriaCh3pr30}
%\renewcommand{\dirname}{notes/phy452/}
\newcommand{\keywords}{Statistical mechanics, PHY452H1S, quantum anharmonic oscillator, heat capacity, average energy, partition function}
%
%\newcommand{\nbref}[1]{notes/phy452/mathematica/#1}
%
%\input{../peeter_prologue_print2.tex}
%
%\beginArtNoToc
%
%\generatetitle{Quantum anharmonic oscillator}
%%\chapter{Quantum anharmonic oscillator}
\label{chap:pathriaCh3pr30}
%
\makeoproblem{Quantum anharmonic oscillator}{pr:pathriaCh3pr30:3:30}{\citep{pathriastatistical} pr. 3.30}{
The energy levels of a quantum-mechanical, one-dimensional, anharmonic oscillator may be approximated as
\begin{equation}\label{eqn:pathriaCh3pr30:20}
\epsilon_n = \lr{n + \inv{2}} \Hbar \omega - x \lr{ n + \inv{2} }^2 \Hbar \omega\qquad \mbox{\(n = 0, 1, 2, \cdots\)}
\end{equation}
%
The parameter \(x\), usually \(\ll 1\), represents the degree of anharmonicity.  Show that, to the first order in \(x\) and the fourth order in \(u \equiv \Hbar \omega/\kB T\), the specific heat of a system of \(N\) such oscillators is given by
\begin{equation}\label{eqn:pathriaCh3pr30:40}
C = N \kB \lr{
\lr{1 - \inv{12} u^2 + \inv{240} u^4}
+ 4 x \lr{ \inv{u} + \inv{80} u^3 }
}.
\end{equation}
} % makeoproblem
%
\makeanswer{pr:pathriaCh3pr30:3:30}{
We can expand the partition function in a first order Taylor series about \(x = 0\), then evaluate the sums
\begin{equation}\label{eqn:pathriaCh3pr30:60}
\begin{aligned}
Z_1
&= \sum_{n = 0}^\infty \exp\lr{ -\beta \lr{n + \inv{2}} \Hbar \omega  + \beta x \lr{ n + \inv{2} }^2 \Hbar \omega } \\
&= \sum_{n = 0}^\infty \exp\lr{ - \lr{n + \inv{2}} u + x \lr{ n + \inv{2} }^2 u } \\
&\approx \sum_{n = 0}^\infty e^{ - \lr{n + \inv{2}} u } \lr{ 1 + x u \lr{ n + \inv{2} }^2 }.
\end{aligned}
\end{equation}
%
The quadratic sum can be evaluated indirectly as it can be expressed as a derivative
\begin{equation}\label{eqn:pathriaCh3pr30:80}
\begin{aligned}
Z_1
&= \lr{ 1 + x u \frac{d^2}{du^2} } \sum_{n = 0}^\infty e^{ - \lr{n + \inv{2}} u } \\
&= \lr{ 1 + x u \frac{d^2}{du^2} } e^{-u/2} \sum_{n = 0}^\infty e^{ - n u } \\
&= \lr{ 1 + x u \frac{d^2}{du^2} } e^{-u/2} \frac{ 1 }{1 - e^{-u} } \\
&= \lr{ 1 + x u \frac{d^2}{du^2} } \frac{ 1 }{e^{u/2} - e^{-u/2} } \\
&= \lr{ 1 + x u \frac{d^2}{du^2} } \frac{ 1 }{2 \sinh(u/2)}.
\end{aligned}
\end{equation}
%
Finally, evaluation of the derivatives gives us
\begin{equation}\label{eqn:pathriaCh3pr30:100}
Z_1
=
\frac{ 1 }{\sinh(u/2)}
\lr{ 1 + x u \frac{2 \coth^2(u/2) - 1}{8} }.
\end{equation}
%
Now we'd like to compute the specific heat in terms of derivatives of \(u\).  First, for the average energy
\begin{equation}\label{eqn:pathriaCh3pr30:120}
\begin{aligned}
\expectation{H}
&= -N \PD{\beta}{} \ln Z_1 \\
&= -N \Hbar \omega \PD{u}{} \ln Z_1.
\end{aligned}
\end{equation}
%
The specific heat is
\begin{equation}\label{eqn:pathriaCh3pr30:140}
\begin{aligned}
\CV
&= \PD{T}{\expectation{H}} \\
&= \PD{T}{u} \PD{u}{\expectation{H}} \\
&= \frac{\Hbar \omega}{\kB} \PD{T}{(1/T)} \PD{u}{\expectation{H}} \\
&= -\frac{\Hbar \omega}{\kB T^2} \PD{u}{\expectation{H}} \\
&= -\frac{\kB u^2}{\Hbar \omega} \PD{u}{\expectation{H}},
\end{aligned}
\end{equation}
%
or
\begin{equation}\label{eqn:pathriaCh3pr30:160}
\CV = N \kB u^2 \PDSq{u}{} \ln Z_1.
\end{equation}
%
Actually computing that is messy algebra (See \nbref{pathria_3_30.nb}), and the result isn't particularly interesting looking.  The plot \cref{fig:pathria_3_30:pathria_3_30Fig1A} is interesting though and shows negative heat capacities near zero and a funny little jog near \(\CV = 0\).
% pathria_3_30.nb
\imageFigure{../figures/phy452-basicstatmech/pathria_3_30Fig1A}{Quantum anharmonic heat capacity.}{fig:pathria_3_30:pathria_3_30Fig1A}{0.2}
%
Also confirmed in the Mathematica notebook is equation \ref{eqn:pathriaCh3pr30:40}, which follows by first doing a first order series expansion in \(x\), then a subsequent series expansion in \(u\).
%
} % makeanswer
%
%\EndArticle
