%
% Copyright � 2013 Peeter Joot.  All Rights Reserved.
% Licenced as described in the file LICENSE under the root directory of this GIT repository.
%
\makeoproblem{Diffusion}{basicStatMech:problemSet2:1}{2013 ps2, p1}{
%
The usual diffusion equation for the probability density in one dimension is given by
\begin{equation}\label{eqn:basicStatMechProblemSet2Problem1:20}
\PD{t}{P}(x, t) = D \PDSq{x}{P}(x, t),
\end{equation}
where \(D\) is the diffusion constant.  Define the Fourier components of the probability distribution via
\begin{subequations}
\begin{equation}\label{eqn:basicStatMechProblemSet2Problem1:40}
P(x, t) = \int_{-\infty}^\infty \frac{dk}{2 \pi} \tilde{P}(k, t) \exp\left( i k x \right)
\end{equation}
\begin{equation}\label{eqn:basicStatMechProblemSet2Problem1:60}
\tilde{P}(k, t) = \int_{-\infty}^\infty dx P(x, t) \exp\left( -i k x \right).
\end{equation}
\end{subequations}
%
This is useful since the diffusion equation is linear in the probability and each Fourier component will evolve independently.  Using this, solve the diffusion equation to obtain \(P(k,t)\) in Fourier space given the initial \(\tilde{P}(k,0)\).
%
\makesubproblem{Assuming an initial Gaussian profile}{basicStatMech:problemSet2:1a}
\begin{equation}\label{eqn:basicStatMechProblemSet2Problem1:80}
P(x, 0) = \inv{\sqrt{2 \pi \sigma^2}} \exp
\left(
-\frac{x^2}{2 \sigma^2}
\right),
\end{equation}
%
obtain the probability density \(P(x,t)\) at a later time \(t\).  (NB: Fourier transform, get the solution, transform back.)  Schematically plot the profile at the initial time and a later time.
%
\makesubproblem{A small modulation on top of a uniform value}{basicStatMech:problemSet2:1b}
Let the probability density be proportional to
\begin{equation}\label{eqn:basicStatMechProblemSet2Problem1:100}
\inv{L} + A \sin(k_0 x),
\end{equation}
at an initial time \(t = 0\).  Assume this is in a box of large size \(L\), but ignore boundary effects except to note that it will help to normalize the constant piece, assuming the oscillating piece integrates to zero.  Also note that we have
to assume \(A < 1/L\) to ensure that the probability density is positive.  Obtain \(P(x,t)\) at a later time \(t\). Roughly how long does the modulation take to decay away? Schematically plot the profile at the initial time and a later time.
} % makeoproblem
%
\makeanswer{basicStatMech:problemSet2:1}{
Inserting the transform definitions we have
\begin{equation}\label{eqn:basicStatMechProblemSet2Problem1:120}
\begin{aligned}
0
&= \left( \PD{t}{} - D \PDSq{x}{} \right) P \\
&=
\left( \PD{t}{} - D \PDSq{x}{} \right)
\int_{-\infty}^\infty \frac{dk}{2 \pi} \tilde{P}(k, t) \exp\left( i k x \right) \\
&=
\int_{-\infty}^\infty \frac{dk}{2 \pi}
\left(
\PD{t}{}
\tilde{P}(k, t)
+
k^2 D
\tilde{P}(k, t)
\right)
\exp\left( i k x \right),
\end{aligned}
\end{equation}
%
We conclude that
\begin{equation}\label{eqn:basicStatMechProblemSet2Problem1:140}
0 =
\tilde{P}(k, t)
+
k^2 D
\tilde{P}(k, t),
\end{equation}
or
\begin{equation}\label{eqn:basicStatMechProblemSet2Problem1:160}
\tilde{P}(k, t) = A(k) e^{-k^2 D t}.
\end{equation}
%
If the Fourier transform of the distribution is constant until time \(t\), so that \(\tilde{P}(k, t < 0) = \tilde{P}(k, 0)\), we can write
\boxedEquation{eqn:basicStatMechProblemSet2Problem1:180}{
\tilde{P}(k, t) = \tilde{P}(k, 0) e^{-k^2 D t}.
}
%
The time evolution of the distributions transform just requires multiplication by the decreasing exponential factor \(e^{-k^2 D t}\).
%
\paragraph{Propagator for the diffusion equation}
%
We can also use this to express the explicit time evolution of the distribution
\begin{equation}\label{eqn:basicStatMechProblemSet2Problem1:200}
\begin{aligned}
P(x, t)
&=
\int_{-\infty}^\infty \frac{dk}{2 \pi}
\tilde{P}(k, 0)
e^{-k^2 D t}
\exp\left( i k x \right) \\
&=
\int_{-\infty}^\infty \frac{dk}{2 \pi}
\int_{-\infty}^\infty dx' P(x', 0) \exp\left( -i k x' \right)
e^{-k^2 D t}
\exp\left( i k x \right) \\
&=
\int_{-\infty}^\infty
dx' P(x', 0)
\int_{-\infty}^\infty
\frac{dk}{2 \pi}
\exp\left( -k^2 D t + i k (x - x') \right).
\end{aligned}
\end{equation}
%
Our distribution time evolution is given by convolve with a propagator function
\begin{subequations}
\begin{equation}\label{eqn:basicStatMechProblemSet2Problem1:220}
P(x, t) = \int dx' P(x', 0) G(x', x)
\end{equation}
\begin{equation}\label{eqn:basicStatMechProblemSet2Problem1:240}
G(x', x)
=
\int_{-\infty}^\infty
\frac{dk}{2 \pi}
\exp\left( -k^2 D t + i k (x - x') \right).
\end{equation}
\end{subequations}
%
For \(t \ge 0\) we can complete the square, finding that this propagator is
\begin{equation}\label{eqn:basicStatMechProblemSet2Problem1:260}
\begin{aligned}
G(x', x)
&=
\int_{-\infty}^\infty
\frac{dk}{2 \pi}
\exp\left( -k^2 D t + i k (x - x') \right) \\
&=
\exp\left( \left(\frac{i (x - x')}{2 \sqrt{D t}} \right)^2 \right) \times \\
&\quad \int_{-\infty}^\infty
\frac{dk}{2 \pi}
\exp\left( - \left(k \sqrt{D t} + \frac{i (x - x')}{2 \sqrt{D t}} \right)^2 \right),
\end{aligned}
\end{equation}
or
\boxedEquation{eqn:basicStatMechProblemSet2Problem1:360}{
G(x', x)
=
\frac{1}{\sqrt{4 \pi D t}} \exp\left(
-\frac{(x - x')^2}{4 D t}
\right).
}
%
A schematic plot of this function as a function of \(t\) for fixed \(x - x'\) is plotted in \cref{fig:problemSet2distPropagator:problemSet2distPropagatorFig1}.
% attemptAtProblemSet2Problem1iiIntegrals.nb
\imageFigure{../figures/phy452-basicstatmech/problemSet2distPropagatorFig1}{Propagator for the diffusion equation.}{fig:problemSet2distPropagator:problemSet2distPropagatorFig1}{0.2}
%
\makeSubAnswer{Gaussian}{basicStatMech:problemSet2:1a}
%
For the Gaussian of \eqnref{eqn:basicStatMechProblemSet2Problem1:80} we compute the initial time Fourier transform
\begin{equation}\label{eqn:basicStatMechProblemSet2Problem1:280}
\begin{aligned}
\tilde{P}(k)
&=
\int_{-\infty}^\infty dx
\inv{\sqrt{2 \pi \sigma^2}} \exp
\left(
-\frac{x^2}{2 \sigma^2}
-i k x
\right) \\
&=
\inv{\sqrt{2 \pi \sigma^2}}
\exp
\left(
-\left( \frac{\sqrt{ 2 \sigma^2}}{2} k i
\right)^2
\right) \\
&\quad \int_{-\infty}^\infty dx
\exp
\left(
-\left( \frac{x}{\sqrt{2 \sigma^2} } + \frac{\sqrt{ 2 \sigma^2}}{2} k i
\right)^2
\right)
=
\exp\left(
-\frac{\sigma^2 k^2}{2}
\right).
\end{aligned}
\end{equation}
%
The time evolution of the generating function is
\begin{equation}\label{eqn:basicStatMechProblemSet2Problem1:300}
\tilde{P}(k, t)
=
\exp\left(
-\frac{\sigma^2 k^2}{2} - D k^2 t
\right),
\end{equation}
and we can find our time evolved probability density by inverse transforming
\begin{equation}\label{eqn:basicStatMechProblemSet2Problem1:320}
\begin{aligned}
P(x, t)
&=
\int_{-\infty}^\infty \frac{dk}{2 \pi}
\exp\left(
-\frac{\sigma^2 k^2}{2} - D k^2 t + i k x
\right) \\
&=
\exp\left(
i \frac{x}{ 2 \sqrt{\frac{\sigma^2}{2} + D t} }
\right)^2
\int_{-\infty}^\infty \frac{dk}{2 \pi} \\
&\quad \exp\left(
-\left(
k \sqrt{\frac{\sigma^2}{2} + D t}
+ i \frac{x}{ 2
\sqrt{\frac{\sigma^2}{2} + D t}
}
\right)^2
\right).
\end{aligned}
\end{equation}
%
For \(t \ge 0\) this is
\boxedEquation{eqn:basicStatMechProblemSet2Problem1:340}{
P(x, t)
=
\inv{\sqrt{2 \pi \left( \sigma^2 + 2 D t \right) } }
\exp\left(
-\frac{x^2}{2 \left( \sigma^2 + 2 D t \right) }
\right).
}
%
As a check, we see that this reproduces the \(t = 0\) value as expected.  A further check using Mathematica (\nbref{attemptAtProblemSet2Problem1iiIntegrals.nb}) applying the propagator \eqnref{eqn:basicStatMechProblemSet2Problem1:360}, also finds the same result as this manual calculation.
%
This is plotted for \(D = \sigma = 1\) in \cref{fig:problemSet2gaussianTimeEvolution:problemSet2gaussianTimeEvolutionFig2} for a couple different times \(t \ge 0\).
%
\imageFigure{../figures/phy452-basicstatmech/problemSet2gaussianTimeEvolutionFig2}{Gaussian PDF time evolution with diffusion.}{fig:problemSet2gaussianTimeEvolution:problemSet2gaussianTimeEvolutionFig2}{0.2}
%
\makeSubAnswer{Boxed constant with small oscillation}{basicStatMech:problemSet2:1b}
%
The normalization of the distribution depends on the interval boundaries.  With the box range given by \(x \in [a, a + L]\) we have
\begin{equation}\label{eqn:basicStatMechProblemSet2Problem1:380}
\int_a^{a + L} dx \left( \inv{L} + A \sin( k_0 x) \right) dx
=
1 - \frac{A}{k_0} \left( \cos( k_0(a + L) ) - \cos( k_0 a ) \right).
\end{equation}
%
With an even range for box \(x \in [-L/2, L/2]\) this is unity.
%
To find the distribution at a later point in time we can utilize the propagator
\begin{equation}\label{eqn:basicStatMechProblemSet2Problem1:400}
P(x, t)
=
\int_{-L/2}^{L/2} dx' \inv{2 \sqrt{\pi D t} }
\left( \inv{L} + A \sin( k_0 x' ) \right) \exp
\left(
- \frac{(x' - x)^2}{2 \sqrt{D t} }
\right).
\end{equation}
%
Let's write this as
\begin{subequations}
\begin{equation}\label{eqn:basicStatMechProblemSet2Problem1:520}
P(x, t) = P_{\mathrm{rect}}(x, t) + P_{\mathrm{sin}}(x, t)
\end{equation}
\begin{equation}\label{eqn:basicStatMechProblemSet2Problem1:540}
P_{\mathrm{rect}}(x, t)
=
\inv{2 L \sqrt{\pi D t} }
\int_{-L/2}^{L/2} dx'
\exp
\left(
- \frac{(x' - x)^2}{2 \sqrt{D t} }
\right)
\end{equation}
\begin{equation}\label{eqn:basicStatMechProblemSet2Problem1:560}
\begin{aligned}
P_{\mathrm{sin}}(x, t)
&=
\frac{A}{2 \sqrt{\pi D t} }
\int_{-L/2}^{L/2} dx'
\sin( k_0 x' ) \\
&\quad\exp
\left(
- \frac{(x' - x)^2}{2 \sqrt{D t} }
\right).
\end{aligned}
\end{equation}
\end{subequations}
%
Applying a \(u = (x' - x)/\sqrt{4 D t}\) change of variables for the first term, we can reduce it to a difference of error functions
\begin{equation}\label{eqn:basicStatMechProblemSet2Problem1:420}
\begin{aligned}
P_{\mathrm{rect}}(x, t)
&=
\inv{L}
\int_{-L/2}^{L/2} dx' \inv{2 \sqrt{\pi D t} }
\exp
\left(
- \frac{(x' - x)^2}{2 \sqrt{D t} }
\right) \\
&=
\inv{L \sqrt{\pi}}
\int_{-\frac{L/2 +x}{2 \sqrt{D t}}}^{\frac{L/2 - x}{2 \sqrt{Dt}}}
du
e^{-u^2} \\
&=
\inv{2 L} \left(
\erf\left( \frac{L/2 -x}{2 \sqrt{D t}} \right)
-\erf\left( -\frac{L/2 +x}{2 \sqrt{D t}} \right)
\right).
\end{aligned}
\end{equation}
%
Following Mathematica, lets introduce a two argument error function for the difference between two points
\begin{equation}\label{eqn:basicStatMechProblemSet2Problem1:480}
\erf(z_0, z_1) \equiv \erf(z_1) - \erf(z_0).
\end{equation}
%
Using that our rectangular function's time evolution can be written
\begin{equation}\label{eqn:basicStatMechProblemSet2Problem1:500}
P_{\mathrm{rect}}(x, t)
=
\inv{2 L} \erf
\left(
-\frac{L/2 +x}{2 \sqrt{D t}}
,
\frac{L/2 -x}{2 \sqrt{D t}}
\right).
\end{equation}
%
For \(L = D = 1\), and \(t = 10^{-8}\), this is plotted in \cref{fig:problemSet2Problem1iiRectPropagator:problemSet2Problem1iiRectPropagatorFig3}.  Somewhat surprisingly, this difference of error functions does appear to result in a rectangular function for small \(t\).
% attemptAtProblemSet2Problem1iiIntegrals.nb
\imageFigure{../figures/phy452-basicstatmech/problemSet2Problem1iiRectPropagatorFig3}{Rectangular part of the probability distribution, small \(t\).}{fig:problemSet2Problem1iiRectPropagator:problemSet2Problem1iiRectPropagatorFig3}{0.2}
%
The time evolution of this non-oscillation part of the probability distribution is plotted as a function of both \(t\) and \(x\) in \cref{fig:problemSet2Problem1iiRectPropagator:problemSet2Problem1iiRectPropagatorFig4}.
% attemptAtProblemSet2Problem1iiIntegrals.nb
\imageFigure{../figures/phy452-basicstatmech/problemSet2Problem1iiRectPropagatorFig4}{Time evolution of the rectangular part of the probability distribution.}{fig:problemSet2Problem1iiRectPropagator:problemSet2Problem1iiRectPropagatorFig4}{0.2}
%
For the sine piece we can also find a solution in terms of (complex) error functions
\begin{equation}\label{eqn:basicStatMechProblemSet2Problem1:440}
\begin{aligned}
&P_{\mathrm{sin}}(x, t) \\
&=
A
\int_{-L/2}^{L/2} dx' \inv{2 \sqrt{\pi D t} }
\sin( k_0 x' )
\exp
\left(
- \frac{(x' - x)^2}{2 \sqrt{D t} }
\right) \\
&=
\frac{A}{\sqrt{\pi}}
\int_{-\frac{L/2 +x}{2 \sqrt{D t}}}^{\frac{L/2 - x}{2 \sqrt{Dt}}}
du
\sin( k_0 ( x + 2 u \sqrt{D t} ) ) e^{-u^2} \\
&=
\frac{A}{2 i \sqrt{\pi}}
\int_{-\frac{L/2 +x}{2 \sqrt{D t}}}^{\frac{L/2 - x}{2 \sqrt{Dt}}}
du
\left(
e^{ i k_0 ( x + 2 u \sqrt{D t} ) }
-e^{ -i k_0 ( x + 2 u \sqrt{D t} ) }
\right)
e^{-u^2} \\
%%&=
%%\frac{A}{2 i \sqrt{\pi}}
%%\left(
%%e^{ i k_0 x }
%%\int_{-\frac{L/2 +x}{2 \sqrt{D t}}}^{\frac{L/2 - x}{2 \sqrt{Dt}}}
%%du
%%e^{ -u^2 + 2 i k_0 u \sqrt{D t} }
%%-
%%e^{ -i k_0 x }
%%\int_{-\frac{L/2 +x}{2 \sqrt{D t}}}^{\frac{L/2 - x}{2 \sqrt{Dt}}}
%%du
%%e^{ -u^2 -2 i k_0 u \sqrt{D t} }
%%\right) \\
%%&=
%%\frac{A}{2 i \sqrt{\pi}}
%%e^{ -k_0^2 D t }
%%\left(
%%e^{ i k_0 x }
%%\int_{-\frac{L/2 +x}{2 \sqrt{D t}}}^{\frac{L/2 - x}{2 \sqrt{Dt}}}
%%du
%%e^{ -(u - i k_0 \sqrt{D t})^2 }
%%-
%%e^{ -i k_0 x }
%%\int_{-\frac{L/2 +x}{2 \sqrt{D t}}}^{\frac{L/2 - x}{2 \sqrt{Dt}}}
%%du
%%e^{ -(u + i k_0 \sqrt{D t})^2 }
%%\right) \\
%%&=
%%\frac{A}{2 i \sqrt{\pi}}
%%e^{ -k_0^2 D t }
%%\left(
%%e^{ i k_0 x }
%%\int_{-\frac{L/2 +x}{2 \sqrt{D t}} - i k_0 \sqrt{D t}}^{\frac{L/2 - x}{2 \sqrt{Dt}} - i k_0 \sqrt{D t}}
%%dv
%%e^{ -v^2 }
%%-
%%e^{ -i k_0 x }
%%\int_{-\frac{L/2 +x}{2 \sqrt{D t}} + i k_0 \sqrt{D t}}^{\frac{L/2 - x}{2 \sqrt{Dt}} + i k_0 \sqrt{D t}}
%%dv
%%e^{ -v^2 }
%%\right)
,
\end{aligned}
\end{equation}
which reduces to
\begin{equation}\label{eqn:basicStatMechProblemSet2Problem1:n}
\begin{aligned}
P_{\mathrm{sin}}&(x, t)
= \\
&\frac{A}{4 i }
e^{ -k_0^2 D t }
e^{ i k_0 x }
\erf\left(
   -\frac{L/2 +x}{2 \sqrt{D t}} - i k_0 \sqrt{D t},
   \frac{L/2 - x}{2 \sqrt{Dt}} - i k_0 \sqrt{D t}
\right) \\
&\quad -
\frac{A}{4 i }
e^{ -k_0^2 D t }
e^{ -i k_0 x }
\erf\left(
   -\frac{L/2 +x}{2 \sqrt{D t}} + i k_0 \sqrt{D t},
   \frac{L/2 - x}{2 \sqrt{Dt}} + i k_0 \sqrt{D t}
\right).
\end{aligned}
\end{equation}
This is plotted for \(A = D = L = 1\), \(k_0 = 8 \pi\), and \(t = 10^{-8}\) in \cref{fig:problemSet2Problem1iiSinePropagator:problemSet2Problem1iiSinePropagatorFig5}.
% attemptAtProblemSet2Problem1iiIntegrals.nb
\imageFigure{../figures/phy452-basicstatmech/problemSet2Problem1iiSinePropagatorFig5}{Verification at \(t \rightarrow 0\), sine like diffusion.}{fig:problemSet2Problem1iiSinePropagator:problemSet2Problem1iiSinePropagatorFig5}{0.2}
%
The diffusion of this, again for \(A = D = L = 1\), \(k_0 = 8 \pi\), and \(t \in [10^{-5}, 0.01]\) is plotted in \cref{fig:problemSet2Problem1iiSinePropagator:problemSet2Problem1iiSinePropagatorFig6}.  Again we see that we have the expected sine for small \(t\).
%
\imageFigure{../figures/phy452-basicstatmech/problemSet2Problem1iiSinePropagatorFig6}{Diffusion of the oscillatory term.}{fig:problemSet2Problem1iiSinePropagator:problemSet2Problem1iiSinePropagatorFig6}{0.2}
%
Putting both the rectangular and the windowed sine portions of the probability distribution together, we have the diffusion result for the entire distribution
\boxedEquation{eqn:basicStatMechProblemSet2Problem1:460}{
\begin{aligned}
P&(x, t)
=
\inv{2 L} \erf
\left(
-\frac{L/2 +x}{2 \sqrt{D t}}
,
\frac{L/2 -x}{2 \sqrt{D t}}
\right) \\
&\quad +
\frac{A}{4 i }
e^{ -k_0^2 D t + i k_0 x }
\erf\left(
   -\frac{L/2 +x}{2 \sqrt{D t}} - i k_0 \sqrt{D t},
   \frac{L/2 - x}{2 \sqrt{Dt}} - i k_0 \sqrt{D t}
\right) \\
&\quad -
\frac{A}{4 i }
e^{ -k_0^2 D t -i k_0 x }
\erf\left(
   -\frac{L/2 +x}{2 \sqrt{D t}} + i k_0 \sqrt{D t},
   \frac{L/2 - x}{2 \sqrt{Dt}} + i k_0 \sqrt{D t}
\right)
\end{aligned}
}
%
It is certainly ugly looking!  We see that the oscillation die off is dependent on the \(\exp( -k_0^2 D t)\) term.  In time
\begin{equation}\label{eqn:basicStatMechProblemSet2Problem1:580}
t = \inv{k_0^2 D},
\end{equation}
that oscillation dies away to \(1/e\) of its initial amplitude.  This dispersion is plotted at times \(t = 10^{-5}\) and \(t = 1/(k_0^2 D)\) for \(L = D = 1\), \(k_0 = 8 \pi\) and \(A = 1/2\) in \cref{fig:problemSet2Problem1iiSinePropagator:problemSet2Problem1iiSinePropagatorFig7}.
%
\imageFigure{../figures/phy452-basicstatmech/problemSet2Problem1iiSinePropagatorFig7}{Initial time distribution and dispersion of the oscillatory portion to \(1/e\) of initial amplitude.}{fig:problemSet2Problem1iiSinePropagator:problemSet2Problem1iiSinePropagatorFig7}{0.2}
%
Similar to the individual plots of \(P_{\mathrm{rect}}(x, t)\) and \(P_{\mathrm{sin}}(x, t)\) above, we plot the time evolution of the total probability dispersion \(P(x, t)\) in \cref{fig:problemSet2Problem1iiSinePropagator:problemSet2Problem1iiSinePropagatorFig8}.  We see in the plots above that the rectangular portion of this distribution will also continue to flatten over time after most of the oscillation has also died off.
% attemptAtProblemSet2Problem1iiIntegrals.nb
\imageFigure{../figures/phy452-basicstatmech/problemSet2Problem1iiSinePropagatorFig8}{Diffusion of uniform but oscillating probability distribution.}{fig:problemSet2Problem1iiSinePropagator:problemSet2Problem1iiSinePropagatorFig8}{0.2}
%
\paragraph{An easier solution for the sinusoidal part}
%
After working this problem, talking with classmates about how they solved it (because I was sure I'd done this windowed oscillating distribution the hard way), I now understand what was meant by ``ignore boundary effects''.  That is, ignore the boundary effects in the sinusoid portion of the distribution.  I didn't see how we could ignore the boundary effects because doing so would make the sine Fourier transform non-convergent.  Ignoring pesky ideas like convergence we can ``approximate'' the Fourier transform of the windowed sine as
\begin{equation}\label{eqn:basicStatMechProblemSet2Problem1:600}
\begin{aligned}
\tilde{P}_{\mathrm{sin}}(k)
&\approx
A \int_{-\infty}^\infty \sin (k_0 x) e^{-i k x} dx \\
&=
\frac{A }{2 i}
\int_{-\infty}^\infty
\left(
e^{i (k_0 - k) x}
-e^{-i (k_0 + k) x} \right)
dx \\
&=
\frac{A \pi}{i} \left(
\delta(k - k_0) - \delta(k + k_0)
\right).
\end{aligned}
\end{equation}
%
Now we can inverse Fourier transform the diffusion result with ease since we've got delta functions.  That is
\begin{equation}\label{eqn:basicStatMechProblemSet2Problem1:620}
\begin{aligned}
P_{\mathrm{sin}}(x, t) 
&\approx
\inv{2 \pi}
\frac{A \pi}{i}
\int
\left(
\delta(k - k_0) - \delta(k + k_0)
\right)
e^{-D k^2 t} e^{i k x} dk \\
&= A e^{-D k_0^2 t} \frac{ e^{i k_0 x} - e^{-i k_0 x}}{2 i} \\
&= A e^{-D k_0^2 t} \sin( k_0 x ).
\end{aligned}
\end{equation}
%
Plotted in \cref{fig:problemSet2Problem1iiSinePropagator:problemSet2Problem1iiSinePropagatorFig9} we see things die off uniformly, since we didn't start with a windowed sine.  Together, the rectangular function and the sine, after diffusion is plotted in \cref{fig:problemSet2Problem1iiSinePropagator:problemSet2Problem1iiSinePropagatorFig10}.  Because the sine wasn't confined to the window, we see initial time oscillations, but they die off, looking fairly similar to what we saw when the initial conditions were completely zero outside of the window.
% attemptAtProblemSet2Problem1iiIntegrals.nb
\imageFigure{../figures/phy452-basicstatmech/problemSet2Problem1iiSinePropagatorFig9}{Diffusion of (non-windowed) sine.}{fig:problemSet2Problem1iiSinePropagator:problemSet2Problem1iiSinePropagatorFig9}{0.2}
% attemptAtProblemSet2Problem1iiIntegrals.nb
\imageFigure{../figures/phy452-basicstatmech/problemSet2Problem1iiSinePropagatorFig10}{Diffusion of rectangular function and (non-windowed) sine composition.}{fig:problemSet2Problem1iiSinePropagator:problemSet2Problem1iiSinePropagatorFig10}{0.2}
%
My expectation is that we'd get exactly the same result with application of the propagator.  Let's try that.
\begin{equation}\label{eqn:basicStatMechProblemSet2Problem1:640}
\begin{aligned}
P_{\mathrm{sin}}(x, t)
&=
\frac{A}{\sqrt{4 \pi D t}}
\int \sin(k_0 x')
\exp
\mathLabelBox
[
   labelstyle={xshift=1cm, yshift=0.25cm},
   linestyle={out=270,in=90, latex-}
]
{
\left(
-\frac{ ( x' - x)^2}{4 D t}
\right)
}
{\(u = (x' - x)/\sqrt{4 D t}\)}
dx' \\
&=
\frac{A}{\sqrt{\pi}}
\int \sin\lr{
k_0 \lr{\sqrt{ 4 D t} u + x}
}
e^{-u^2}
du \\
&=
\frac{A}{2 i \sqrt{\pi}}
\int
\lr{
e^{
i k_0 \lr{\sqrt{ 4 D t} u + x}
}
-
e^{
-i k_0 \lr{\sqrt{ 4 D t} u + x}
}
}
e^{-u^2}
du \\
&=
\frac{A}{2 i \sqrt{\pi}}
\int
du
\Biglr{ \\
&\quad
   e^{i k_0 x}
   \exp\lr{
      - \lr{u - i k_0 \sqrt{D t}}^2
   }
   \exp\lr{
      \lr{-i k_0 \sqrt{D t}}^2
   } \\
&\quad
   -e^{-i k_0 x}
   \exp\lr{
      - \lr{u + i k_0 \sqrt{D t}}^2
   }
   \exp\lr{
      \lr{i k_0 \sqrt{D t}}^2
   }
}.
\end{aligned}
\end{equation}
%
This is just
\begin{equation}\label{eqn:basicStatMechProblemSet2Problem1:660}
P_{\mathrm{sin}}(x, t)
=
A e^{-k_0^2 D t} \sin( k_0 x).
\end{equation}
%
Observe that we get exactly the same result if we use the propagator on this non-windowed sine.  It is just harder to do the calculation.
%
\paragraph{Given this, I wasn't sure what to make of the grading response:}
%
You cannot use the same Green function/propagator as the first case since the boundary conditions differ.  This is why you have probability density leaking our of \(\pm L/2\) at \(t > 0\).  Better to simply perform the Fourier transform on the initial condition and use the first order differential equation.  e.g. propagate time and inverse Fourier transform back.
%
\paragraph{I'd asked of this:}
%
There was no use of boundary conditions in the derivation of the propagator, at least that I can see.  This appears to be confirmed by using it to calculate the diffusion of the sine portion of the probability density (i.e. \eqnref{eqn:basicStatMechProblemSet2Problem1:620} vs. \eqnref{eqn:basicStatMechProblemSet2Problem1:660}).  I think the leaking out of the probability density at \(t > 0\) is strictly due to the diffusion of the rectangle function.  It has to go somewhere if it diffuses.  Because I applied the propagator to a windowed sine instead of an unbounded one, that leakage of the initial rectangular function is more evident in my first plot, but exists regardless.
%
\paragraph{Eric's response was interesting:}
%
There are two points of contention here, I think:
%
\begin{enumerate}
\item That the same propagator (or Green's function) can be applied equally to both situations,
\item That the sinusoidal decay can be obtained equally well with both the Green's function and the FT/iFT method.
\end{enumerate}
%
Let's focus on 1.  Finding Green's functions for the diffusion equation require specification of boundary conditions (much like how solving linear diff equations, one needs to specify boundary conditions).  For instance, in the ``box'' problem, the Green's function should have the boundary condition of \(G(x,x',t,t') = 0\) if \(\Abs{x}\) or \(\Abs{x'} > L/2\) (recall that Green's function is the density profile at time \(t'\) and location \(x'\) if one were to inject a delta function density profile at time \(t\) and location \(x\)).  The Green's function you found in the ``free'' system case does not satisfy this condition, hence it is the wrong Green's function for the problem.  As you found, performing exact calculations using that propagator results in densities flowing outside the box, which is not possible by definition.
%
Another way to see that the Green's function is wrong: the Green's function you found has translational symmetry.  Namely, \(G(x,x',t,t') = G(x-x',t,t')\).  But imposing the box should break the translational symmetry of the problem, so the Green's function shouldn't have this symmetry any more (unlike the ``free'' system case in part 1).
%
Point here being that it is incorrect to use the same Green's function in both part 1 and part 2 and will lead to wrong answers like densities flowing out of the box.
%
Now the second contention: I agree that if you took the \(L \rightarrow \infty\) limit, the two calculations should give you the same answer.  This is simply because the \(L \rightarrow \infty\) limit the boundaries don't matter any more.  This is not in contradiction with what I just said because we are taking a limiting procedure.
All in all, marks were deducted because densities leaked out of \(\Abs{x} > L/2\), which is unphysical (and if I'm not mistaken, looking at graphs \cref{fig:problemSet2Problem1iiRectPropagator:problemSet2Problem1iiRectPropagatorFig4} and \cref{fig:problemSet2Problem1iiSinePropagator:problemSet2Problem1iiSinePropagatorFig6}, this behavior plagued both the constant and the sinusoidal parts).  I appreciate you solving the problem from a different angle, but the answer simply ended up being qualitatively wrong in that regard.  You did obtain the decay factor which is good and your answer would've been correct if the boundaries of the box didn't exist.  For problems like the box, it is far simpler (as you have found) to perform direct calculations since the boundary conditions are not trivial to into account.
%
\paragraph{My final thoughts.}
%
Because we are ignoring boundary conditions for the sine, treating it as unbounded, is why the Green's function works for that case (even if it isn't correct for the box itself).
%
I don't think there's actually any leakage of probability for the sine portion (I just shouldn't have graphed it outside of the ``box'').  I say this because using this symmetric interval, the sine doesn't contribute any non-zero total probability.  The total area under the sine is zero (as a PV integral \(\lim_{L->\infty} \int_{-L}^L\)), and that's true for any point in time.
%
I'm curious how to calculate a Green's function for the box boundary conditions.  I don't recall encountering one with boundary conditions before this, and obviously the usual Fourier transform method doesn't work.
}
