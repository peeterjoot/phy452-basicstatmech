%
% Copyright � 2013 Peeter Joot.  All Rights Reserved.
% Licenced as described in the file LICENSE under the root directory of this GIT repository.
%
%\input{../blogpost.tex}
%\renewcommand{\basename}{pathriaDiatomic}
%\renewcommand{\dirname}{notes/phy452/}
%\newcommand{\keywords}{Statistical mechanics, PHY452H1S, diatomic molecule gas, Gibbs sum, partition function, pressure, average energy, average occupancy, average diatomic separation, Pathria}
%
%\input{../peeter_prologue_print2.tex}
%
%\beginArtNoToc
%
%\generatetitle{Pathria chapter 4 diatomic molecule problem}
\label{chap:pathriaDiatomic}
%
\makeoproblem{Diatomic molecule}{pr:pathriaDiatomic:4:7}{\citep{pathriastatistical} pr. 4.7}{
Consider a classical system of non-interacting, diatomic molecules enclosed in a box of volume \(V\) at temperature \(T\).  The Hamiltonian of a single molecule is given by
\begin{equation}\label{eqn:pathriaDiatomic:4:7}
H(\Br_1, \Br_2, \Bp_1, \Bp_2) = \inv{2m} \lr{ \Bp_1^2 + \Bp_2^2 }
+\inv{2} K \Abs{\Br_1 - \Br_2}^2.
\end{equation}
%
Study the thermodynamics of this system, including the dependence of the quantity \(\expectation{r_{12}^2}\) on \(T\).
} % makeoproblem
%
\makeanswer{pr:pathriaDiatomic:4:7}{
%
\paragraph{Partition function}
First consider the partition function for a single diatomic pair
\begin{equation}\label{eqn:pathriaDiatomic:27}
\begin{aligned}
Z_1
&=
\inv{h^6}
\int d^6 \Bp d^6 \Br
e^{-\beta
\frac{ \Bp_1^2 + \Bp_2^2 }{2m}
}
e^{-\beta K
\frac{ \Abs{\Br_1 - \Br_2}^2 }{2}
} \\
&=
\inv{h^6}
\lr{\frac{2 \pi m}{\beta}}^{6/2}
\int d^3 \Br_1 d^3 \Br_2
e^{-\beta K
\frac{ \Abs{\Br_1 - \Br_2}^2 }{2}.
}
\end{aligned}
\end{equation}
%
Now we can make a change of variables to simplify the exponential.  Let's write
\begin{subequations}
\begin{equation}\label{eqn:pathriaDiatomic:47}
\Bu = \Br_1 - \Br_2
\end{equation}
\begin{equation}\label{eqn:pathriaDiatomic:67}
\Bv = \Br_2,
\end{equation}
\end{subequations}
or
\begin{subequations}
\begin{equation}\label{eqn:pathriaDiatomic:87}
\Br_2 = \Bv
\end{equation}
\begin{equation}\label{eqn:pathriaDiatomic:107}
\Br_1
=
\Bu + \Bv.
\end{equation}
\end{subequations}
%
Our volume element is
\begin{equation}\label{eqn:pathriaDiatomic:127}
d^3 \Br_1 d^3 \Br_2
= d^3 \Bu d^3 \Bv \frac{\partial(\Br_1, \Br_2)}{\partial(\Bu, \Bv)}.
\end{equation}
%
It wasn't obvious to me that this change of variables preserves the volume element, but a quick calculation shows this to be the case
\begin{equation}\label{eqn:pathriaDiatomic:147}
\begin{aligned}
\frac{\partial(\Br_1, \Br_2)}{\partial(\Bu, \Bv)}
&=
\begin{vmatrix}
\PDi{u_{1}}{r_{11}} &\PDi{u_{2}}{r_{11}} &\PDi{u_{3}}{r_{11}} &\PDi{v_{1}}{r_{11}} &\PDi{v_{2}}{r_{11}} &\PDi{v_{3}}{r_{11}} \\
\PDi{u_{1}}{r_{12}} &\PDi{u_{2}}{r_{12}} &\PDi{u_{3}}{r_{12}} &\PDi{v_{1}}{r_{12}} &\PDi{v_{2}}{r_{12}} &\PDi{v_{3}}{r_{12}} \\
\PDi{u_{1}}{r_{13}} &\PDi{u_{2}}{r_{13}} &\PDi{u_{3}}{r_{13}} &\PDi{v_{1}}{r_{13}} &\PDi{v_{2}}{r_{13}} &\PDi{v_{3}}{r_{13}} \\
\PDi{u_{1}}{r_{21}} &\PDi{u_{2}}{r_{21}} &\PDi{u_{3}}{r_{21}} &\PDi{v_{1}}{r_{21}} &\PDi{v_{2}}{r_{21}} &\PDi{v_{3}}{r_{21}} \\
\PDi{u_{1}}{r_{22}} &\PDi{u_{2}}{r_{22}} &\PDi{u_{3}}{r_{22}} &\PDi{v_{1}}{r_{22}} &\PDi{v_{2}}{r_{22}} &\PDi{v_{3}}{r_{22}} \\
\PDi{u_{1}}{r_{23}} &\PDi{u_{2}}{r_{23}} &\PDi{u_{3}}{r_{23}} &\PDi{v_{1}}{r_{23}} &\PDi{v_{2}}{r_{23}} &\PDi{v_{3}}{r_{23}}
\end{vmatrix} \\
&=
\begin{vmatrix}
1 & 0 & 0 & 1 & 0 & 0 \\
0 & 1 & 0 & 0 & 1 & 0 \\
0 & 0 & 1 & 0 & 0 & 1 \\
0 & 0 & 0 & 1 & 0 & 0 \\
0 & 0 & 0 & 0 & 1 & 0 \\
0 & 0 & 0 & 0 & 0 & 1
\end{vmatrix} \\
&= 1.
\end{aligned}
\end{equation}
%
Our remaining integral can now be evaluated
\begin{equation}\label{eqn:pathriaDiatomic:167}
\begin{aligned}
\int d^3 \Br_1 d^3 \Br_2
e^{-\beta K
\frac{ \Abs{\Br_1 - \Br_2}^2 }{2}
}
&= \int d^3 \Bu d^3 \Bv e^{-\beta K \Abs{\Bu}^2 /2 } \\
&= V \int d^3 \Bu e^{-\beta K \Abs{\Bu}^2 /2 } \\
&= V \int d^3 \Bu e^{-\beta K \Abs{\Bu}^2 /2 } \\
&= V \lr{ \frac{ 2 \pi }{ K \beta } }^{3/2}.
\end{aligned}
\end{equation}
%
Our partition function is now completely evaluated
\begin{equation}\label{eqn:pathriaDiatomic:187}
Z_1 =
V
\inv{h^6}
\lr{\frac{2 \pi m}{\beta}}^{3}
\lr{ \frac{ 2 \pi }{ K \beta } }^{3/2}.
\end{equation}
%
As a function of \(V\) and \(T\) as in the text, we write
\begin{subequations}
\begin{equation}\label{eqn:pathriaDiatomic:207}
Z_1 = V f(T)
\end{equation}
\begin{equation}\label{eqn:pathriaDiatomic:227}
f(T) =
\lr{\frac{m }{h^2 } \sqrt{\frac{(2\pi)^3}{K}} }^3
\lr{\kB T}^{9/2}.
\end{equation}
\end{subequations}
%
\paragraph{Gibbs sum}
%
Our Gibbs sum, summing over the number of molecules (not atoms), is
\begin{equation}\label{eqn:pathriaDiatomic:247}
\ZG = \sum_{N_r = 0}^\infty \frac{z^{N_r}}{N_r!} Z_1^{N_r} = e^{ z V f(T) },
\end{equation}
or
\begin{equation}\label{eqn:pathriaDiatomic:267}
q = \ln \ZG = z V f(T) = P V \beta.
\end{equation}
%
The fact that we can sum this as an exponential series so nicely looks like it's one of the main advantages to this grand partition function (Gibbs sum).  We can avoid any of the large \(N!\) approximations that we have to use when the number of particles is explicitly fixed.
%
\paragraph{Pressure}
%
The pressure follows
\begin{equation}\label{eqn:pathriaDiatomic:287}
P = z f(T) \kB T
=
e^{\mu/\kB T}
\lr{\frac{m }{h^2 } \sqrt{\frac{(2\pi)^3}{K}} }^3
\lr{\kB T}^{11/2}.
\end{equation}
%
\paragraph{Average energy}
\begin{equation}\label{eqn:pathriaDiatomic:307}
\begin{aligned}
\expectation{H}
&= -\PD{\beta}{q} \\
&= - z V \frac{9}{2} \frac{f(T)}{T} \PD{\beta}{T} \\
&= z V \frac{9}{2} \frac{f(T)}{T^3} \inv{\kB},
\end{aligned}
\end{equation}
%
or
\begin{equation}\label{eqn:pathriaDiatomic:327}
\expectation{H}
=
e^{\mu/\kB T}
V \frac{9}{2} \kB^2
\lr{\frac{m }{h^2 } \sqrt{\frac{(2\pi)^3}{K}} }^3
\lr{\kB T}^{3/2}.
\end{equation}
%
\paragraph{Average occupancy}
\begin{equation}\label{eqn:pathriaDiatomic:347}
\begin{aligned}
\expectation{N}
&= z \PD{z}{} \ln \ZG \\
&= z \PD{z}{} \lr{ z V f(T) } \\
&= z V f(T),
\end{aligned}
\end{equation}
but this is just \(q\), or
\begin{equation}\label{eqn:pathriaDiatomic:367}
\expectation{N}
=
e^{\mu/\kB T}
V
\lr{\frac{m }{h^2 } \sqrt{\frac{(2\pi)^3}{K}} }^3
\lr{\kB T}^{9/2}.
\end{equation}
%
\paragraph{Free energy}
\begin{equation}\label{eqn:pathriaDiatomic:387}
\begin{aligned}
F
&= - \kB T \ln \frac{ \ZG }{z^N} \\
&= - \kB T \lr{ q - N \ln z } \\
&= N \kB T \beta \mu - \kB T q \\
&= z V f(T) \mu - \kB T z V f(T) \\
&= z V f(T) \lr{ \mu - \kB T }.
\end{aligned}
\end{equation}
\begin{equation}\label{eqn:pathriaDiatomic:407}
F
=
e^{\mu/\kB T}
V
\lr{ \mu - \kB T }
\lr{\frac{m }{h^2 } \sqrt{\frac{(2\pi)^3}{K}} }^3
\lr{\kB T}^{9/2}.
\end{equation}
%
\paragraph{Entropy}
\begin{equation}\label{eqn:pathriaDiatomic:427}
\begin{aligned}
S
&= \frac{U - F}{T} \\
&=
\frac{V}{T}
e^{\mu/\kB T}
\lr{\kB T}^{3/2}
\lr{\frac{m }{h^2 } \sqrt{\frac{(2\pi)^3}{K}} }^3
\lr{
\frac{9}{2} \kB^2
-
\lr{ \mu - \kB T }
\lr{\kB T}^{3}
}
.
\end{aligned}
\end{equation}
%
\paragraph{Expectation of atomic separation}
%
The momentum portions of the average will just cancel out, leaving just
\begin{equation}\label{eqn:pathriaDiatomic:447}
\begin{aligned}
\expectation{r_{12}^2}
&= \frac{ \int d^3 \Br_1 d^3 \Br_2 \lr{\Br_1 - \Br_2}^2 e^{-\beta K \lr{\Br_1 - \Br_2}^2 /2 } }{ \int d^3 \Br_1 d^3 \Br_2 e^{-\beta K \lr{\Br_1 - \Br_2}^2 /2 } } \\
&= \frac{ \int d^3 \Bu \Bu^2 e^{-\beta K \Bu^2 /2 } }{ \int d^3 \Bu e^{-\beta K \Bu^2 /2 } } \\
&= \frac{ \int da db dc \lr{ a^2 + b^2 + c^2} e^{-\beta K \lr{a^2 + b^2 + c^2} /2 } }{ \int e^{-\beta K \lr{a^2 + b^2 + c^2} /2 } } \\
&= 3 \frac{ \int da a^2 e^{-\beta K a^2/2} \int db dc e^{-\beta K \lr{b^2 + c^2} /2 } }{ \int e^{-\beta K \lr{a^2 + b^2 + c^2} /2 } } \\
&= 3 \frac{ \int da a^2 e^{-\beta K a^2/2} }{ \int e^{-\beta K a^2/2} }.
\end{aligned}
\end{equation}
%
Expanding the numerator by parts we have
\begin{equation}\label{eqn:pathriaDiatomic:467}
\begin{aligned}
\int da a^2 e^{-\beta K a^2/2}
&= \int a d\lr{\frac{ e^{-\beta K a^2/2} } { - 2 \beta K/2 } } \\
&= \frac{1}{\beta K} \int e^{-\beta K a^2/2}.
\end{aligned}
\end{equation}
%
This gives us
\boxedEquation{eqn:pathriaDiatomic:487}{
\expectation{r_{12}^2}
=
\frac{3}{\beta K}
=
\frac{3 \kB T}{K}.
}
%
This result doesn't make sense as \(T \rightarrow 0\).  We need a QM treatment that will prevent the interatomic spacing in the molecule spacing from hitting zero at low temperatures.  Is this valid for very high temperatures too?  Does this linear temperature dependence model the bond breaking that would occur when enough energy is supplied to the molecule?
} % makeanswer
%
%\EndArticle
