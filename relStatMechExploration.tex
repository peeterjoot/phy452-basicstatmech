%
% Copyright � 2013 Peeter Joot.  All Rights Reserved.
% Licenced as described in the file LICENSE under the root directory of this GIT repository.
%
%\input{../blogpost.tex}
%\renewcommand{\basename}{relStatMechExploration}
%\renewcommand{\dirname}{notes/phy452/}
%\newcommand{\keywords}{Statistical mechanics, PHY452H1S, special relativity, four vector, Boltzmann factor, energy, momentum, four momentum, microstate, equilibrium, temperature, reservoir, subsystem, Pathria}
%
%\input{../peeter_prologue_print2.tex}
%
%\beginArtNoToc
%
%\generatetitle{Relativistic generalization of statistical mechanics}
\label{chap:relStatMechExploration}

\paragraph{Motivation}

I was wondering how to generalize the arguments of \citep{pathriastatistical} to relativistic systems.  Here's a bit of blundering through the non-relativistic arguments of that text, tweaking them slightly.

I'm sure this has all been done before, but was a useful exercise to understand the non-relativistic arguments of Pathria better.

\paragraph{Generalizing from energy to four momentum}

Generalizing the arguments of \S 1.1.

Instead of considering that the total energy of the system is fixed, it makes sense that we'd have to instead consider the total four-momentum of the system fixed, so if we have \(N\) particles, we have a total four momentum
\begin{equation}\label{eqn:relStatMechExploration:20}
P = \sum_i n_i P_i = \sum n_i \lr{ \epsilon_i/c, \Bp_i },
\end{equation}
where \(n_i\) is the total number of particles with four momentum \(P_i\).  We can probably expect that the \(n_i\)'s in this relativistic system will be smaller than those in a non-relativistic system since we have many more states when considering that we can have both specific energies and specific momentum, and the combinatorics of those extra degrees of freedom.  However, we'll still have
\begin{dmath}\label{eqn:relStatMechExploration:100}
N = \sum_i n_i.
\end{dmath}

Only given a specific observer frame can these these four-momentum components \(\lr{\epsilon_i/c, \Bp_i}\) be expressed explicitly, as in
\begin{subequations}
\begin{dmath}\label{eqn:relStatMechExploration:40}
\epsilon_i = \gamma_i m_i c^2
\end{dmath}
\begin{dmath}\label{eqn:relStatMechExploration:60}
\Bp_i = \gamma_i m \Bv_i
\end{dmath}
\begin{dmath}\label{eqn:relStatMechExploration:80}
\gamma_i = \inv{\sqrt{1 - \Bv_i^2/c^2}},
\end{dmath}
\end{subequations}
where \(\Bv_i\) is the velocity of the particle in that observer frame.

\paragraph{Generalizing the number if microstates, and notion of thermodynamic equilibrium}

Generalizing the arguments of \S 1.2.

We can still count the number of all possible microstates, but that number, denoted \(\Omega(N, V, E)\), for a given total energy needs to be parameterized differently.  First off, any given volume is observer dependent, so we likely need to map
\begin{dmath}\label{eqn:relStatMechExploration:120}
V \rightarrow \int d^4 x = \int dx^0 \wedge dx^1 \wedge dx^2 \wedge dx^3.
\end{dmath}

Let's still call this \(V\), but know that we mean this to be four volume element, bounded in both space and time, referred to a fixed observer's frame.  So, lets write the total number of microstates as
\begin{dmath}\label{eqn:relStatMechExploration:140}
\Omega(N, V, P) = \Omega \lr{ N, \int d^4 x, E/c, P^1, P^2, P^3},
\end{dmath}

where \(P = \lr{E/c, \BP}\) is the total four momentum of the system.  If we have a system subdivided into to two systems in contact as in \cref{fig:relStatMechExploration:relStatMechExplorationFig1}, where the two systems have total four momentum \(P_1\) and \(P_2\) respectively.

\imageFigure{../figures/phy452-basicstatmech/relStatMechExplorationFig1}{Two physical systems in thermal contact.}{fig:relStatMechExploration:relStatMechExplorationFig1}{0.2}

In the text the total energy of both systems was written
\begin{dmath}\label{eqn:relStatMechExploration:160}
E^{(0)} = E_1 + E_2,
\end{dmath}
so we'll write
\begin{equation}\label{eqn:relStatMechExploration:180}
{P^{(0)}}^\mu = P_1^\mu + P_2^\mu = \text{constant},
\end{equation}
so that the total number of microstates of the combined system is now
\begin{dmath}\label{eqn:relStatMechExploration:200}
\Omega^{(0)}(P_1, P_2) = \Omega_1(P_1) \Omega_2(P_2).
\end{dmath}

As before, if \(\overbar{P}^\mu_i\) denotes an equilibrium value of \(P_i^\mu\), then maximizing \eqnref{eqn:relStatMechExploration:200} requires all the derivatives (no sum over \(\mu\) here)
\begin{dmath}\label{eqn:relStatMechExploration:220}
\PDc{P^\mu_1}{\Omega_1(P_1)}{P_1 = \overbar{P_1}}
\Omega_2(\overbar{P}_2)
+
\Omega_1(\overbar{P}_1)
\PDc{P^\mu}{\Omega_2(P_2)}{P_2 = \overbar{P_2}}
\times
\PD{P_1^\mu}{P_2^\mu}
= 0.
\end{dmath}

With each of the components of the total four-momentum \(P^\mu_1 + P^\mu_2\) separately constant, we have \(\PDi{P_1^\mu}{P_2^\mu} = -1\), so that we have
\begin{dmath}\label{eqn:relStatMechExploration:240}
\PDc{P^\mu_1}{\ln \Omega_1(P_1)}{P_1 = \overbar{P_1}}
=
\PDc{P^\mu}{\ln \Omega_2(P_2)}{P_2 = \overbar{P_2}},
\end{dmath}
as before.  However, we now have one such identity for each component of the total four momentum \(P\) which has been held constant.  Let's now define
\begin{dmath}\label{eqn:relStatMechExploration:260}
\beta_\mu \equiv \PDc{P^\mu}{\ln \Omega(N, V, P)}{N, V, P = \overbar{P}}.
\end{dmath}

Our old scalar temperature is then
\begin{dmath}\label{eqn:relStatMechExploration:280}
\beta_0 = c \PDc{E}{\ln \Omega(N, V, P)}{N, V, P = \overbar{P}} = c \beta = \frac{c}{\kB T},
\end{dmath}
but now we have three additional such constants to figure out what to do with.  A first start would be figuring out how the Boltzmann probabilities should be generalized.

\paragraph{Equilibrium between a system and a heat reservoir}

Generalizing the arguments of \S 3.1.

As in the text, let's consider a very large heat reservoir \(A'\) and a subsystem \(A\) as in \cref{fig:relStatMechExploration:relStatMechExplorationFig2} that has come to a state of mutual equilibrium.  This likely needs to be defined as a state in which the four vector \(\beta_\mu\) is common, as opposed to just \(\beta_0\) the temperature field being common.

\imageFigure{../figures/phy452-basicstatmech/relStatMechExplorationFig2}{A system \(A\) immersed in heat reservoir \(A'\).}{fig:relStatMechExploration:relStatMechExplorationFig2}{0.2}

If the four momentum of the heat reservoir is \(P_r'\) with \(P_r\) for the subsystem, and
\begin{equation}\label{eqn:relStatMechExploration:300}
P_r + P_r' = P^{(0)} = \text{constant}.
\end{equation}

Writing
\begin{equation}\label{eqn:relStatMechExploration:320}
\Omega'({P^\mu_r}') = \Omega'(P^{(0)} - {P^\mu_r}) \propto P_r,
\end{equation}
for the number of microstates in the reservoir, so that a Taylor expansion of the logarithm around \(P_r' = P^{(0)}\) (with sums implied) is
\begin{dmath}\label{eqn:relStatMechExploration:340}
\ln \Omega'({P^\mu_r}') =
\ln \Omega'({P^{(0)}})
+
\PDc{{P^\mu}'}{\ln \Omega'}{P' = P^{(0)}} \lr{ P^{(0)} - P^\mu}
\approx
\text{constant} - \beta_\mu' P^\mu.
\end{dmath}

Here we've inserted the definition of \(\beta^\mu\) from \eqnref{eqn:relStatMechExploration:260}, so that at equilibrium, with \(\beta_\mu' = \beta_\mu\), we obtain
\begin{equation}\label{eqn:relStatMechExploration:360}
\Omega'({P^\mu_r}') =
\exp\lr{
- \beta_\mu P^\mu
}
=
\exp\lr{
- \beta E
}
\exp\lr{
- \beta_1 P^1
}
\exp\lr{
- \beta_2 P^3
}
\exp\lr{
- \beta_3 P^3
}.
\end{equation}

\paragraph{Next steps}

This looks consistent with the outline provided in \href{http://physics.stackexchange.com/a/4950/3621}{http://physics.stackexchange.com/a/4950/3621} by Lubos to the stackexchange ``is there a relativistic quantum thermodynamics'' question.
Looking for references on these topics I found \citep{synge1957relativistic}, which shows that I got the Boltzmann factor correct.  However, it looks like study of relativistic fluid mechanics would be a good first step before attempting to further study this text.  It is also formulated using the old Minkowski (imaginary time) style relativistic notation, which requires a bit of ``translation'' when reading.   A more modern treatment can be found in \citep{hakim2011introduction}, but this is a very inimidating text, with currents and the energy momentum tensor written in delta function current notation.

It would be interesting to explore these to look for an explaination of why non-relativistic stat mech can be used for photon problems, and also to find examples of what sort of problems this relativistic approach is required for.  The domain of application for this relativistic theory appears to be cosmological (and perhaps also nuclear and high energy systems).

%\EndArticle
