%
% Copyright © 2016 Peeter Joot.  All Rights Reserved.
% Licenced as described in the file LICENSE under the root directory of this GIT repository.
%
\makeoproblem{Ergodicity in 1D system}{pr:midterm:4}{2013 midterm pr 4}
{
Consider a system with \(N\) particles constrained to one dimension.  Is this system ergodic?
} %makeoproblem

\makeanswer{pr:midterm:4}{

A portion of this system is illustrated in \cref{fig:lecture8:lecture8Fig3}.

\imageFigure{../figures/phy452-basicstatmech/lecture8Fig3}{One dimensional collision of particles.}{fig:lecture8:lecture8Fig3}{0.15}

%The statement here that the collision was in 1D was a hint that we can actually calculate the result.

Start with a pair of collisions and work out that the velocities are exchanged.  
\begin{equation}\label{eqn:midterm1review:160}
(v_i, v_{i+1}) \rightarrow (v_{i+1}, v_{i})
\end{equation}

This is verified in \nbref{midtermQ4twoEqualMassesCollision.nb}, but is also easy to show.  Our momentum and energy conservation relationships, for equal masses, are

\begin{subequations}
\begin{equation}\label{eqn:midterm1review:180}
v_1 + v_2 = w_1 + w_2,
\end{equation}
\begin{equation}\label{eqn:midterm1review:200}
v_1^2 + v_2^2 = w_1^2 + w_2^2.
\end{equation}
\end{subequations}

Solving for \(w_1\) we have
\begin{dmath}\label{eqn:midterm1review:220}
v_1^2 + v_2^2 
= w_1^2 + (v_1 + v_2 - w_1)^2
= 2 w_1^2 + v_1^2 + v_2^2 + 2 v_1 v_2 + 2 (v_1 + v_2)w_1,
\end{dmath}
or
\begin{dmath}\label{eqn:3nParticlePhaseSpaceVolume:240}
0 
= \lr{ w_1 - \frac{v_1 + v_2}{2}}^2 
+ v_1 v_2 
- \lr{\frac{v_1 + v_2}{2}}^2
= \lr{ w_1 - \frac{v_1 + v_2}{2}}^2 
+ 
\frac{4 v_1 v_2 - v_1^2 - v_2^2 - 2 v_1 v_2}{4}
= \lr{ w_1 - \frac{v_1 + v_2}{2}}^2 
- 
\frac{
\lr{v_1 - v_2}^2
}{4}.
\end{dmath}

This gives us

\begin{equation}\label{eqn:midterm1review:260}
w_1 = \frac{v_1 + v_2}{2} \pm \Abs{\frac{v_1 - v_2}{2}}.
\end{equation}

This yields solutions

\begin{equation}\label{eqn:midterm1review:280}
\{w_1, w_2\} \in \{v_1, v_2\}, \{v_2, v_1\}.
\end{equation}

The first corresponds to conservation of energy and momentum prior to the collision, and the second, the post collision event where we have an exchange of velocities.  This won't result in all the phase space being explored (i.e. a non-\textAndIndex{ergodic} process), and was meant to show that things are extremely restrictive in 1D.
} % makeanswer
