%
% Copyright � 2013 Peeter Joot.  All Rights Reserved.
% Licenced as described in the file LICENSE under the root directory of this GIT repository.
%
%\input{../blogpost.tex}
%\renewcommand{\basename}{chemicalPotentialLowTempPathria}
%\renewcommand{\dirname}{notes/phy452/}
%\newcommand{\keywords}{Statistical mechanics, PHY452H1S, Fermi gas, chemical potential, Fermi energy, low temperature}
%
%\input{../peeter_prologue_print2.tex}
%
%\beginArtNoToc
%
%\generatetitle{Low temperature Fermi gas chemical potential}
%\chapter{Low temperature Fermi gas chemical potential}
\label{chap:chemicalPotentialLowTempPathria}
%
\makeproblem{Low temperature Fermi gas \textAndIndex{chemical potential}}{pr:chemicalPotentialLowTempPathria:1}{
%
\citep{pathriastatistical} \S 8.1 equation (33) provides an implicit function for \(\mu \equiv \kB T \ln z\)
\begin{equation}\label{eqn:chemicalPotentialLowTempPathria:20}
n = \frac{4 \pi g}{3}
\lr{ \frac{2m}{h^2} }
^{3/2}
\mu
^{3/2}
\lr{ 1 + \frac{\pi^2}{8} \frac{ (\kB T)^2 }{ \mu^2 } + \cdots },
\end{equation}
or
\begin{equation}\label{eqn:chemicalPotentialLowTempPathria:40}
\EF^{3/2} = \mu^{3/2}
\lr{ 1 + \frac{\pi^2}{8} \frac{ (\kB T)^2 }{ \mu^2 } + \cdots}.
\end{equation}
%
In class, we assumed that \(\mu\) was quadratic in \(\kB T\) as a mechanism to invert this non-linear equation.  Without making this quadratic assumption find the lowest order, non-constant approximation for \(\mu(T)\).
} % makeproblem
%
\makeanswer{pr:chemicalPotentialLowTempPathria:1}{
%
To determine an approximate inversion, let's start by multiplying \eqnref{eqn:chemicalPotentialLowTempPathria:40} by \(\mu^{1/2}/\EF^2\) to non-dimensionalize things
\begin{equation}\label{eqn:chemicalPotentialLowTempPathria:80}
\lr{ \frac{\mu}{\EF} }
^{1/2} =
\lr{ \frac{\mu}{\EF}}^2
+ \frac{\pi^2}{8}
\lr{\frac{\kB T}{\EF}}^2,
\end{equation}
%
or
\begin{equation}\label{eqn:chemicalPotentialLowTempPathria:100}
\lr{ \frac{\mu}{\EF} }
^{1/2}
=
\inv{ 1 -
\lr{ \frac{\mu}{\EF}}
^{3/2} }
\frac{\pi^2}{8}
\lr{\frac{\kB T}{\EF}}^2.
\end{equation}
If we are looking for an approximation in the neighborhood of \(\mu = \EF\), then the LHS factor is approximately one, whereas the fractional difference term is large (with a corresponding requirement for \(\kB T/\EF\) to be small.  We must then have
\begin{equation}\label{eqn:chemicalPotentialLowTempPathria:120}
\lr{ \frac{\mu}{\EF}}
^{3/2}
\approx
1 - \frac{\pi^2}{8}
\lr{\frac{\kB T}{\EF}}^2,
\end{equation}
or
\begin{equation}\label{eqn:chemicalPotentialLowTempPathria:140}
\begin{aligned}
\mu
&\approx
\EF
\left(
1 - \frac{\pi^2}{8}
\lr{\frac{\kB T}{\EF}}^2
\right)^{2/3} \\
&\approx
\EF
\left(
1 - \frac{2}{3} \frac{\pi^2}{8}
\lr{\frac{\kB T}{\EF}}^2
\right).
\end{aligned}
\end{equation}
%
This gives us the desired result
\boxedEquation{eqn:midterm2reflection:160}{
\mu \approx
\EF
\left(
1 - \frac{\pi^2}{12}
\lr{\frac{\kB T}{\EF}}^2
\right).
}
} % makeanswer
%
%\EndArticle
