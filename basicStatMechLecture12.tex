%
% Copyright � 2013 Peeter Joot.  All Rights Reserved.
% Licenced as described in the file LICENSE under the root directory of this GIT repository.
%
%\input{../blogpost.tex}
%\renewcommand{\basename}{basicStatMechLecture12}
%\renewcommand{\dirname}{notes/phy452/}
%\newcommand{\keywords}{Statistical mechanics, PHY452H1S, partition function, Helmholtz free energy, entropy, Spin Hamiltonian, trace}
%\input{../peeter_prologue_print2.tex}
%
%\beginArtNoToc
%\generatetitle{PHY452H1S Basic Statistical Mechanics.  Lecture 12: Helmholtz free energy.  Taught by Prof.\ Arun Paramekanti}
%\label{chap:basicStatMechLecture12}
%
\section{Canonical partition and Helmholtz free energy.}
%
We found
\begin{subequations}
\begin{equation}\label{eqn:basicStatMechLecture12:20}
\frac{\sigma_{\txtE}}{E} \propto \frac{T \sqrt{\CV}}{E} \kB^2
\end{equation}
\begin{equation}\label{eqn:basicStatMechLecture12:40}
Z = \sum_{\{c\}} e^{-\beta E(c)}
\end{equation}
\begin{equation}\label{eqn:basicStatMechLecture12:60}
\CV \sim N
\end{equation}
\begin{equation}\label{eqn:basicStatMechLecture12:80}
E \sim N,
\end{equation}
\end{subequations}
where the \textAndIndex{partition function} acts as a probability distribution so that we can define an average as
\begin{equation}\label{eqn:basicStatMechLecture12:100}
\expectation{A} =
\frac{\sum_{\{c\}} A(c) e^{-\beta E(c)}}{Z}.
\end{equation}
If we suppose that the energy is typically close to the average energy as in \cref{fig:lecture12:lecture12Fig1}.
\imageFigure{../figures/phy452-basicstatmech/lecture12Fig1}{Peaked energy distribution.}{fig:lecture12:lecture12Fig1}{0.2}
, then we can approximate the partition function as
\begin{dmath}\label{eqn:basicStatMechLecture12:120a}
Z
\approx e^{-\beta \expectation{E}} \sum_{\{c\}} \delta_{E, \overbar{E}}
= e^{-\beta \expectation{E}} e^{S/\kB},
\end{dmath}
where we've used \(S = \kB \ln \Omega\) to express the number of states where the energy matches the average energy \(\Omega = \sum \delta_{E, \overbar{E}}\).
%
This gives us
\begin{dmath}\label{eqn:basicStatMechLecture12:120}
Z
= e^{-\beta (\expectation{E} - \kB T S/\kB) }
= e^{-\beta (\expectation{E} - T S) },
\end{dmath}
or
\boxedEquation{eqn:basicStatMechLecture12:140}{
Z
= e^{-\beta F},
}
where we define the \underlineAndIndex{Helmholtz free energy} \(F\) as
\boxedEquation{eqn:basicStatMechLecture12:160}{
F = \expectation{E} - T S.
}
%
Equivalently, the log of the partition function provides us with the partition function
\begin{equation}\label{eqn:basicStatMechLecture12:180}
F = - \kB T \ln Z.
\end{equation}
%
Recalling our expression for the average energy, we can now write that in terms of the free energy
\begin{dmath}\label{eqn:basicStatMechLecture12:200}
\expectation{E} =
\frac{\sum_{\{c\}} E(c) e^{-\beta E(c)}}
{\sum_{\{c\}} e^{-\beta E(c)}}
=
-
\PD{\beta}{}\ln Z
=
\PD{\beta}{(\beta F)}.
\end{dmath}
%
\section{Quantum mechanical picture.}
%
Consider a subsystem as in \cref{fig:lecture12:lecture12Fig2} where we have states of the form
\imageFigure{../figures/phy452-basicstatmech/lecture12Fig2}{subsystem in heat bath.}{fig:lecture12:lecture12Fig2}{0.2}
\begin{equation}\label{eqn:basicStatMechLecture12:220}
\ket{\Psi_{\text{full}}} = \ket{\chi_{\text{subsystem}}} \ket{\phi_{\text{bath}}},
\end{equation}
and a total Hamiltonian operator of the form
\begin{equation}\label{eqn:basicStatMechLecture12:240}
H_{\text{full}} = H_{\text{subsystem}} + H_{\text{bath}} (+ H_{\text{coupling}}),
\end{equation}
where the total energy of the state, given energy eigenvalues \(\calE_n\) and \(\lambda_n\) for the states \(\ket{\chi_{\text{subsystem}}}\) and \(\ket{\phi_{\text{bath}}}\) respectively, is given by the sum
\begin{equation}\label{eqn:basicStatMechLecture12:260}
E = \calE_m + \lambda_n.
\end{equation}
%
Here \(\calE_m, \lambda_n\) are many body energies, so that \(\delta E \sim \#e^{-\#N}\).
%
We can now write the total number of states as
\begin{dmath}\label{eqn:basicStatMechLecture12:280}
\Omega(E) =
\mathLabelBox{
\sum_m
}{subsystem}
\mathLabelBox
[
   labelstyle={below of=m\themathLableNode, below of=m\themathLableNode}
]
{
\sum_n
}{bath}
\delta(E - \calE_m -\lambda_n)
=
\sum_m e^{\inv{\kB} S(E - \calE_m)}
\approx
\sum_m
e^{\inv{\kB} S(E)}
e^{-\beta \calE_m}
\end{dmath}
\begin{dmath}\label{eqn:basicStatMechLecture12:300}
Z
= \sum_m e^{-\beta \calE_m}
= \tr \lr{ e^{-\beta \hat{H}_{\text{subsystem}}} }.
\end{dmath}
We've ignored the coupling term in \eqnref{eqn:basicStatMechLecture12:240}.  This is actually a problem in quantum mechanics since we require this coupling to introduce state changes.
%
\makeexample{Spins}{example:basicStatMechLecture12:1}{
Given \(N\) spin \(1/2\) objects \(\uparrow\), \(\downarrow\), satisfying
\begin{dmath}\label{eqn:basicStatMechLecture12:320}
S_z = \pm \inv{2} \Hbar,
\end{dmath}
where \(S_z\) is the magnitude of the spin operator
\begin{dmath}\label{eqn:basicStatMechLecture12:321}
\hat{S}_z = \pm \inv{2} \Hbar \hat{\sigma}_z,
\end{dmath}
and \(\hat{\sigma}_z\) is the Pauli matrix.  Dropping \(\Hbar\) we have
\begin{dmath}\label{eqn:basicStatMechLecture12:340}
S_z \rightarrow \pm \inv{2} \sigma.
\end{dmath}
%
Our system has a state \(\ket{\sigma_1, \sigma_2, \cdots \sigma_N}\) where \(\sigma_i = \pm 1\).  The total number of states is \(2^N\).
%
Recall from \citep{desai2009quantum} \S 6.6 that \(\hat{H} = (\Bsigma \cdot \BB)^2/2m\) and a gauge transformation \(\Bp \rightarrow \Bp - e \BA/c\) results in an interaction term \(-e \Hbar \BS \cdot \BB/m c\).  With \(e = \Hbar = c = 1\) and \(\BB = B \zcap\), the portion of the Hamiltonian for just that spin interaction is
\begin{equation}\label{eqn:basicStatMechLecture12:360}
\hat{H} = - B \sum_i \hat{S}_{z_i}.
\end{equation}
%
This is the associated with the Zeeman effect, where states can be split by a magnetic field, as in \cref{fig:lecture12:lecture12Fig3}.
%
\imageFigure{../figures/phy452-basicstatmech/lecture12Fig3}{Zeeman splitting.}{fig:lecture12:lecture12Fig3}{0.15}
%
Our minimum and maximum energies are
\begin{subequations}
\begin{equation}\label{eqn:basicStatMechLecture12:380}
E_{\mathrm{min}} = -\frac{B}{2} \sum_{i = 1}^N (+1) = -\frac{B}{2} N
\end{equation}
\begin{equation}\label{eqn:basicStatMechLecture12:400}
E_{\mathrm{max}} = \frac{B}{2} \sum_{i = 1}^N (-1) = \frac{B}{2} N.
\end{equation}
\end{subequations}
%
The total energy difference is
\begin{equation}\label{eqn:basicStatMechLecture12:420}
\Delta E = B N,
\end{equation}
and the energy differences are
\begin{equation}\label{eqn:basicStatMechLecture12:440}
\delta E \sim \frac{B N}{2^N} \sim \# e^{-\# N}.
\end{equation}
%
FIXME: where did this exponential come from?
%
This is a measure of the average energy difference between two adjacent energy levels.  In a real system we cannot assume that we have non-interacting spins.  Any weak interaction will split our degenerate energy levels as in \cref{fig:lecture12:lecture12Fig4}.
%
\imageFigure{../figures/phy452-basicstatmech/lecture12Fig4}{Interaction splitting.}{fig:lecture12:lecture12Fig4}{0.2}
%
We can now express the partition function
\begin{dmath}\label{eqn:basicStatMechLecture12:460}
Z
= \sum_{\{\sigma\}} e^{-\beta
\lr{ -\frac{B}{2} \sum_i \sigma_i}
}
=
\mathLabelBox
[
   labelstyle={xshift=2cm},
   linestyle={out=270,in=90, latex-}
]
{
\lr{
\sum_{\sigma_1 \in \{-1,1\}}
\exp \lr{ -\frac{\beta B}{2} \sigma_1}
}
\lr{
\sum_{\sigma_2 \in \{-1,1\}}
\exp \lr{ -\frac{\beta B}{2} \sigma_2}
}
\cdots
}{\(N\) times.}
=
\lr{
\exp \lr{ -\frac{\beta B}{2} (1)}
+
\exp \lr{ -\frac{\beta B}{2} (-1)}
}^N
=
\lr{ 2 \cosh\lr{ \frac{B}{2 \kB T}} }^N.
\end{dmath}
%
Our free energy is
\begin{equation}\label{eqn:basicStatMechLecture12:480}
F = - \kB T N \ln \lr{
2 \cosh
\lr{
\frac{B}{2 \kB T}
}
}.
\end{equation}
%
For the expected value of the spin we find
\begin{equation}\label{eqn:basicStatMechLecture12:500}
\expectation{S_z} = \sum_i \expectation{S_{z_i}},
\end{equation}
where the expectation for a single particle's operator is
\begin{dmath}\label{eqn:basicStatMechLecture12:520}
\expectation{S_{z_i}}
=
\frac
{
\sum_\sigma \frac{\sigma}{2} e^{\beta B \sigma/2}
}
{
\sum_\sigma e^{\beta B \sigma/2}
}
=
\inv{2}
\frac{ 2 \sinh\lr{\beta B \sigma/2} }
{ 2 \cosh\lr{\beta B \sigma/2} }
= \inv{2} \tanh \lr{\frac{B}{2 \kB T}}.
\end{dmath}
} % example
%
\makeexample{Spin entropy in zero magnetic field}{example:basicStatMechLecture12:2}{
In \citep{kittel1980thermal} \S 6, is stated
%
Consider an atomic of spin I, where I may represent both electronic and nuclear spins.  The internal partition function associated with the spin alone is:
\begin{equation}\label{eqn:basicStatMechLecture12:21}
Z_{\mathrm{int}} = 2 I + 1.
\end{equation}
%
\paragraph{My question reading this:}
If I is the magnitude of the spin, then how would it also show up like this in the partition function (with no temperature or energy dependence?)
%
\paragraph{Prof Paramekanti's answer:}
%
At zero magnetic field, all spin states have exactly the same energy, which can be arbitrarily be chosen to be zero. The partition function then simply counts the number of configurations (since \(\exp(-\beta H) = 1\), the trace is just a sum of ``ones'').  For a spin-\(I\),the number of states is just (\(2 I +1\)).
}
%
%\EndArticle
