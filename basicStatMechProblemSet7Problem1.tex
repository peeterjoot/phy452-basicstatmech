%
% Copyright � 2013 Peeter Joot.  All Rights Reserved.
% Licenced as described in the file LICENSE under the root directory of this GIT repository.
%
\makeoproblem{1D, 2D Bose-Einstein condensation}{basicStatMech:problemSet7:1}{2013 ps 7 p1}{
\makesubproblem{}{basicStatMech:problemSet7:1a}
Obtain the \textAndIndex{density of states} \(N(\epsilon)\) in one and two dimensions for a particle with an energy-momentum relation
\begin{equation}\label{eqn:basicStatMechProblemSet7Problem1:20}
E_\Bk = \frac{\Hbar^2 \Bk^2}{2 m}.
\end{equation}
%
\makesubproblem{}{basicStatMech:problemSet7:1b}
Using this, show that for particles whose number is conserved the BEC transition temperature vanishes in these cases - so we can always pick a \textAndIndex{chemical potential} \(\mu < 0\) which preserves a constant density at any temperature.
%
} % makeoproblem
%
\makeanswer{basicStatMech:problemSet7:1}{
\makeSubAnswer{}{basicStatMech:problemSet7:1a}
%
We'd like to evaluate
\begin{dmath}\label{eqn:basicStatMechProblemSet7Problem1:460}
N_d(\epsilon) \equiv
\sum_\Bk
\delta(\epsilon - \epsilon_\Bk)
\approx
\frac{L^d}{(2 \pi)^d} \int d^d \Bk \delta\lr{ \epsilon - \frac{\Hbar^2 k^2}{2 m}},
\end{dmath}
%
We'll use
\begin{dmath}\label{eqn:basicStatMechProblemSet7Problem1:440}
\delta(g(x)) = \sum_{x_0} \frac{\delta(x - x_0)}{\Abs{g'(x_0)}},
\end{dmath}
where the roots of \(g(x)\) are \(x_0\).  With
\begin{dmath}\label{eqn:basicStatMechProblemSet7Problem1:480}
g(k) = \epsilon - \frac{\Hbar^2 k^2}{2 m},
\end{dmath}
the roots \(k^\conj\) of \(g(k) = 0\) are
\begin{dmath}\label{eqn:basicStatMechProblemSet7Problem1:500}
k^\conj = \pm \sqrt{\frac{2 m \epsilon }{\Hbar^2}}.
\end{dmath}
%
The derivative of \(g(k)\) evaluated at these roots are
\begin{dmath}\label{eqn:basicStatMechProblemSet7Problem1:520}
g'(k^\conj) =
-\frac{\Hbar^2 k^\conj}{m}
=
\mp
\frac{\Hbar^2}{m}
\frac{\sqrt{2 m \epsilon}}{ \Hbar }
=
\mp
\frac{\Hbar \sqrt{2 m \epsilon} }{m}.
\end{dmath}
%
In 2D, we can evaluate over a shell in \(k\) space
\begin{dmath}\label{eqn:basicStatMechProblemSet7Problem1:540}
N_2(\epsilon)
=
\frac{A}{(2 \pi)^2} \int_0^\infty 2 \pi k dk
\lr{ \delta\lr{ k - k^\conj }
+ \delta\lr{ k + k^\conj } }
\frac{m}{\Hbar \sqrt{2 m \epsilon} }
=
\frac{A}{2 \pi} \cancel{k^\conj}
\frac{m}{\Hbar^2 \cancel{k^\conj} },
\end{dmath}
or
\boxedEquation{eqn:basicStatMechProblemSet7Problem1:560}{
N_2(\epsilon)
=
\frac{2 \pi A m}{h^2}.
}
%
In 1D we have
\begin{dmath}\label{eqn:basicStatMechProblemSet7Problem1:580}
N_1(\epsilon)
=
\frac{L}{2 \pi} \int_{-\infty}^\infty dk
\lr{ \delta\lr{ k - k^\conj }
+ \delta\lr{ k + k^\conj } }
\frac{m}{\Hbar \sqrt{2 m \epsilon} }
=
\frac{2 L}{2 \pi}
\frac{m}{\Hbar \sqrt{2 m \epsilon} }.
\end{dmath}
%
Observe that this time for 1D, unlike in 2D when we used a radial shell in \(k\) space, we have contributions from both the delta function roots.  Our end result is
\boxedEquation{eqn:basicStatMechProblemSet7Problem1:600}{
N_1(\epsilon)
=
\frac{2 L}{h}
\sqrt{\frac{m}{2 \epsilon}}.
}
%
\makeSubAnswer{}{basicStatMech:problemSet7:1b}
%
To consider the question of the BEC temperature, we'll need to calculate the density.  For the 2D case we have
\begin{dmath}\label{eqn:basicStatMechProblemSet7Problem1:620}
\rho
= \frac{N}{A}
= \frac{1}{A} A \int \frac{d^2 \Bk}{(2 \pi)^2} f(e_\Bk)
= \frac{1}{A}
\frac{2 \pi A m}{h^2}
\int_0^\infty d\epsilon \inv{ z^{-1} e^{\beta \epsilon} -1 }
=
\frac{2 \pi m}{h^2 \beta}
\int_0^\infty dx \inv{ z^{-1} e^{x} -1 }
=
-\frac{2 \pi m \kB T}{h^2} \ln (1 - z)
=
-\inv{\lambda^2} \ln (1 - z).
\end{dmath}
%
Recall for the 3D case that we had an upper bound as \(z \rightarrow 1\).  We don't have that for this 2D density, so for any value of \(\kB T > 0\), a corresponding value of \(z\) can be found.  That is
\begin{equation}\label{eqn:basicStatMechProblemSet7Problem1:640}
z
= 1 - e^{-\rho \lambda^2}
= 1 - e^{-\rho h^4/(2 \pi m \kB T)^2}.
\end{equation}
%
For the 1D case we have
\begin{dmath}\label{eqn:basicStatMechProblemSet7Problem1:660}
\rho
= \frac{N}{L}
= \frac{1}{L} L \int \frac{dk}{2 \pi} f(e_\Bk)
= \frac{1}{L}
\frac{2 L}{h}
\sqrt{\frac{m}{2}}
\int_0^\infty
d\epsilon
\inv{\sqrt{\epsilon}}
\inv{ z^{-1} e^{\beta \epsilon} -1 }
=
\inv{h} \sqrt{\frac{2 m}{\beta}} \int_0^\infty \frac{x^{1/2 - 1}}{z^{-1} e^x - 1}
=
\inv{h} \sqrt{\frac{2 m}{\beta}} \Gamma(1/2) f^-_{1/2}(z),
\end{dmath}
%
or
\begin{dmath}\label{eqn:basicStatMechProblemSet7Problem1:680}
\rho
=
\inv{\lambda} f^-_{1/2}(z).
\end{dmath}
%
See \cref{fig:basicStatMechProblemSet7Problem1:basicStatMechProblemSet7Problem1Fig1} for plots of \(f^-_\nu(z)\) for \(\nu \in \{1/2, 1, 3/2\}\), the respective results for the 1D, 2D and 3D densities respectively.
%
\imageFigure{../figures/phy452-basicstatmech/basicStatMechProblemSet7Problem1Fig1}{Density integrals for 1D, 2D and 3D cases.}{fig:basicStatMechProblemSet7Problem1:basicStatMechProblemSet7Problem1Fig1}{0.2}
% FIXME: L19 .  reference this plot (or move it there)
%
We've found that \(f^-_{1/2}(z)\) is also unbounded as \(z \rightarrow 1\), so while we cannot invert this easily as in the 2D case, we can at least say that there will be some \(z\) for any value of \(\kB T > 0\) that allows the density (and thus the number of particles) to remain fixed.
}
