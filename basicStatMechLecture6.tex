%
% Copyright � 2013 Peeter Joot.  All Rights Reserved.
% Licenced as described in the file LICENSE under the root directory of this GIT repository.
%
%\input{../blogpost.tex}
%\renewcommand{\basename}{basicStatMechLecture6}
%\renewcommand{\dirname}{notes/phy452/}
%\newcommand{\keywords}{Statistical mechanics, PHY452H1S, Liouville's theorem, phase space, phase space volume, ergodic, ensemble average, ideal gas, gamma function, N dimension volume, indistinguishable states, entropy, Boltzmann's constant}
%\input{../peeter_prologue_print2.tex}
%
%\beginArtNoToc
%\generatetitle{PHY452H1S Basic Statistical Mechanics.  Lecture 6: Volumes in phase space.  Taught by Prof.\ Arun Paramekanti}
\label{chap:basicStatMechLecture6}
%
%\section{Disclaimer}
%
%Peeter's lecture notes from class.  May not be entirely coherent.

\section{Liouville's theorem}

We've looked at the continuity equation of phase space density
\begin{equation}\label{eqn:basicStatMechLecture6:20}
0 =
\PD{t}{\rho} + \sum_{i_\alpha} \left(
\PD{p_{i_\alpha}}{\left( \rho \dot{p}_{i_\alpha} \right)} + \PD{x_{i_\alpha}}{\left( \rho \dot{x}_{i_\alpha} \right) }
\right),
\end{equation}
which with
\begin{equation}\label{eqn:basicStatMechLecture6:40}
\PD{p_{i_\alpha}}{\dot{p}_{i_\alpha}} + \PD{x_{i_\alpha}}{\dot{x}_{i_\alpha}} = 0,
\end{equation}
led us to \underlineAndIndex{Liouville's theorem}
\boxedEquation{eqn:basicStatMechLecture6:30}{
\ddt{\rho}(x, p, t) = 0.
}

We define \underlineAndIndex{ergodic}, meaning that with time, as you wait for \(t \rightarrow \infty\), all \underline{available} phase space will be covered.  Not all systems are necessarily ergodic, but the hope is that all sufficiently complicated systems will be so.
We hope that
\begin{equation}\label{eqn:basicStatMechLecture6:60}
\rho(x, p, t \rightarrow \infty) \implies \PD{t}{\rho} = 0 \qquad \mbox{in steady state}.
\end{equation}

In particular for \(\rho = \text{constant}\), we see that our continuity equation \eqnref{eqn:basicStatMechLecture6:20} results in \eqnref{eqn:basicStatMechLecture6:40}. % (essentially a statement that the magnitude of the mixed partials of the Hamiltonian are equal).

For example in a SHO system with a cyclic phase space, as in \cref{fig:basicStatMechLecture6:basicStatMechLecture6Fig1}.
\imageFigure{../figures/phy452-basicstatmech/basicStatMechLecture6Fig1}{Phase space volume trajectory.}{fig:basicStatMechLecture6:basicStatMechLecture6Fig1}{0.3}
\begin{equation}\label{eqn:basicStatMechLecture6:80}
\expectation{A} = \inv{\tau} \int_0^\tau dt A( x_0(t), p_0(t) ),
\end{equation}
or equivalently with an \underlineAndIndex{ensemble average}, imagining that we are averaging over a number of different systems
\begin{equation}\label{eqn:basicStatMechLecture6:100}
\expectation{A} = \inv{\tau} \int dx dp A( x, p )
\mathLabelBox{
\rho(x, p)
}{constant}.
\end{equation}

If we say that
\begin{equation}\label{eqn:basicStatMechLecture6:120}
\rho(x, p) = \text{constant} = \inv{\Omega},
\end{equation}
so that
\begin{equation}\label{eqn:basicStatMechLecture6:140}
\expectation{A} = \inv{\Omega} \int dx dp A( x, p ),
\end{equation}
then what is this constant?  We fix this by the constraint
\begin{equation}\label{eqn:basicStatMechLecture6:160}
\int dx dp \rho(x, p) = 1.
\end{equation}

So, \(\Omega\) is the allowed ``volume'' of phase space, the number of states that the system can take that is consistent with conservation of energy.

What's the probability for a given configuration.  We'll have to enumerate all the possible configurations.  For a coin toss example, we can also ask how many configurations exist where the sum of ``coin tosses'' are fixed.

\paragraph{Relating Louiville's theorem to the number of configurations}

Some rough notes based on an after-class discussion:

We can examine the phase space trajectories of a number of particles.  For example in a SHO configuration, we could have a number of particles moving along elliptic trajectories, all starting in a region of phase space.  Drawing a boundary around this phase space points containing these initial conditions, this boundary will deform according to the Hamiltonian action, but will continue to contain all the particles.  Given a number of different possible initial conditions, we can pick any volume containing these same number of particles and look at the evolution of that volume in time.  All such configurations will have the same average behaviour over a large number of cycles, and it is this similarity in average that allows us to average over all the possible configurations instead of performing an average over time of a single configuration.  This freedom to average over configurations is our starting point for the remainder of the statistical mechanics that we will proceed with shortly.

How does this relate to Louville's theorem?  We start by demonstrating that phase space density is conserved, and then assume that we can approximate the systems of interest as ergodic systems.  For such an ergodic system, we have constant phase space density, so the number of configurations will be proportional to the volume of phase space that we are investigating.  It is really that idea, that the number of configurations is proportional to the region of phase space volume that is accessible given the energy constraints of the system that we want to use as the starting point for most of the rest of the course, and everything up to now is an attempt to justify this as a starting point.

\makeexample{Ideal gas calculation of \(\Omega\)}{example:lec6:1}{

\begin{enumerate}
\item \(N\) gas atoms at phase space points \(\Bx_i, \Bp_i\)
\item constrained to volume \(V\)
\item Energy fixed at \(E\).
\end{enumerate}

\begin{dmath}\label{eqn:basicStatMechLecture6:180}
\Omega(N, V, E) = \int_V
d\Bx_1
d\Bx_2
\cdots
d\Bx_N
\int
d\Bp_1
d\Bp_2
\cdots
d\Bp_N
\delta \left(
E
- \frac{\Bp_1^2}{2m}
- \frac{\Bp_2^2}{2m}
\cdots
- \frac{\Bp_N^2}{2m}
\right)
=
\mathLabelBox
[
   labelstyle={xshift=2cm},
   linestyle={out=270,in=90, latex-}
]
{
V^N
}{Real space volume, not \(N\) dimensional ``volume''}
\int
d\Bp_1
d\Bp_2
\cdots
d\Bp_N
\delta \left(
E
- \frac{\Bp_1^2}{2m}
- \frac{\Bp_2^2}{2m}
\cdots
- \frac{\Bp_N^2}{2m}
\right).
\end{dmath}

With \(\gamma\) defined implicitly by
\begin{equation}\label{eqn:basicStatMechLecture6:200}
\frac{d\gamma}{dE} = \Omega,
\end{equation}
so that with \textAndIndex{Heavyside theta} as in \cref{fig:basicStatMechLecture6:basicStatMechLecture6Fig2}.
\begin{subequations}
\begin{equation}\label{eqn:basicStatMechLecture6:220}
\Theta(x) =
\left\{
\begin{array}{l l}
1 & \quad x \ge 0 \\
0 & \quad x < 0
\end{array}
\right.
\end{equation}
\begin{equation}\label{eqn:basicStatMechLecture6:480}
\frac{d\Theta}{dx} = \delta(x),
\end{equation}
\end{subequations}
\imageFigure{../figures/phy452-basicstatmech/basicStatMechLecture6Fig2}{Heavyside theta, \(\Theta(x)\).}{fig:basicStatMechLecture6:basicStatMechLecture6Fig2}{0.3}
we have
\begin{equation}\label{eqn:basicStatMechLecture6:240}
\gamma(N, V, E) = V^N
\int
d\Bp_1
d\Bp_2
\cdots
d\Bp_N
\Theta \left(
E
- \sum_i \frac{\Bp_i^2}{2m}
\right).
\end{equation}

In three dimensions \((p_x, p_y, p_z)\), the dimension of momentum part of the phase space is 3.  In general the dimension of the space is \(3N\).  Observe that the structure of this Heavyside integral
\begin{equation}\label{eqn:basicStatMechLecture6:440}
\int
d\Bp_1
d\Bp_2
\cdots
d\Bp_N
\Theta \left(
E
- \sum_i \frac{\Bp_i^2}{2m}
\right),
\end{equation}
has the same structure as that of the volume of a sphere.  If the radius of a 3D sphere is \(R\), it's volume can be written
\begin{equation}\label{eqn:basicStatMechLecture6:440b}
\int
dx dy dz
\Theta \left(
R
- \sqrt{x^2 + y^2 + z^2}
\right),
\end{equation}
the integral of the volume element \(dx dy dz\) over the region \(x^2 + y^2 + z^2 \le R^2\).  By analogy \eqnref{eqn:basicStatMechLecture6:440} has the structure of the volume of a ``sphere'' in \(3N\)- dimensions.  Such a volume was found in the problems to be
\begin{subequations}
\begin{equation}\label{eqn:basicStatMechLecture6:460}
V_m
=
\frac{ \pi^{m/2} R^{m} }
{
   \Gamma\left( m/2 + 1 \right)
}.
\end{equation}
\begin{equation}\label{eqn:basicStatMechLecture6:320}
\Gamma(x) = \int_0^\infty dy e^{-y} y^{x-1}
\end{equation}
\begin{equation}\label{eqn:basicStatMechLecture6:340}
\Gamma(x + 1) = x \Gamma(x) = x!.
\end{equation}
\end{subequations}

Since we have
\begin{equation}\label{eqn:basicStatMechLecture6:260}
\Bp_1^2 + \cdots \Bp_N^2 \le 2 m E,
\end{equation}
the radius is
\begin{equation}\label{eqn:basicStatMechLecture6:280}
\text{radius} = \sqrt{ 2 m E}.
\end{equation}

This gives
\begin{dmath}\label{eqn:basicStatMechLecture6:300}
\gamma(N, V, E)
= V^N \frac{ \pi^{3 N/2} ( 2 m E)^{3 N/2}}{\Gamma( 3N/2 + 1) }
= V^N \frac{2}{3N} \frac{ \pi^{3 N/2} ( 2 m E)^{3 N/2}}{\Gamma( 3N/2 ) },
\end{dmath}

and
\begin{equation}\label{eqn:basicStatMechLecture6:360}
\Omega(N, V, E) = V^N \pi^{3 N/2} ( 2 m E)^{3 N/2 - 1} \frac{2 m}{\Gamma( 3N/2 ) }
\end{equation}

This result is almost correct, and we have to correct in 2 ways.  We have to fix the counting since we need an assumption that all the particles are indistinguishable.

\begin{enumerate}
\item Indistinguishability.  We must divide by \(N!\).
\item \(\Omega\) is not dimensionless.  We need to divide by \(h^{3N}\), where \(h\) is Plank's constant.
\end{enumerate}

In the real world we have to consider this as a quantum mechanical system.  Imagine a two dimensional phase space.  The allowed points are illustrated in \cref{fig:basicStatMechLecture6:basicStatMechLecture6Fig3}.

\imageFigure{../figures/phy452-basicstatmech/basicStatMechLecture6Fig3}{Phase space volume adjustment for the uncertainty principle.}{fig:basicStatMechLecture6:basicStatMechLecture6Fig3}{0.3}

Since \(\Delta x \Delta p \sim \Hbar\), the question of how many boxes there are, we calculate the total volume, and then divide by the volume of each box.  This sort of handwaving wouldn't be required if we did a proper quantum mechanical treatment.  The corrected result is
\boxedEquation{eqn:basicStatMechLecture6:380}{
\Omega_{\mathrm{correct}} = \frac{V^N}{N!} \inv{h^{3N}} \frac{( 2 \pi m E)^{3 N/2 }}{E} \frac{1}{\Gamma( 3N/2 ) }
}

} % makeexample
