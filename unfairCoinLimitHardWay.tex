%
% Copyright � 2013 Peeter Joot.  All Rights Reserved.
% Licenced as described in the file LICENSE under the root directory of this GIT repository.
%
%\input{../blogpost.tex}
%\renewcommand{\basename}{unfairCoinLimitHardWay}
%\renewcommand{\dirname}{notes/phy452/}
%\newcommand{\keywords}{Statistical mechanics, PHY452H1S, binomial distribution, Stirling's approximation, logarithm Taylor expansion, unfair coin toss}
%
%\input{../peeter_prologue_print2.tex}
%
%\beginArtNoToc
%
%\generatetitle{Limit of unfair coin distribution, the hard way}
%%\chapter{Limit of unfair coin distribution, the hard way}
%\label{chap:unfairCoinLimitHardWay}

\makeoproblem{Large N limit for unfair coin}{pr:unfairCoinDistributionHardWay:1}{2013 problem set 1, p2}{
We calculated the distribution for the sum of random variables associated with \(N\) unfair coin tosses, where the probabilities were \(r\), and \(s = 1 - r\) for heads and tails respectively.  Assigning heads and tails values of \(-1\) and \(+1\) respectively, the probability distribution of the sum \(X\) of the total numbers of heads and tails values for \(N\) such tosses was found to be
\begin{subequations}
\begin{equation}\label{eqn:unfairCoinLimitHardWay:20}
P_N(r, k) = \binom{N}{k} r^{N-k} s^k,
\end{equation}
\begin{equation}\label{eqn:unfairCoinLimitHardWay:40}
k = \frac{N + X}{2}.
\end{equation}
\end{subequations}
Calculate the limit for \(N \gg 1\) and \(N \gg X\), without using the central limit theorem (i.e. using Stirling's approximation and Taylor series expansion for the logs.)
} % makeoproblem

\makeanswer{pr:unfairCoinDistributionHardWay:1}{
Application of Stirling's approximation gives us
\begin{dmath}\label{eqn:unfairCoinLimitHardWay:60}
P_N(r, k)
\approx
\frac{
\sqrt{2 \pi N} \cancel{e^{-N}} N^N
}
{
\sqrt{2 \pi (N - k)} \cancel{e^{-N + k}} (N - k)^{N-k}
\sqrt{2 \pi k} \cancel{e^{-k}} k^{k}
}
r^{N-k} s^k
=
\sqrt{\frac{N}{2 \pi k(N-k)}}
N^{N
\mathLabelBox
[
   labelstyle={xshift=2cm},
   linestyle={out=270,in=90, latex-}
]
{
-k + k
}{Add and subtract}
}
\lr{ \frac{r}{N-k} }
^{N-k}
\lr{ \frac{s}{k} }^k
=
\sqrt{\frac{N}{2 \pi k(N-k)}}
\lr{ \frac{N r}{N-k} }
^{N-k}
\lr{ \frac{N s}{k} }^k.
\end{dmath}

The \(N r/(N-k)\) term looks like it can probably be coerced into \(1/(1 - y/N)\) form that will allow for Taylor expansion of the log.  With that change of variables, we find
%1 - \frac{y}{N} = \inv{N r}(N - k)
\begin{equation}\label{eqn:unfairCoinLimitHardWay:80}
k = N(1 - r) - y r  = N s + y r,
\end{equation}
so
\begin{subequations}
\begin{equation}\label{eqn:unfairCoinLimitHardWay:100}
\frac{k}{Ns} = 1 + \frac{y r}{N s}
\end{equation}
\begin{equation}\label{eqn:unfairCoinLimitHardWay:120}
\frac{N - k}{N r} = 1 - \frac{y}{N}.
\end{equation}
\end{subequations}

This is a bit unsymmetrical, so let's write \(y r = x\) so that
\begin{subequations}
\begin{equation}\label{eqn:unfairCoinLimitHardWay:140}
\frac{N - k}{N r} = 1 - \frac{x}{N r}
\end{equation}
\begin{equation}\label{eqn:unfairCoinLimitHardWay:160}
\frac{k}{Ns} = 1 + \frac{x}{N s}.
\end{equation}
\end{subequations}

We've also got terms in \(k\) and \(N - k\) above that we need to express.  With \(k = N s + x\), we have \(N - k = N r - x\), and
\begin{dmath}\label{eqn:unfairCoinLimitHardWay:180}
\frac{(N - k) k}{N}
= \inv{N}(N r - x)(N s + x)
= \inv{N}( -x^2 + N^2 r s + x N(r - s) )
= -\frac{x^2}{N} + N r s + x( 2 r - 1 )
\approx N r s.
\end{dmath}

Taking logs of \eqnref{eqn:unfairCoinLimitHardWay:60} we have
\begin{dmath}\label{eqn:unfairCoinLimitHardWay:200}
\ln P_N(r, k)
\approx
\ln \sqrt{\frac{1}{2 \pi N r s}}
-
(N r - x)
\ln
\lr{ 1 - \frac{x}{N r} }
-
(N s + x)
\ln
\lr{ 1 + \frac{x}{N s} }
\approx
\ln \sqrt{\frac{1}{2 \pi N r s}}
+
(x - N r)
\lr{
- \frac{x}{N r}
- \inv{2}
\lr{ \frac{x}{N r} }^2
}
-
(x + N s)
\lr{
\frac{x}{N s}
- \inv{2}
\lr{\frac{x}{N s} }^2
}
=
\ln \sqrt{\frac{1}{2 \pi N r s}}
+ \inv{2} \frac{x^2}{N}
\lr{ \inv{r} + \inv{s} }
- \frac{x^2}{N}
\lr{ \inv{r} + \inv{s} }
+ \frac{x^3}{(N)^2}
\lr{ \inv{r^2} - \inv{s^2} }.
\end{dmath}

Dropping the \(O(1/N^2)\) term and noting that
\begin{dmath}\label{eqn:unfairCoinLimitHardWay:220}
-\inv{2}
\lr{ \inv{r} + \inv{s} }
=
-\inv{2 r s} (r + (1 - r)
=
-\inv{2 r s}.
\end{dmath}

We have
\begin{equation}\label{eqn:unfairCoinLimitHardWay:240}
P_N(r, k) \approx
\inv{\sqrt{2 \pi N r s}}
\exp
\lr{ -\frac{x^2}{2 N r s} }.
\end{equation}

With
\begin{dmath}\label{eqn:unfairCoinLimitHardWay:260}
x = k - N s
= \inv{2} ( N + X ) - N s
= \inv{2} ( N + X - 2 N s )
= \inv{2} ( X + N ( 1 - 2 s) )
= \inv{2} ( X - N ( 1 - 2 r) ),
\end{dmath}
we have
\boxedEquation{eqn:unfairCoinLimitHardWay:280}{
P_N(r, k) \approx
\inv{\sqrt{2 \pi N r s}}
\exp
\lr{ -\frac{
\lr{X - N (1 - 2r) }^2
}{8 N r s} }.
}

This recovers the result obtained with the central limit theorem (after that result was adjusted to account for parity).
} % makeanswer

%Since the problem didn't say we couldn't use
% and got massively penalized (even though the problem didn't say we couldn't use the central limit theorem and it had been covered in class, something that I should probably dispute)

%\EndArticle
%\EndNoBibArticle
