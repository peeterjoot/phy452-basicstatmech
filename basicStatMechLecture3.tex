%
% Copyright � 2013 Peeter Joot.  All Rights Reserved.
% Licenced as described in the file LICENSE under the root directory of this GIT repository.
%
%\input{../blogpost.tex}
%\renewcommand{\basename}{basicStatMechLecture3}
%\renewcommand{\dirname}{notes/phy452/}
%\newcommand{\keywords}{Statistical mechanics, PHY452H1S}
%\input{../peeter_prologue_print2.tex}
%
%\beginArtNoToc
%\generatetitle{PHY452H1S Basic Statistical Mechanics.  Lecture 3: Random walk.  Taught by Prof.\ Arun Paramekanti}
\label{chap:basicStatMechLecture3}

%\section{Disclaimer}
%
%Peeter's lecture notes from class.  May not be entirely coherent.

\section{Spatial Random walk.}
%\fxwarning{review lecture 3}{work through this lecture in detail.}
%
\paragraph{One dimensional case}
%
%\cref{fig:basicStatMechLecture3:basicStatMechLecture3Fig1}.
\imageFigure{../figures/phy452-basicstatmech/basicStatMechLecture3Fig1}{One dimensional random walk.}{fig:basicStatMechLecture3:basicStatMechLecture3Fig1}{0.1}

With a time interval \(\delta t\), our total time is
\begin{equation}\label{eqn:basicStatMechLecture3:20}
t = N \delta t.
\end{equation}

We form a random variable for the total distance moved
\begin{equation}\label{eqn:basicStatMechLecture3:40}
X = \sum_{i = 1}^N \delta x_i
\end{equation}
\begin{equation}\label{eqn:basicStatMechLecture3:60}
\expectation{X} = \sum_{i = 1}^N \expectation{\delta x_i} = 0
\end{equation}
\begin{dmath}\label{eqn:basicStatMechLecture3:80}
\expectation{X^2}
=
\expectation{
\sum_{i = 1}^N \delta x_i
\sum_{j = 1}^N \delta x_j
}
= \sum_{i,j = 1}^N \expectation{\delta x_i \delta x_j}
= \sum_{i = 1}^N \expectation{(\delta x_i)^2 }
= \sum_{i = 1}^N \expectation{(\pm \delta x)^2 }
= (\delta x)^2 \sum_{i = 1}^N 1
= N (\delta x)^2.
\end{dmath}

In the above, I assume that the cross terms were killed with an assumption of uncorrelation.
%
\paragraph{Two dimensional case}
%
%\cref{fig:basicStatMechLecture3:basicStatMechLecture3Fig2}.
\imageFigure{../figures/phy452-basicstatmech/basicStatMechLecture3Fig2}{Two dimensional random walk.}{fig:basicStatMechLecture3:basicStatMechLecture3Fig2}{0.2}

The two dimensional case can be used for either spatial generalization of the above, or a one dimensional problem with time evolution as illustrated in \cref{fig:basicStatMechLecture3:basicStatMechLecture3Fig3a}.
% what was this about:
%\cref{fig:basicStatMechLecture3:basicStatMechLecture3Fig3}.
%\imageFigure{../figures/phy452-basicstatmech/basicStatMechLecture3Fig3}{A couple Gaussians}{fig:basicStatMechLecture3:basicStatMechLecture3Fig3}{0.2}
\imageFigure{../figures/phy452-basicstatmech/basicStatMechLecture3Fig3a}{Spatial and temporal evolution.}{fig:basicStatMechLecture3:basicStatMechLecture3Fig3a}{0.15}
\begin{dmath}\label{eqn:basicStatMechLecture3:100}
\calP( x, t )
=
\inv{2} \calP(x + \delta x, t - \delta t)
+
\inv{2} \calP(x - \delta x, t - \delta t)
\approx
\inv{2}
\left(
\calP(x, t - \delta t) +
\cancel{\PD{x}{\calP}(x, t - \delta t) \delta x}
+
\inv{2} \PDSq{x}{\calP}(x, t - \delta t) (\delta x)^2
\right)
+
\inv{2}
\left(
\calP(x, t - \delta t)
-
\cancel{\PD{x}{\calP}(x, t - \delta t) \delta x}
+
\inv{2} \PDSq{x}{\calP}(x, t - \delta t) (\delta x)^2
\right)
=
\calP(x, t - \delta t)
+
\inv{2} \PDSq{x}{\calP}(x, t - \delta t) (\delta x)^2.
\end{dmath}

Since we have a small correction \((\delta x)^2 \approx 0\), this is approximately
\begin{equation}\label{eqn:basicStatMechLecture3:120}
\calP(x, t)
\approx
\calP(x, t - \delta t)
+
\inv{2} \PDSq{x}{\calP}(x, t) (\delta x)^2.
\end{equation}
\begin{dmath}\label{eqn:basicStatMechLecture3:140}
\delta t \PD{t}{\calP}(x, t) \approx
\inv{2} \PDSq{x}{\calP}(x, t) (\delta x)^2
\end{dmath}
\begin{equation}\label{eqn:basicStatMechLecture3:160}
\PD{t}{\calP}(x, t) =
\mathLabelBox
[
   labelstyle={xshift=2cm},
   linestyle={out=270,in=90, latex-}
]
{
\inv{2}
\frac{(\delta x)^2}{\delta t}
}{\(\equiv D\), the \underlineAndIndex{diffusion constant}}
\PDSq{x}{\calP}(x, t)
%= D \PDSq{x}{\calP}(x, t)
\end{equation}

We see that random microscopics lead to a well defined macroscopic equation.
\begin{equation}\label{eqn:basicStatMechLecture3:180}
N \calP(x, t) =
\mathLabelBox
[
   labelstyle={xshift=2cm},
   linestyle={out=270,in=90, latex-}
]
{
C(x, t)
}{particle density or concentration}.
\end{equation}

\section{Continuity equation and Ficks law.}
%
\paragraph{Requirements for well defined diffusion results}
%
\begin{enumerate}
\item Particle number conservation (local)

Imagine that we put a drop of ink in water.  We'll see the ink gradually spread out and appear to disappear.  But the particles are still there.  We require particle number concentration for well defined diffusion results.

This is expressed as a \underlineAndIndex{continuity equation} and illustrated in \cref{fig:basicStatMechLecture3:basicStatMechLecture3Fig4}.
\begin{equation}\label{eqn:basicStatMechLecture3:200}
\PD{t}{\calP} + \PD{x}{J} = 0.
\end{equation}

\imageFigure{../figures/phy452-basicstatmech/basicStatMechLecture3Fig4}{probability changes by flows through the boundary.}{fig:basicStatMechLecture3:basicStatMechLecture3Fig4}{0.2}

Here \(J\) is the probability current density, and \(\calP\) is the probability density.

\item \underline{Phenomenological law}, or \underlineAndIndex{Fick's law}.
\begin{equation}\label{eqn:basicStatMechLecture3:220}
J = - D \PD{x}{\calP}.
\end{equation}

The probability current is linearly proportional to the gradient of the probability density.
%\cref{fig:basicStatMechLecture3:basicStatMechLecture3Fig5}.
\imageFigure{../figures/phy452-basicstatmech/basicStatMechLecture3Fig5}{Probability current proportional to gradient.}{fig:basicStatMechLecture3:basicStatMechLecture3Fig5}{0.2}
\end{enumerate}

Combining the above we have
\begin{equation}\label{eqn:basicStatMechLecture3:240}
0 =
\PD{t}{\calP} + \PD{x}{}
\left(
- D \PD{x}{\calP}
\right).
\end{equation}

So for constant diffusion rates \(D\) we also arrive at the random walk diffusion result
\boxedEquation{eqn:basicStatMechLecture3:260}{
\PD{t}{\calP} = D \PDSq{x}{\calP} .
}

\section{Random walk in velocity space.}

%\cref{fig:basicStatMechLecture3:basicStatMechLecture3Fig6}.
\imageFigure{../figures/phy452-basicstatmech/basicStatMechLecture3Fig6}{Single atom velocity view.}{fig:basicStatMechLecture3:basicStatMechLecture3Fig6}{0.2}

We are imagining that we are following one particle (particle \(i\)) in a gas, initially propagating without interaction at some velocity \(\Bv_i\).  After one collision we have
\begin{equation}\label{eqn:basicStatMechLecture3:280}
\Bv_i(1) = \Bv_i(0) + \Delta \Bv_i(1).
\end{equation}

After two collisions
\begin{equation}\label{eqn:basicStatMechLecture3:300}
\Bv_i(2) = \Bv_i(1) + \Delta \Bv_i(2).
\end{equation}

After \(N_{\txtc}\) collisions
\begin{equation}\label{eqn:basicStatMechLecture3:320}
\Bv_i(N_{\txtc}) = \Bv_i(0)
+ \Delta \Bv_i(1)
+ \Delta \Bv_i(2)
+ \cdots
+ \Delta \Bv_i(N_{\txtc}).
\end{equation}

Where \(N_{\txtc}\) is the number of collisions.  We expect
\begin{subequations}
\begin{equation}\label{eqn:basicStatMechLecture3:340}
\expectation{\Bv_i} = \Bv_i(0)
\end{equation}
\begin{equation}\label{eqn:basicStatMechLecture3:360}
\expectation{\Bv_i^2} \propto N_{\txtc}
\end{equation}
\begin{equation}\label{eqn:basicStatMechLecture3:380}
\calP_{N_{\txtc}}(\Bv) \propto e^{-
\frac{(\Bv - \Bv_0)^2}{2 N_{\txtc}}
}.
\end{equation}
\end{subequations}

Here the mean is expected to be zero because the individual collisions are thought to be uncorrelated.

We'll see that there is something wrong with this, and will figure out how to fix this in the next lecture.

In particular, the kinetic energy of any given particle is
\begin{equation}\label{eqn:basicStatMechLecture3:400}
\text{KE}_i = \inv{2} m \Bv_i^2,
\end{equation}
and the sum of this is fixed for the complete system.

However, if we sum the kinetic energy squares above \eqnref{eqn:basicStatMechLecture3:360} shows that it is growing with the number of collisions.  Fixing this we will arrive at the Maxwell Boltzmann distribution.

%\EndNoBibArticle
