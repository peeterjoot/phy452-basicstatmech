%
% Copyright � 2013 Peeter Joot.  All Rights Reserved.
% Licenced as described in the file LICENSE under the root directory of this GIT repository.
%
%\input{../blogpost.tex}
%\renewcommand{\basename}{basicStatMechLecture20}
%\renewcommand{\dirname}{notes/phy452/}
%\newcommand{\keywords}{Statistical mechanics, PHY452H1S, Bose condensation, chemical potential, density of states, specific heat, energy density, Helium-4, photon gas, polarization, average energy density, phonon}
%\input{../peeter_prologue_print2.tex}
%
%\beginArtNoToc
%\generatetitle{PHY452H1S Basic Statistical Mechanics.  Lecture 20: Bosons.  Taught by Prof.\ Arun Paramekanti}
%\chapter{Bosons}
\label{chap:basicStatMechLecture20}

%\section{Disclaimer}
%
%Peeter's lecture notes from class.  May not be entirely coherent.
%
\paragraph{Bose condensate}
In order to maintain a conservation of particles in a Bose condensate as we decrease temperature, we are forced to change the \textAndIndex{chemical potential} to compensate.  This is illustrated in \cref{fig:lecture20:lecture20Fig1}.

\imageFigure{../figures/phy452-basicstatmech/lecture20Fig1}{Chemical potential in Bose condensation region.}{fig:lecture20:lecture20Fig1}{0.2}
Bose condensation occurs for \(T < T_{\mathrm{BEC}}\).  At this point our number density becomes (except at \(\Bk = 0\))
\begin{equation}\label{eqn:basicStatMechLecture20:20}
n(\Bk) = \inv{e^{\beta \epsilon_\Bk} - 1}.
\end{equation}

Except for \(\Bk = 0\), \(n(\Bk)\) is well defined, and not described by this distribution.  We are forced to say that
\begin{dmath}\label{eqn:basicStatMechLecture20:40}
N
= N_0 + \sum_{\Bk \ne 0} n(\Bk)
= N_0 + V
\int \frac{d^3 \Bk}{(2 \pi)^3} \inv{ e^{\beta \epsilon_\Bk} - 1 }.
\end{dmath}

Introducing the \textAndIndex{density of states}, our density is
\begin{equation}\label{eqn:basicStatMechLecture20:60}
\rho = \rho_0 + \int_0^\infty d\epsilon \frac{N(\epsilon)}{e^{\beta \epsilon} - 1 },
\end{equation}
where
\begin{equation}\label{eqn:basicStatMechLecture20:80}
N(\epsilon) = \inv{4 \pi^2} \lr{ \frac{2m}{\Hbar} }^{3/2} \epsilon^{1/2}.
\end{equation}

We worked out last time that
% FIXME: reference
\begin{dmath}\label{eqn:basicStatMechLecture20:100}
\rho = \rho_0 + \rho \lr{ \frac{T}{T_{\mathrm{BEC}}} }^{3/2},
\end{dmath}
or
\begin{equation}\label{eqn:basicStatMechLecture20:120}
\rho_0 = \rho \lr{
1 -
\lr{ \frac{T}{T_{\mathrm{BEC}}} }^{3/2}
}.
\end{equation}

This is plotted in \cref{fig:lecture20:lecture20Fig2}.
\imageFigure{../figures/phy452-basicstatmech/lecture20Fig2x}{Density variation with temperature for Bosons.}{fig:lecture20:lecture20Fig2}{0.2}
\begin{dmath}\label{eqn:basicStatMechLecture20:140}
\rho_0 = \frac{N_{\Bk = 0}}{V}.
\end{dmath}

For \(T \ge T_{\mathrm{BEC}}\), we have \(\rho_0 = 0\).  This condensation temperature is
\begin{dmath}\label{eqn:basicStatMechLecture20:160}
T_{\mathrm{BEC}} \propto \rho^{2/3}.
\end{dmath}

This is plotted in \cref{fig:lecture20:lecture20Fig3}.

\imageFigure{../figures/phy452-basicstatmech/lecture20Fig3}{Temperature vs pressure demarcation by \(T_{\mathrm{BEC}}\) curve.}{fig:lecture20:lecture20Fig3}{0.2}

There is a line for each density that marks the boundary temperature for which we have or do not have this condensation phenomena where \(\Bk = 0\) states start filling up.

\paragraph{Specific heat: \(T < T_{\mathrm{BEC}}\)}
\begin{dmath}\label{eqn:basicStatMechLecture20:180}
\frac{E}{V}
= \int \frac{d^3 \Bk}{(2 \pi)^3}
\inv{ e^{\beta \Hbar^2 k^2/2m} - 1}
\frac{\Hbar^2 k^2}{2m}
=
\int_0^\infty d\epsilon N(\epsilon) \inv{ e^{\beta \epsilon} - 1 } \epsilon
\propto
\int_0^\infty d\epsilon \frac{\epsilon^{3/2}}{ e^{\beta \epsilon} - 1 }
\propto
\lr{ \kB T}
^{5/2},
\end{dmath}
so that
\begin{dmath}\label{eqn:basicStatMechLecture20:200}
\frac{C}{V} \propto
\lr{\kB T}^
{3/2}.
\end{dmath}

Compare this to the classical and Fermionic specific heat as plotted in \cref{fig:lecture20:lecture20Fig4}.

\imageFigure{../figures/phy452-basicstatmech/lecture20Fig4}{Specific heat for Bosons, Fermions, and classical ideal gases.}{fig:lecture20:lecture20Fig4}{0.2}

One can measure the specific heat in this Bose condensation phenomena for materials such as Helium-4 (spin 0).  However, it turns out that Helium-4 is actually quite far from an ideal Bose gas.

\paragraph{Photon gas}

A system that is much closer to an ideal Bose gas is that of a gas of photons.  To a large extent, photons do not interact with each other.  This allows us to calculate black body phenomena and the low temperature (cosmic) background radiation in the universe.

An important distinction between a photon sea and some of these other systems is that the photon number is actually not fixed.

Photon numbers are not ``conserved''.

If a photon interacts with an atom, it can impart energy and disappear.  An excited atom can emit a photon and change its energy level.  In a thermodynamic system we can generally expect that introducing heat will generate more photons, whereas a cold sink will tend to generate fewer photons.

We have a few special details that distinguish photons that we'll have to consider.

\begin{itemize}
\item spin 1.
\item massless, moving at the speed of light.
\item have two polarization states.
\end{itemize}

Because we do not have a constraint on the number of particles, we essentially have no chemical potential, even in the grand canonical scheme.

Writing
\begin{dmath}\label{eqn:basicStatMechLecture20:220}
\lambda =
\left\{
\begin{array}{l l}
+1 & \quad \mbox{Right circular polarization} \\
-1 & \quad \mbox{Left circular polarization}
\end{array}
\right.
\end{dmath}

Our number density, since we have no chemical potential, is of the form
\begin{dmath}\label{eqn:basicStatMechLecture20:240}
n_{\Bk, \lambda}
=
\inv{e^{\beta \epsilon_{\Bk, \lambda}} - 1 },
\end{dmath}

Observe that the average number of photons in this system is temperature dependent.  Because this chemical potential is not there, it can be quite easy to work out a number of the thermodynamic results.

\paragraph{Photon average energy density}

We'll now calculate the average energy density of the photons.  The energy of a single photon is
\begin{equation}\label{eqn:basicStatMechLecture20:260}
\epsilon_{\Bk, \lambda} = \Hbar c k = \Hbar \omega,
\end{equation}
so that the average energy density is
\begin{dmath}\label{eqn:basicStatMechLecture20:280}
\frac{E}{V}
=
\sum_{\Bk, \lambda} \inv{ e^{ \beta \epsilon_\Bk} - 1} \epsilon_\Bk
\rightarrow
\mathLabelBox
[
   labelstyle={xshift=4cm, yshift=1cm},
   linestyle={out=270,in=90, latex-}
]
{2}{number of polarizations}
\int \frac{d^3 \Bk}{(2 \pi)^3}
\frac{ \Hbar c k}{ e^{ \beta \epsilon_\Bk} - 1}
=
2 \int_0^\infty d\epsilon
\mathLabelBox
[
   labelstyle={xshift=4cm, yshift=1cm},
   linestyle={out=270,in=90, latex-}
]
{
\inv{(2 \pi)^3} 4 \pi \frac{\epsilon^2}{(\Hbar c)^3}
}{Photon density of states \eqnref{eqn:relativisticDensityOfStates:380}}
\frac{\epsilon}{e^{\beta \epsilon} - 1}
=
\inv{\pi^2} \inv{ (\Hbar c)^3 }
\int_0^\infty d\epsilon \frac{\epsilon^3}{e^{\beta \epsilon} - 1}.
\end{dmath}

This integral is calculated in \nbref{lecture20PlotAndIntegral.nb}, with the result
\begin{dmath}\label{eqn:basicStatMechLecture20:320}
\int_0^\infty d\epsilon \frac{\epsilon^3}{e^{\beta \epsilon} - 1} =
\frac{\pi ^4}{15 \beta ^4},
\end{dmath}
for an end result of
\begin{dmath}\label{eqn:basicStatMechLecture20:300}
\frac{E}{V}
=
\frac{\pi^2}{15} \inv{(\Hbar c)^3}
\lr{\kB T}^4,
\end{dmath}
from which we see that the specific heat of a 3D Bose system is of the form
\begin{equation}\label{eqn:basicStatMechLecture20:n}
\CV \propto T^3.
\end{equation}

%Plotting the number
%\cref{fig:lecture20:lecture20Fig5}.
%\imageFigure{../figures/phy452-basicstatmech/lecture20Fig5}{CAPTION}{fig:lecture20:lecture20Fig5}{0.2}

%\EndNoBibArticle
