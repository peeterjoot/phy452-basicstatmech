%
% Copyright � 2013 Peeter Joot.  All Rights Reserved.
% Licenced as described in the file LICENSE under the root directory of this GIT repository.
%
\paragraph{Linear energy momentum relationships}

Suppose that we have a linear energy momentum relationship like

\begin{dmath}\label{eqn:basicStatMechLecture17:220}
\epsilon_\Bk = v \Abs{\Bk}.
\end{dmath}

An example of such a relationship is the high velocity relation between the energy and momentum of a particle

\begin{dmath}\label{eqn:basicStatMechLecture17:240}
\epsilon_\Bk = \sqrt{ m_0^2 c^4 + p^2 c^2 } \sim \Abs{\Bp} c.
\end{dmath}

Another example is graphene, a carbon structure of the form \cref{fig:lecture17:lecture17Fig3}.  The energy and momentum for such a structure is related in roughly as shown in \cref{fig:lecture17:lecture17Fig4}, where

\imageFigure{../figures/phy452-basicstatmech/lecture17Fig3}{Graphene bond structure.}{fig:lecture17:lecture17Fig3}{0.3}
\imageFigure{../figures/phy452-basicstatmech/lecture17Fig4}{Graphene energy momentum dependence.}{fig:lecture17:lecture17Fig4}{0.3}

\begin{dmath}\label{eqn:basicStatMechLecture17:260}
\epsilon_\Bk = \pm v_{\txtF} \Abs{\Bk}.
\end{dmath}

%\paragraph{Relativistic gas}

Some examples of linear energy momentum materials are

\begin{itemize}
\item Relativistic gas

\begin{dmath}\label{eqn:basicStatMechLecture18:340}
\epsilon_\Bk = \pm \Hbar v \Abs{\Bk}.
\end{dmath}
\begin{dmath}\label{eqn:basicStatMechLecture18:360}
\epsilon = \sqrt{(m_0 c^2)^2 + c^2 (\Hbar \Bk)^2}.
\end{dmath}

\item graphene

\item massless Dirac Fermion

%\cref{fig:lecture18:lecture18Fig5}.
\imageFigure{../figures/phy452-basicstatmech/lecture18Fig5}{Relativistic gas energy distribution.}{fig:lecture18:lecture18Fig5}{0.3}

We can think of this state distribution in a condensed matter view, where we can have a hole to electron state transition by supplying energy to the system (i.e. shining light on the substrate).  This can also be thought of in a relativistic particle view where the same state transition can be thought of as a positron electron pair transition.  A round trip transition will have to supply energy like \(2 m_0 c^2\) as illustrated in \cref{fig:lecture18:lecture18Fig6}.

\imageFigure{../figures/phy452-basicstatmech/lecture18Fig6}{Hole to electron round trip transition energy requirement.}{fig:lecture18:lecture18Fig6}{0.2}

\end{itemize}

\paragraph{Graphene}

Consider graphene, a 2D system.  We want to determine the \textAndIndex{density of states} \(N(\epsilon)\), 

\begin{dmath}\label{eqn:basicStatMechLecture18:380}
\int \frac{d^2 \Bk}{(2 \pi)^2} \rightarrow \int_{-\infty}^\infty d\epsilon N(\epsilon),
\end{dmath}

We'll find a density of states distribution like \cref{fig:lecture18:lecture18Fig7}.

\imageFigure{../figures/phy452-basicstatmech/lecture18Fig7}{Density of states for 2D linear energy momentum distribution.}{fig:lecture18:lecture18Fig7}{0.2}

\begin{dmath}\label{eqn:basicStatMechLecture18:400}
N(\epsilon) = \text{constant factor} \frac{\Abs{\epsilon}}{v},
\end{dmath}

\begin{dmath}\label{eqn:basicStatMechLecture18:420}
C \sim \frac{d}{dT} \int N(\epsilon) n_{\txtF}(\epsilon) \epsilon d\epsilon,
\end{dmath}

\begin{dmath}\label{eqn:basicStatMechLecture18:440}
\Delta E 
\sim 
\mathLabelBox
{
T}{window}
\times
\mathLabelBox
[
   labelstyle={below of=m\themathLableNode, below of=m\themathLableNode}
]
{
T}{energy}
\times
\mathLabelBox
[
   labelstyle={xshift=2cm},
   linestyle={out=270,in=90, latex-}
]
{
T}{number of states}
\sim T^3,
\end{dmath}
so that
\begin{dmath}\label{eqn:basicStatMechLecture18:460}
C_{\mathrm{Dimensionless}} \sim T^2.
\end{dmath}

%\EndNoBibArticle
