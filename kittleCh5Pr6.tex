%
% Copyright � 2013 Peeter Joot.  All Rights Reserved.
% Licenced as described in the file LICENSE under the root directory of this GIT repository.
%
%\input{../blogpost.tex}
%\renewcommand{\basename}{kittleCh5Pr6}
%\renewcommand{\dirname}{notes/phy452/}
%\newcommand{\keywords}{Statistical mechanics, PHY452H1S, grand canonical partition, average number of particles, thermal average energy, Gibbs sum}
%
%\input{../peeter_prologue_print2.tex}
%
%\beginArtNoToc
%
%\generatetitle{Gibbs sum for a two level system}
%\label{chap:kittleCh5Pr6}
%
\makeoproblem{Gibbs sum for a two level system}{pr:kittleCh5Pr6:6}{\citep{kittel1980thermal} pr. 6}{
\makesubproblem{}{pr:kittleCh5Pr6:6:a}
Consider a system that may be unoccupied with energy zero or occupied by one particle in either of two states, one of energy zero and one of energy \(\epsilon\).  Find the Gibbs sum for this system.  Our assumption excludes the possibility of one particle in each state at the same time.%  Notice that we include in the sum a term for \(N = 0\) as a particular state of a system of a variable number of particles.
%
\makesubproblem{}{pr:kittleCh5Pr6:6:b}
Find the average thermal occupancy of the system.
%
\makesubproblem{}{pr:kittleCh5Pr6:6:c}
Find the thermal average occupancy of the state at energy \(\epsilon\).
%
\makesubproblem{}{pr:kittleCh5Pr6:6:d}
Find an expression for the thermal average energy of the system.
%
%\makesubproblem{}{pr:kittleCh5Pr6:6:e}
%
%Allow the possibility that the orbital at \(0\) and at \(\epsilon\) may be occupied each by one particle at the same time.
% Because \(\ZG\) can be factored as shown, we have in effect two independent systems.
} % makeoproblem
\makeanswer{pr:kittleCh5Pr6:6}{ } % makeanswer
%
\makeSubAnswer{Gibbs sum}{pr:kittleCh5Pr6:6:a}
%
We can write the grand partition function immediately
\begin{equation}\label{eqn:kittleCh5Pr6:20}
\begin{aligned}
\ZG 
&=
\mathLabelBox{
1
}{\(N = 0\)}
e^{-\beta(0)}
+
\mathLabelBox
[
   labelstyle={below of=m\themathLableNode, below of=m\themathLableNode}
]
{
z^1
}{\(N = 1\)}
\lr{
e^{-\beta(0)}
+
e^{-\beta \epsilon }
} \\
&= 1 + z + z e^{-\beta \epsilon}.
\end{aligned}
\end{equation}
%
\makeSubAnswer{Thermal average occupancy}{pr:kittleCh5Pr6:6:b}
%
This also follows almost immediately
\begin{equation}\label{eqn:kittleCh5Pr6:40}
\begin{aligned}
\expectation{N}
&= \frac{ 0 \times 1 + 1 \times z \lr{ 1 + e^{-\beta \epsilon}}} { \ZG } \\
&= \frac{ z + z e^{-\beta \epsilon} } { \ZG }.
\end{aligned}
\end{equation}
%
\makeSubAnswer{Thermal average occupancy at energy \(\epsilon\)}{pr:kittleCh5Pr6:6:c}
The average occupancy at energy \(\epsilon\) just has one contributing term in the sum
\begin{equation}\label{eqn:kittleCh5Pr6:60}
\expectation{N(\epsilon)}
=
\frac{ z e^{-\beta \epsilon} }
{
\ZG
}.
\end{equation}
%
\makeSubAnswer{Thermal average energy}{pr:kittleCh5Pr6:6:d}
%
The average thermal energy is
\begin{equation}\label{eqn:kittleCh5Pr6:80}
\begin{aligned}
\expectation{U}
&= \frac{ 0 \lr{1 + z } + \epsilon z e^{-\beta \epsilon} } { \ZG } \\
&= \frac{ \epsilon z e^{-\beta \epsilon} } { \ZG }.
\end{aligned}
\end{equation}
%
%\makeSubAnswer{Generalization}{pr:kittleCh5Pr6:6:e}
%\EndArticle
