%
% Copyright � 2013 Peeter Joot.  All Rights Reserved.
% Licenced as described in the file LICENSE under the root directory of this GIT repository.
%
\makeoproblem{Sackur-Tetrode entropy of an Ideal Gas}{basicStatMech:problemSet4:1}{\citep{schroeder2000thermal} 2.35}{
The entropy \index{Sackur-Tetrode entropy} of an ideal gas is given by
\begin{equation}\label{eqn:basicStatMechProblemSet4Problem1:20}
S =
N \kB
\lr{
\ln
\lr{
\frac{V}{N}
\lr{
\frac{4 \pi m E}{3 N h^2}
}^{3/2}
}
+ \frac{5}{2}
}
\end{equation}

Find the temperature of this gas via \((\partial S/ \partial E)_{V,N} = 1/T\). Find the energy per particle at which the entropy becomes negative. Is there any meaning to this temperature?
} % makeoproblem

\makeanswer{basicStatMech:problemSet4:1}{

Taking derivatives we find
\begin{dmath}\label{eqn:basicStatMechProblemSet4Problem1:40}
\inv{T} =
\PD{E}{}
\lr{
\cancel{
N \kB
\ln
\frac{V}{N}
}
+
N \kB
\frac{3}{2} \ln
\lr{ \frac{4 \pi m E}{3 N h^2} }
+
\cancel{N \kB
\frac{5}{2}
}
}
=
\frac{3}{2} N \kB \inv{E}
\end{dmath}

or
\boxedEquation{eqn:basicStatMechProblemSet4Problem1:60}{
%\inv{T} = \frac{3}{2 E} N \kB
T = \frac{2}{3} \frac{E}{N \kB }.
}

The energies for which the entropy is negative are given by
\begin{equation}\label{eqn:basicStatMechProblemSet4Problem1:80}
\lr{
\frac{4 \pi m E}{3 N h^2}
}^{3/2}
\le \frac{N}{V} e^{-5/2},
\end{equation}

or
\begin{equation}\label{eqn:basicStatMechProblemSet4Problem1:100}
E \le
\frac{3 N h^2}{4 \pi m} \lr{\frac{N}{V e^{5/2}} }^{2/3}
=
\frac{3 h^2 N^{5/3}}{4 \pi m V^{2/3} e^{5/2}}.
\end{equation}

In terms of the temperature \(T\) this negative entropy condition is given by
\begin{equation}\label{eqn:basicStatMechProblemSet4Problem1:120}
\cancel{\frac{3 N}{2}} \kB T \le \cancel{\frac{3 N}{2}} \lr{\frac{ N}{V}}^{2/3} \frac{h^2}{e^{5/2}},
\end{equation}
or
\boxedEquation{eqn:basicStatMechProblemSet4Problem1:140}{
\frac{\sqrt{2 \pi m \kB T}}{h} \le \lr{\frac{N}{V}}^{1/3} \inv{e^{5/4}}.
}

There will be a particle density \(V/N\) for which this distance \(h/\sqrt{2 \pi m \kB T}\) will start approaching the distance between atoms.  This distance constrains the validity of the ideal gas law entropy equation.  Putting this quantity back into the entropy \eqnref{eqn:basicStatMechProblemSet4Problem1:20} we have
\begin{equation}\label{eqn:basicStatMechProblemSet4Problem1:160}
\frac{S}{N \kB} = \ln \frac{V}{N} \lr{\frac{\sqrt{2 \pi m \kB T}}{h}}^3 + \frac{5}{2}
\end{equation}

We see that a positive entropy requirement puts a bound on this distance (as a function of temperature) since we must also have
\begin{equation}\label{eqn:basicStatMechProblemSet4Problem1:180}
\frac{h}{\sqrt{2 \pi m \kB T}} \ll \lr{\frac{V}{N}}^{1/3},
\end{equation}
for the gas to be in the classical domain.  I'd actually expect a gas to liquefy before this transition point, making such a low temperature nonphysical.  To get a feel for whether this is likely the case, we should expect that the logarithm argument to be
\begin{equation}\label{eqn:basicStatMechProblemSet4Problem1:200}
\frac{V}{N} \lr{\frac{\sqrt{2 \pi m \kB T}}{h}}^3,
\end{equation}
at the point where gasses liquefy (at which point we assume the ideal gas law is no longer accurate) to be well above unity.  Checking this for 1 liter of a gas with \(10^23\) atoms for hydrogen, helium, and neon respectively we find the values for \ref{eqn:basicStatMechProblemSet4Problem1:200} are
\begin{equation}\label{eqn:basicStatMechProblemSet4Problem1:220}
173.682, 130.462, 23993.
\end{equation}

At least for these first few cases we see that the ideal gas law has lost its meaning well before the temperatures below which the entropy would become negative.
%There are probably some combinations of \((V, N, E)\) for which a negative entropy temperature is possible while still in the gaseous domain, but if so,
}
