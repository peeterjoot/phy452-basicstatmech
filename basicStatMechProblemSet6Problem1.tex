%
% Copyright � 2013 Peeter Joot.  All Rights Reserved.
% Licenced as described in the file LICENSE under the root directory of this GIT repository.
%
\makeproblem{Maximum entropy principle}{basicStatMech:problemSet6:1}{
%{2013 problem set 6, problem 1}
%(3 points)
Consider the ``Gibbs entropy''
\begin{dmath}\label{eqn:basicStatMechProblemSet6Problem1:20}
S = - \kB \sum_i p_i \ln p_i,
\end{dmath}
where \(p_i\) is the equilibrium probability of occurrence of a microstate \(i\) in the ensemble.

\makesubproblem{}{basicStatMech:problemSet6:1a}
For a microcanonical ensemble with \(\Omega\) configurations (each having the same energy), assigning an equal probability \(p_i= 1/\Omega\) to each microstate leads to \(S = \kB \ln \Omega\).  Show that this result follows from maximizing the Gibbs entropy with respect to the parameters \(p_i\) subject to the constraint of

\begin{dmath}\label{eqn:basicStatMechProblemSet6Problem1:40}
\sum_i p_i = 1
\end{dmath}

(for \(p_i\) to be meaningful as probabilities).  In order to do the minimization with this constraint, use the method of Lagrange multipliers - first, do an unconstrained minimization of the function
\begin{dmath}\label{eqn:basicStatMechProblemSet6Problem1:60}
S - \alpha \sum_i p_i,
\end{dmath}
then fix \(\alpha\) by demanding that the constraint be satisfied.

\makesubproblem{}{basicStatMech:problemSet6:1b}
For a canonical ensemble (no constraint on total energy, but all microstates having the same number of particles \(N\)), maximize the Gibbs entropy with respect to the parameters \(p_i\) subject to the constraint of
\begin{dmath}\label{eqn:basicStatMechProblemSet6Problem1:200}
\sum_i p_i = 1,
\end{dmath}
(for \(p_i\) to be meaningful as probabilities) and with a given fixed average energy
\begin{dmath}\label{eqn:basicStatMechProblemSet6Problem1:80}
\expectation{E} = \sum_i E_i p_i,
\end{dmath}
where \(E_i\) is the energy of microstate \(i\).  Use the method of Lagrange multipliers, doing an unconstrained minimization of the function
\begin{dmath}\label{eqn:basicStatMechProblemSet6Problem1:100}
S - \alpha \sum_i p_i - \beta \sum_i E_i p_i,
\end{dmath}
then fix \(\alpha, \beta\) by demanding that the constraint be satisfied.  What is the resulting \(p_i\)?

\makesubproblem{}{basicStatMech:problemSet6:1c}
For a grand canonical ensemble (no constraint on total energy, or the number of particles), maximize the Gibbs entropy with respect to the parameters \(p_i\) subject to the constraint of
\begin{dmath}\label{eqn:basicStatMechProblemSet6Problem1:120}
\sum_i p_i = 1,
\end{dmath}
(for \(p_i\) to be meaningful as probabilities) and with a given fixed average energy
\begin{dmath}\label{eqn:basicStatMechProblemSet6Problem1:140}
\expectation{E} = \sum_i E_i p_i,
\end{dmath}
and a given fixed average particle number
\begin{dmath}\label{eqn:basicStatMechProblemSet6Problem1:160}
\expectation{N} = \sum_i N_i p_i.
\end{dmath}

Here \(E_i, N_i\) represent the energy and number of particles in microstate \(i\).  Use the method of Lagrange multipliers, doing an unconstrained minimization of the function
\begin{dmath}\label{eqn:basicStatMechProblemSet6Problem1:180}
S - \alpha \sum_i p_i - \beta \sum_i E_i p_i - \gamma \sum_i N_i p_i,
\end{dmath}

then fix \(\alpha, \beta, \gamma\) by demanding that the constrains be satisfied.  What is the resulting \(p_i\)?
} % makeproblem

\makeanswer{basicStatMech:problemSet6:1}{
\makeSubAnswer{}{basicStatMech:problemSet6:1a}

Writing
\begin{dmath}\label{eqn:basicStatMechProblemSet6Problem1:220}
f
= S - \alpha \sum_{j = 1}^\Omega p_j,
= -\sum_{j = 1}^\Omega p_j \lr{ \kB \ln p_j + \alpha },
\end{dmath}
our unconstrained minimization requires
\begin{dmath}\label{eqn:basicStatMechProblemSet6Problem1:240}
0
= \PD{p_i}{f}
=
-\lr{
\kB \lr{ \ln p_i + 1 } + \alpha
}.
\end{dmath}

Solving for \(p_i\) we have
\begin{dmath}\label{eqn:basicStatMechProblemSet6Problem1:260}
p_i = e^{-\alpha/\kB - 1}.
\end{dmath}

The probabilities for each state are constant.  To fix that constant we employ our constraint
\begin{dmath}\label{eqn:basicStatMechProblemSet6Problem1:280}
1
= \sum_{j = 1}^\Omega p_j
= \sum_{j = 1}^\Omega e^{-\alpha/\kB - 1}
= \Omega e^{-\alpha/\kB - 1},
\end{dmath}
or
\begin{dmath}\label{eqn:basicStatMechProblemSet6Problem1:300}
\alpha/\kB + 1 = \ln \Omega.
\end{dmath}

Inserting \eqnref{eqn:basicStatMechProblemSet6Problem1:300} fixes the probability, giving us the first of the expected results

\boxedEquation{eqn:basicStatMechProblemSet6Problem1:320}{
p_i = e^{-\ln \Omega} = \inv{\Omega}.
}

Using this we our Gibbs entropy can be summed easily
\begin{dmath}\label{eqn:basicStatMechProblemSet6Problem1:340}
S
= -\kB
\sum_{j = 1}^\Omega p_j \ln p_j
= -\kB
\sum_{j = 1}^\Omega \inv{\Omega} \ln \inv{\Omega}
= -\kB \frac{\Omega}{\Omega} \lr{ -\ln \Omega },
\end{dmath}

or

\boxedEquation{eqn:basicStatMechProblemSet6Problem1:360}{
S = \kB \ln \Omega.
}

\makeSubAnswer{}{basicStatMech:problemSet6:1b}

For the ``action'' like quantity that we want to minimize, let's write
\begin{dmath}\label{eqn:basicStatMechProblemSet6Problem1:380}
f = S - \alpha \sum_j p_j - \beta \sum_j E_j p_j,
\end{dmath}
for which we seek \(\alpha\), \(\beta\) such that
\begin{dmath}\label{eqn:basicStatMechProblemSet6Problem1:400}
0
= \PD{p_i}{f}
= -\PD{p_i}{}
\sum_j
p_j
\lr{
\kB \ln p_j + \alpha + \beta E_j
}
=
-\kB (\ln p_i + 1) - \alpha - \beta E_i,
\end{dmath}
or
\begin{dmath}\label{eqn:basicStatMechProblemSet6Problem1:420}
p_i =
\exp\lr{
- \lr{\alpha - \beta E_i}/\kB - 1
}.
\end{dmath}

Our probability constraint is
\begin{dmath}\label{eqn:basicStatMechProblemSet6Problem1:440}
1
= \sum_j
\exp\lr{
- \lr{\alpha - \beta E_j}/\kB - 1
}
=
\exp\lr{
- \alpha/\kB - 1
}
\sum_j
\exp\lr{
- \beta E_j/\kB
},
\end{dmath}
or
\begin{dmath}\label{eqn:basicStatMechProblemSet6Problem1:460}
\exp\lr{
\alpha/\kB + 1
}
=
\sum_j
\exp\lr{
- \beta E_j/\kB
}.
\end{dmath}

Taking logs we have
\begin{dmath}\label{eqn:basicStatMechProblemSet6Problem1:480}
\alpha/\kB + 1 =
\ln \sum_j
\exp\lr{
- \beta E_j/\kB
}.
\end{dmath}

We could continue to solve for \(\alpha\) explicitly but don't care any more than this.  Plugging back into the probability \eqnref{eqn:basicStatMechProblemSet6Problem1:420} obtained from the unconstrained minimization we have
\begin{dmath}\label{eqn:basicStatMechProblemSet6Problem1:500}
p_i =
\exp\lr{
-\ln \sum_j
\exp\lr{
- \beta E_j/\kB
}
}
\exp\lr{
- \beta E_i/\kB
},
\end{dmath}

or

\boxedEquation{eqn:basicStatMechProblemSet6Problem1:520}{
p_i =
\frac{
   \exp\lr{
      - \beta E_i/\kB
   }
}
{
   \sum_j
   \exp\lr{
      - \beta E_j/\kB
   }
}.
}

To determine \(\beta\) we must look implicitly to the energy constraint, which is
\begin{dmath}\label{eqn:basicStatMechProblemSet6Problem1:540}
\expectation{E}
= \sum_i E_i p_i
=
\sum_i
E_i
\lr{
   \frac{
      \exp\lr{
         - \beta E_i/\kB
      }
   }
   {
      \sum_j
      \exp\lr{
         - \beta E_j/\kB
      }
   }
},
\end{dmath}

or
\boxedEquation{eqn:basicStatMechProblemSet6Problem1:560}{
\expectation{E} =
\frac{
   \sum_i E_i \exp\lr{ - \beta E_i/\kB }
}
{
   \sum_j \exp\lr{ - \beta E_j/\kB }
}.
}

The constraint \(\beta\) (\(=1/T\)) is given implicitly by this energy constraint.

\makeSubAnswer{}{basicStatMech:problemSet6:1c}

Again write
\begin{dmath}\label{eqn:basicStatMechProblemSet6Problem1:580}
f = S - \alpha \sum_j p_j - \beta \sum_j E_j p_j - \gamma \sum_j N_j p_j.
\end{dmath}

The unconstrained minimization requires
\begin{dmath}\label{eqn:basicStatMechProblemSet6Problem1:600}
0
= \PD{p_i}{f}
= -\PD{p_i}{}
\lr{
\kB (\ln p_i + 1) + \alpha + \beta E_i + \gamma N_i
},
\end{dmath}
or
\begin{dmath}\label{eqn:basicStatMechProblemSet6Problem1:620}
p_i
=
\exp\lr{ -\alpha/\kB - 1 }
\exp\lr{ -(\beta E_i + \gamma N_i)/\kB }.
\end{dmath}

The unit probability constraint requires
\begin{dmath}\label{eqn:basicStatMechProblemSet6Problem1:640}
1
= \sum_j p_j
=
\exp\lr{ -\alpha/\kB - 1 }
\sum_j
\exp\lr{ -(\beta E_j + \gamma N_j)/\kB },
\end{dmath}

or
\begin{dmath}\label{eqn:basicStatMechProblemSet6Problem1:660}
\exp\lr{ -\alpha/\kB - 1 } =
\inv{
\sum_j
\exp\lr{ -(\beta E_j + \gamma N_j)/\kB }
}.
\end{dmath}

Our probability is then

\boxedEquation{eqn:basicStatMechProblemSet6Problem1:680}{
p_i =
\frac{
\exp\lr{ -(\beta E_i + \gamma N_i)/\kB }
}{
\sum_j
\exp\lr{ -(\beta E_j + \gamma N_j)/\kB }
}.
}

The average energy \(\expectation{E} = \sum_j p_j E_j\) and average number of particles \(\expectation{N} = \sum_j p_j N_j\) are given by
\begin{subequations}
\begin{dmath}\label{eqn:basicStatMechProblemSet6Problem1:700}
\expectation{E} =
\frac{
E_i
\exp\lr{ -(\beta E_i + \gamma N_i)/\kB }
}{
\sum_j
\exp\lr{ -(\beta E_j + \gamma N_j)/\kB }
}
\end{dmath}
\begin{dmath}\label{eqn:basicStatMechProblemSet6Problem1:720}
\expectation{N} =
\frac{
N_i
\exp\lr{ -(\beta E_i + \gamma N_i)/\kB }
}{
\sum_j
\exp\lr{ -(\beta E_j + \gamma N_j)/\kB }
}.
\end{dmath}
\end{subequations}

The values \(\beta\) and \(\gamma\) are fixed implicitly by requiring simultaneous solutions of these equations.
}
