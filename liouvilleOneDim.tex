%
% Copyright � 2013 Peeter Joot.  All Rights Reserved.
% Licenced as described in the file LICENSE under the root directory of this GIT repository.
%
%\input{../blogpost.tex}
%\renewcommand{\basename}{liouvilleOneDim}
%\renewcommand{\dirname}{notes/phy452/}
%%\newcommand{\dateintitle}{}
%\newcommand{\keywords}{Liouville's theorem, Statistical mechanics, PHY452H1S, phase space, phase space current, phase space density}
%
%\input{../peeter_prologue_print2.tex}
%
%\beginArtNoToc
%
%\generatetitle{Liouville's theorem questions on density and current}
%\chapter{Liouville's theorem questions on density and current}
\label{chap:liouvilleOneDim}
\section{Liouville's theorem questions on density and current.}
%\section{Motivation}

In the midterm today we were asked to state and prove Liouville's theorem.  I couldn't remember the proof, having only a recollection that it had something to do with the continuity equation
\begin{equation}\label{eqn:liouvilleOneDim:20}
0 = \PD{t}{\rho} + \PD{x}{j},
\end{equation}
but unfortunately couldn't remember what the \(j\) was.  Looking up the proof, it's actually really simple, just the application of chain rule for a function \(\rho\) that's presumed to be a function of time, position and momentum variables.  It didn't appear to me that this proof has anything to do with any sort of notion of density, so I posed the following questions.
%
\paragraph{Context}
%
The core of the proof can be distilled to one dimension, removing all the indices that obfuscate what's being one.  For that case, application of the chain rule to a function \(\rho(t, x, p)\), we have
\begin{dmath}\label{eqn:liouvilleOneDim:40}
\ddt{\rho}
= \PD{t}{\rho} + \PD{t}{x} \PD{x}{\rho} + \PD{t}{p} \PD{p}{\rho}
= \PD{t}{\rho} + \xdot \PD{x}{\rho} + \pdot \PD{p}{\rho}
= \PD{t}{\rho} + \PD{x}{\lr{\xdot \rho}} + \PD{p}{\lr{\xdot \rho}} - \rho \lr{
\PD{x}{\xdot}
+
\PD{p}{\pdot}
}
= \PD{t}{\rho} + \PD{x}{\lr{\xdot \rho}} + \PD{p}{\lr{\xdot \rho}} - \rho
\mathLabelBox
[
   labelstyle={xshift=2cm},
   linestyle={out=270,in=90, latex-}
]
{
\lr{
\PD{x}{}
\lr{ \PD{p}{H} }
+
\PD{p}{}
\lr{ -\PD{x}{H} }
}
}{\(= 0\)}.
\end{dmath}
%
\paragraph{Wrong interpretation}
%
From this I'd thought that the theorem was about steady states.  If we do have a steady state, where \(d\rho/dt = 0\) we have
\begin{equation}\label{eqn:liouvilleOneDim:60}
0 = \PD{t}{\rho} + \PD{x}{\lr{\xdot \rho}} + \PD{p}{\lr{\pdot \rho}}.
\end{equation}

That would answer the question of what the current is, it's this tuple
\begin{equation}\label{eqn:liouvilleOneDim:80}
\Bj = (\rho \xdot, \rho \pdot),
\end{equation}
so if we introduce a ``phase space'' gradient
\begin{equation}\label{eqn:liouvilleOneDim:100}
\spacegrad = \lr{ \PD{x}{}, \PD{p}{} },
\end{equation}
we've got something that looks like a continuity equation
\begin{equation}\label{eqn:liouvilleOneDim:120}
0 = \PD{t}{\rho} + \spacegrad \cdot \Bj.
\end{equation}

Given this misinterpretation of the theorem, I had the following two questions

\begin{enumerate}
\item This function \(\rho\) appears to be pretty much arbitrary.  I don't see how this connects to any notion of density?
\item If we pick a specific Hamiltonian, say the 1D SHO, what physical interpretation do we have for this ``current'' \(\Bj\)?
\end{enumerate}
%
\paragraph{The clarification}
%
Asking about this, the response was ``Actually, \eqnref{eqn:liouvilleOneDim:60} has to be assumed for the proof.  This equation holds if \(\rho\) is the phase space density and since the pair in \eqnref{eqn:liouvilleOneDim:80} is the current density in phase space.  The theorem then states that \(d\rho/dt = 0\) whether or not one is in the steady state.  This means even as the system is evolving in time, if we sit on a particular phase space point and follow it around as it evolves, the density in our neighbourhood will be a constant.''

%\EndArticle
%\EndNoBibArticle
