%
% Copyright � 2013 Peeter Joot.  All Rights Reserved.
% Licenced as described in the file LICENSE under the root directory of this GIT repository.
%
%\chapter{Midterm 1 review}

\begin{enumerate}

\item Random walk

This was a unit stepping problem as illustrated in \cref{fig:lecture8:lecture8Fig2}.
\imageFigure{../figures/phy452-basicstatmech/lecture8Fig2}{Unit one dimensional random walk}{fig:lecture8:lecture8Fig2}{0.1}
\begin{dmath}\label{eqn:midterm1review:120b}
\expectation{X} = \sum_i \expectation{x_i}
\end{dmath}

\begin{dmath}\label{eqn:midterm1review:120}
\expectation{X^2} =
\expectation{
\lr{\sum_{i = 1}^N x_i}
\lr{\sum_{j = 1}^N x_i}
}
=
\expectation{
\lr{\sum_{i = 1}^N x_i^2}
+
\lr{\sum_{i \ne j = 1}^N x_i x_j}
}
\end{dmath}

\item State and prove Liouville's theorem

\begin{dmath}\label{eqn:midterm1review:140}
\frac{d\rho}{dt} =
\PD{t}{\rho} +
\sum_{i = 1}^{3N}
\PD{x_i}{
\lr{\dot{x}_i \rho}
}
+
\PD{x_i}{
\lr{\dot{x}_i \rho}
}.
\end{dmath}

\item Harmonic oscillator in 1D.

This problem ends up essentially requiring the evaluation of the area in phase space of an ellipse in phase space as in \cref{fig:lecture8:lecture8Fig4}.

\imageFigure{../figures/phy452-basicstatmech/lecture8Fig4}{1D classical SHO phase space}{fig:lecture8:lecture8Fig4}{0.2}

The result is of the form \(\pi a b \times \text{area}\) (the exact expression should be that of \eqnref{eqn:3nParticlePhaseSpaceVolume:100}, with \(N = 1\)).

\end{enumerate}
