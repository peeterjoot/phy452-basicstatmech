%
% Copyright � 2013 Peeter Joot.  All Rights Reserved.
% Licenced as described in the file LICENSE under the root directory of this GIT repository.
%
%\input{../blogpost.tex}
%\renewcommand{\basename}{basicStatMechLecture19}
%\renewcommand{\dirname}{notes/phy452/}
%\newcommand{\keywords}{Statistical mechanics, PHY452H1S, Boson, Bose condensate, fugacity, occupation number, density, zeta function}
%\input{../peeter_prologue_print2.tex}
%
%\beginArtNoToc
%\generatetitle{PHY452H1S Basic Statistical Mechanics.  Lecture 19: Bosons.  Taught by Prof.\ Arun Paramekanti}
%%\chapter{Bosons}
%\label{chap:basicStatMechLecture19}
%
%\section{Disclaimer}
%
%Peeter's lecture notes from class.  May not be entirely coherent.

\paragraph{Summary of our Fermion approach}

We've considered a momentum sphere as in \cref{fig:lecture19:lecture19Fig1}, and performed various appromations of the occupation sums \cref{fig:lecture19:lecture19Fig2}.
\imageFigure{../figures/phy452-basicstatmech/lecture19Fig1}{Summation over momentum sphere.}{fig:lecture19:lecture19Fig1}{0.2}
\imageFigure{../figures/phy452-basicstatmech/lecture19Fig2}{Fermion occupation.}{fig:lecture19:lecture19Fig2}{0.2}
\begin{subequations}
\begin{dmath}\label{eqn:basicStatMechLecture19:20}
\epsilon \sim T^2
\end{dmath}
\begin{dmath}\label{eqn:basicStatMechLecture19:40}
C \sim T
\end{dmath}
\begin{dmath}\label{eqn:basicStatMechLecture19:60}
P \sim \text{constant}.
\end{dmath}
\end{subequations}

The physics of Fermi gases has an extremely wide range of applicability.  Illustrating some of this range, here are some examples of Fermi temperatures (from \(E_{\txtF} = \kB T_{\txtF}\))
\begin{itemize}
\item Electrons in copper: \(T_{\txtF} \sim 10^4 \mbox{K}\)
\item Neutrons in neutron star: \(T_{\txtF} \sim 10^7 - 10^8 \mbox{K}\)
\item Ultracold atomic gases: \(T_{\txtF} \sim (10 - 100) \mbox{n K}\)
\end{itemize}

\paragraph{Moving on to Bosons}

We'd like to work with a fixed number of particles, but the calculations are hard, so we move to the grand canonical ensemble
\begin{dmath}\label{eqn:basicStatMechLecture19:200}
n_{\txtB}(\Bk) = \inv{ e^{\beta(\epsilon_\Bk - \mu)} - 1 }.
\end{dmath}

Again, we'll consider free particles with energy as in \cref{fig:lecture19:lecture19Fig3}, or
\begin{dmath}\label{eqn:basicStatMechLecture19:100}
\epsilon_\Bk = \frac{\Hbar^2 k^2}{2 m}.
\end{dmath}

\imageFigure{../figures/phy452-basicstatmech/lecture19Fig3}{Free particle energy momentum distribution.}{fig:lecture19:lecture19Fig3}{0.2}

Again introducing fugacity \(z = e^{\beta \mu}\), we have
\begin{dmath}\label{eqn:basicStatMechLecture19:220}
n_{\txtB}(\Bk) = \inv{ z^{-1} e^{\beta \epsilon_\Bk} - 1 }.
\end{dmath}

We'll consider systems for which
\begin{dmath}\label{eqn:basicStatMechLecture19:120}
N = \sum_\Bk n_{\txtB}(\Bk) = \text{fixed}.
\end{dmath}

Observe that at large energies we have
\begin{dmath}\label{eqn:basicStatMechLecture19:240}
n_{\txtB}(\text{large} \, \Bk) \sim z e^{-\beta \epsilon_\Bk}.
\end{dmath}
For small energies
\begin{dmath}\label{eqn:basicStatMechLecture19:260}
n_{\txtB}(\Bk \rightarrow 0) \sim \inv{z^{-1} - 1} = \frac{z}{1 - z}.
\end{dmath}

Observe that we require \(z < 1\) (or \(\mu < 0\)) so that the number distribution is strictly positive for all energies.  This tells us that the fugacity is a function of temperature, but there will be a point at which it must saturate.  This is illustrated in \cref{fig:lecture19:lecture19Fig4}.
\imageFigure{../figures/phy452-basicstatmech/lecture19Fig4}{Density times cubed thermal de Broglie wavelength.}{fig:lecture19:lecture19Fig4}{0.2}

Let's calculate this density (assumed fixed for all temperatures)
\begin{dmath}\label{eqn:basicStatMechLecture19:140}
\rho = \frac{N}{V}
= \int \frac{d^3 \Bk}{(2 \pi)^3} \inv{z^{-1} e^{\beta \epsilon_\Bk} -1 }
= \frac{2}{(2 \pi)^2}
\int_0^\infty k^2 dk \inv{z^{-1} e^{\beta \Hbar^2 k^2/2m} -1 }
= \frac{2}{(2 \pi)^2}
\lr{
\frac
{2 m}
{\beta \Hbar^2}
}
^{3/2}
\int_0^\infty
\lr{
\frac
{\beta \Hbar^2}
{2 m}
}
^{3/2}
k^2 dk \inv{z^{-1} e^{\beta \Hbar^2 k^2/2m} -1 }.
\end{dmath}

With the substitution
\begin{dmath}\label{eqn:basicStatMechLecture19:160}
x^2 = \beta \frac{\Hbar^2 k^2}{2m},
\end{dmath}
we find
\begin{dmath}\label{eqn:basicStatMechLecture19:180}
\rho \lambda^3
= \frac{2}{(2 \pi)^2}
\lr{
\frac
{2 \cancel{m}}
{\cancel{\beta \Hbar^2}}
}
^{3/2}
\lr{ \frac{ 2 \pi \cancel{\Hbar^2 \beta}}{\cancel{m}} }^{3/2}
\int_0^\infty
x^2 dx \inv{z^{-1} e^{x^2} -1 }
= \frac{4}{\sqrt{\pi}} \int_0^\infty dx \frac{x^2}{z^{-1} e^{x^2} - 1 }
\equiv g_{3/2}(z).
\end{dmath}

An exact plot of this is shown in \cref{fig:lecture19f32:lecture19f32Fig1}.

\imageFigure{../figures/phy452-basicstatmech/lecture19f32Fig1}{\(g_{3/2}(z)\).}{fig:lecture19f32:lecture19f32Fig1}{0.2}

This implicitly defines a relationship for the fugacity as a function of temperature \(z = z(T)\).
It can be shown that
\begin{dmath}\label{eqn:basicStatMechLecture19:280}
g_{3/2}(z)
= z
+ \frac{z^2}{2^{3/2}}
+ \frac{z^3}{3^{3/2}}
+ \cdots
\end{dmath}

As \(z \rightarrow 1\) we end up with a zeta function, for which we can look up the value
\begin{dmath}\label{eqn:basicStatMechLecture19:300}
g_{3/2}(z \rightarrow 1)
=
\sum_{n = 1}^\infty \inv{n^{3/2}}
= \zeta(3/2)
\approx 2.612,
\end{dmath}
where the \underlineAndIndex{Riemann zeta function} is defined as
\begin{dmath}\label{eqn:basicStatMechLecture19:320}
\zeta(s) = \sum_{ n = 1 } \inv{n^s}.
\end{dmath}
\begin{dmath}\label{eqn:basicStatMechLecture19:340}
g_{3/2}(z) = \rho \lambda^3.
\end{dmath}

At high temperatures we have
\begin{dmath}\label{eqn:basicStatMechLecture19:360}
\rho \lambda^3 \rightarrow 0,
\end{dmath}
(as \(T\) does down, \(\rho \lambda^3\) goes up)
Looking at \(g_{3/2}(z = 1) = \rho \lambda^3(T_{\txtc})\) leads to
\boxedEquation{eqn:basicStatMechLecture19:380}{
\kB T_{\txtc} =
\lr{ \frac{\rho}{\zeta(3/2)} }
^{2/3} \frac{ 2 \pi \Hbar^2}{m}.
}

\paragraph{How do I satisfy number conservation?}

We have a problem here since as \(T \rightarrow 0\) the \(1/\lambda^3 \sim T^{3/2}\) term in \(\rho\) above drops to zero, yet \(g_{3/2}(z)\) cannot keep increasing without bounds to compensate and keep the density fixed.  The way to deal with this was worked out by

\begin{itemize}
\item Bose (1924) for photons (examining statistics for symmetric wave functions).
\item Einstein (1925) for conserved particles.
\end{itemize}

To deal with this issue, we (somewhat arbitrarily, because we need to) introduce a non-zero density for \(\Bk = 0\).  This is an adjustment of the approximation so that we have
\begin{dmath}\label{eqn:basicStatMechLecture19:400}
\sum_{\Bk} \rightarrow \int \frac{d^3 \Bk}{(2 \pi)^3} \qquad \mbox{Except around \(\Bk = 0\)},
\end{dmath}
as in \cref{fig:lecture19:lecture19Fig6}, so that
\imageFigure{../figures/phy452-basicstatmech/lecture19Fig6}{Momentum sphere with origin omitted.}{fig:lecture19:lecture19Fig6}{0.2}
\begin{dmath}\label{eqn:basicStatMechLecture19:420}
\sum_\Bk
= \lr{ \mbox{Contribution at \(\Bk = 0\)} } + V \int \frac{d^3 \Bk}{(2 \pi)^3}.
\end{dmath}

Given this, we have
\begin{dmath}\label{eqn:basicStatMechLecture19:420b}
N
= N_{\Bk = 0}
+ V \int \frac{d^3 \Bk}{(2 \pi)^3} n_{\txtB}(\Bk).
\end{dmath}

We can illustrate this as in \cref{fig:lecture19:lecture19Fig7}.
\imageFigure{../figures/phy452-basicstatmech/lecture19Fig7}{Boson occupation vs momentum.}{fig:lecture19:lecture19Fig7}{0.2}
\begin{dmath}\label{eqn:basicStatMechLecture19:420c}
\rho
= \rho_{\Bk = 0}
+ \inv{\lambda^3}
g_{3/2}(z)
= \rho_{\Bk = 0}
+
\frac{ \lambda(T_{\txtc}) }{ \lambda(T)}
\inv{ \lambda^3(T_{\txtc})}
g_{3/2}(z).
\end{dmath}

At \(T > T_{\txtc}\) we have \(\rho_{\Bk = 0}\), whereas at \(T < T_{\txtc}\) we must introduce a non-zero density if we want to be able to keep a constant density constraint.

%\EndNoBibArticle
