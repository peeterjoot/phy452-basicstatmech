%
% Copyright � 2013 Peeter Joot.  All Rights Reserved.
% Licenced as described in the file LICENSE under the root directory of this GIT repository.
%
%\input{../blogpost.tex}
%\renewcommand{\basename}{basicStatMechLecture11}
%\renewcommand{\dirname}{notes/phy452/}
%\newcommand{\keywords}{Statistical mechanics, PHY452H1S, Maxwell distribution, Hamiltonian, Partition function, phase space region, canonical ensemble, average energy, heat capacity}
%\input{../peeter_prologue_print2.tex}
%
%%\usepackage{textcomp}
%
%\beginArtNoToc
%\generatetitle{PHY452H1S Basic Statistical Mechanics.  Lecture 11: Statistical and thermodynamic connection.  Taught by Prof.\ Arun Paramekanti}
%%\chapter{Statistical and thermodynamic connection}
\label{chap:basicStatMechLecture11}

\section{Connections between statistical and thermodynamic views}

\begin{enumerate}
\item ``Heat''.  Disorganized energy.
\item \(S_{\text{Statistical entropy}}\).  This is the thermodynamic entropy introduced by Boltzmann (microscopic).
\end{enumerate}

\section{Ideal gas}

\begin{equation}\label{eqn:basicStatMechLecture11:20}
H = \sum_{i = 1}^N \frac{\Bp_i^2}{2m}
\end{equation}

\begin{equation}\label{eqn:basicStatMechLecture11:40}
\Omega(E) =
\inv{h^{3N} N!}
\int
d\Bx_1
d\Bx_2
\cdots
d\Bx_N
d\Bp_1
d\Bp_2
\cdots
d\Bp_N
\delta( E - H )
\end{equation}

Let's isolate the contribution of the Hamiltonian from a single particle and all the rest
\begin{equation}\label{eqn:basicStatMechLecture11:60}
H =
\frac{\Bp_1^2}{2m}
+
\sum_{i \ne 1}^N \frac{\Bp_i^2}{2m}
=
\frac{\Bp_1^2}{2m}
+
H',
\end{equation}
so that the number of states in the phase space volume in the phase space region associated with the energy is
\begin{dmath}\label{eqn:basicStatMechLecture11:80}
\Omega(N, E)
=
\frac{V^N}{h^{3N} N!}
\int
d\Bp_1
\int
d\Bp_2
d\Bp_3
\cdots
d\Bp_N
\delta( E - H' - H_1)
=
\frac{V^{N-1}}{h^{3(N-1)} (N-1)!} \frac{V}{h^3 N}
\int
d\Bp_1
\int
d\Bp_2
d\Bp_3
\cdots
d\Bp_N
\delta( E - H' - H_1)
=
\frac{ V }{ h^3 N} \int d\Bp_1 \Omega( N-1, E - H_1 ).
\end{dmath}
With entropy defined by
\begin{equation}\label{eqn:basicStatMechLecture11:220}
S = \kB \ln \Omega,
\end{equation}
we have
\begin{equation}\label{eqn:basicStatMechLecture11:100}
\Omega( N-1, E - H_1 )
=
\exp\lr{
\frac{1}{\kB} S
\lr{N-1, E - \frac{\Bp_1^2}{2m}}
},
\end{equation}
so that
\begin{equation}\label{eqn:basicStatMechLecture11:120}
\Omega(N, E)
=
\frac{ V }{ h^3 N} \int d\Bp_1
\exp\lr{
\frac{1}{\kB} S
\lr{N-1, E - \frac{\Bp_1^2}{2m}}
}.
\end{equation}

For \(N \gg 1\) and \(E \gg \Bp_1^2/2m\), the exponential can be approximated by
\begin{equation}\label{eqn:basicStatMechLecture11:140}
\exp\lr{
\frac{1}{\kB} S
\lr{N-1, E - \frac{\Bp_1^2}{2m}}
}
=
\exp\lr{
\frac{1}{\kB}
\lr{
S(N, E)
-
\lr{\PD{N}{S}}_{E, V}
-
\frac{\Bp_1^2}{2m}
\lr{\PD{E}{S}}_{N, V}
}
},
\end{equation}
so that
\begin{equation}\label{eqn:basicStatMechLecture11:160}
\Omega(N, E) =
\int d\Bp_1
\mathLabelBox
%[
%   labelstyle={xshift=2cm},
%   linestyle={out=270,in=90, latex-}
%]
{
\frac{ V }{ h^3 N}
e^{\frac{S}{\kB}(N, E)}
e^{-\inv{\kB}
\lr{\PD{N}{S}}_{E, V}
}
}{\(B\)}
e^{-
\frac{\Bp_1^2}{2m \kB}
\lr{\PD{E}{S}}_{N, V}
},
\end{equation}
or
\begin{equation}\label{eqn:basicStatMechLecture11:180}
\Omega(N, E) = B
\int d\Bp_1
e^{-
\frac{\Bp_1^2}{2m \kB}
\lr{\PD{E}{S}}_{N, V}
}.
\end{equation}

\begin{equation}\label{eqn:basicStatMechLecture11:200}
\calP(\Bp_1)
\propto
e^{-
\frac{\Bp_1^2}{2m \kB T}
}.
\end{equation}

This is the Maxwell distribution.

\section{Non-ideal gas.  General classical system}

%\cref{fig:lecture11:lecture11Fig1}.
\imageFigure{../figures/phy452-basicstatmech/lecture11Fig1}{Partitioning out a subset of a larger system}{fig:lecture11:lecture11Fig1}{0.3}

Breaking the system into a subsystem \(1\) and the reservoir \(2\) so that with
\begin{equation}\label{eqn:basicStatMechLecture11:240}
H = H_1 + H_2,
\end{equation}
we have
\begin{dmath}\label{eqn:basicStatMechLecture11:260}
\Omega(N, V, E)
=
\int
d\{x_1\}
d\{p_1\}
d\{x_2\}
d\{p_2\}
\delta( E - H_1 - H_2 ) \inv{ h^{3N_1} N_1! h^{3 N_2} N_2!}
\propto
\int
d\{x_1\}
d\{p_1\}
e^{\inv{\kB} S(E - H_1, N - N_1)}.
\end{dmath}

\begin{equation}\label{eqn:basicStatMechLecture11:280}
\Omega(N, V, E)
\sim
\int
d\{x_1\}
d\{p_1\}
\mathLabelBox{
e^{\inv{\kB}S(E, N)}
e^{-\frac{N_1 }{\kB}
\lr{
\PD{N}{S}
}_{E, V}
}
}{``environment'', or ``heat bath''}
e^{-\frac{H_1 }{\kB}
\lr{
\PD{E}{S}
}_{N, V}
}.
\end{equation}

\begin{equation}\label{eqn:basicStatMechLecture11:300}
H_1 =
\sum_{i \in 1} \frac{\Bp_i}{2m}
+\sum_{i \in j} V(\Bx_i - \Bx_j)
+ \sum_{i \in 1} \Phi(\Bx_i).
\end{equation}

\begin{equation}\label{eqn:basicStatMechLecture11:320}
\calP \propto
e^{-\frac{H( \{x_1\} \{p_1\} ) }{\kB T} },
\end{equation}
and for the subsystem
\begin{equation}\label{eqn:basicStatMechLecture11:340}
\calP_1
=
\frac{e^{-\frac{H_1}{\kB T} }}{
\int
d\{x_1\}
d\{p_1\}
e^{-\frac{H_1}{\kB T} }
}
\end{equation}

\section{Canonical ensemble}

Can we use results for this subvolume, can we use this to infer results for the entire system?  Suppose we break the system into a number of smaller subsystems as in \cref{fig:lecture11:lecture11Fig2}.
\imageFigure{../figures/phy452-basicstatmech/lecture11Fig2}{Larger system partitioned into many small subsystems}{fig:lecture11:lecture11Fig2}{0.3}
\begin{equation}\label{eqn:basicStatMechLecture11:360}
\mathLabelBox{
(N, V, E)
}{microcanonical}
\rightarrow (N, V, T)
\end{equation}

We'd have to understand how large the differences between the energy fluctuations of the different subsystems are.  We've already assumed that we have minimal long range interactions since we've treated the subsystem \(1\) above in isolation.  With \(\beta = 1/(\kB T)\) the average energy is
\begin{equation}\label{eqn:basicStatMechLecture11:380}
\expectation{E}
=
\frac{
\int
d\{x_1\}
d\{p_1\}
H
e^{- \beta H }
}{
\int
d\{x_1\}
d\{p_1\}
e^{- \beta H }
}
\end{equation}

\begin{equation}\label{eqn:basicStatMechLecture11:400}
\expectation{E^2}
=
\frac{
\int
d\{x_1\}
d\{p_1\}
H^2
e^{- \beta H }
}{
\int
d\{x_1\}
d\{p_1\}
e^{- \beta H }
}
\end{equation}

We define the \underlineAndIndex{partition function}

\boxedEquation{eqn:basicStatMechLecture11:420}{
Z \equiv
\inv{h^{3N} N!}
\int
d\{x_1\}
d\{p_1\}
e^{- \beta H }.
}

Observe that the derivative of \(Z\) is
\begin{equation}\label{eqn:basicStatMechLecture11:540}
\PD{\beta}{Z} =
-\inv{h^{3N} N!}
\int
d\{x_1\}
d\{p_1\}
H
e^{- \beta H },
\end{equation}

allowing us to express the average energy compactly in terms of the partition function

\boxedEquation{eqn:basicStatMechLecture11:440}{
\expectation{E} = -\inv{Z} \PD{\beta}{Z} = - \PD{\beta}{\ln Z}.
}

Taking second derivatives we find the variance of the energy
\begin{dmath}\label{eqn:basicStatMechLecture11:480}
\PDSq{\beta}{\ln Z}
=
\PD{\beta}{}
\frac{
\int
d\{x_1\}
d\{p_1\}
(-H)
e^{- \beta H }
}
{
\int
d\{x_1\}
d\{p_1\}
e^{- \beta H }
}
=
\frac{
\int
d\{x_1\}
d\{p_1\}
(-H)^2
e^{- \beta H }
}
{
\int
d\{x_1\}
d\{p_1\}
e^{- \beta H }
}
-
\frac{
\lr{
\int
d\{x_1\}
d\{p_1\}
(-H)
e^{- \beta H }
}^2
}
{
\lr{
\int
d\{x_1\}
d\{p_1\}
e^{- \beta H }
}^2
}
\end{dmath}

We have a second thermodynamic result directly obtained from the partition function

\boxedEquation{eqn:basicStatMechLecture11:481}{
\sigma_{\txtE}^2
= \expectation{E^2} - \expectation{E}^2
=
\PDSq{\beta}{\ln Z}
}

We also have
\begin{dmath}\label{eqn:basicStatMechLecture11:460}
\sigma_{\txtE}^2 =
-\PD{\beta}{\expectation{E}} =
\PD{T}{\expectation{E}}
\PD{\beta}{T}
=
-\PD{T}{\expectation{E}}
\PD{\beta}{} \inv{\kB \beta}
=
\PD{T}{\expectation{E}}
\inv{\kB \beta^2}
= \kB T^2 \PD{T}{\expectation{E}}
\end{dmath}

Recalling that the heat capacity was defined by
\begin{equation}\label{eqn:basicStatMechLecture11:560}
\CV = \PD{T}{\expectation{E}},
\end{equation}
we have
\begin{equation}\label{eqn:basicStatMechLecture11:500}
\sigma_{\txtE}^2
= \kB T^2 \CV \propto N
\end{equation}

\begin{equation}\label{eqn:basicStatMechLecture11:520}
\frac{\sigma_{\txtE}}{\expectation{E}} \propto \inv{\sqrt{N}}
\end{equation}

\paragraph{Question:} What is this \(N\) proportionality?

The specific heat is the total energy change in response to a small change in temperature.  This specific heat \(\sim N\) since the energy change \(\sim N\).  So \(\sigma^2_E \sim N\).  This is just the extensivity of the energy for any system with short range interactions.  Doubling the size doubles the energy content.

%\EndNoBibArticle
