%
% Copyright � 2013 Peeter Joot.  All Rights Reserved.
% Licenced as described in the file LICENSE under the root directory of this GIT repository.
%
\makeoproblem{Quantum electric dipole}{basicStatMech:problemSet5:3}{2013 ps5, p3}{
%(3 points)
A quantum electric dipole at a fixed space point has its energy determined by two parts - a part which comes from its angular motion and a part coming from its interaction with an applied electric field \(\calE\). This leads to a quantum Hamiltonian
\begin{equation}\label{eqn:basicStatMechProblemSet5Problem3:20}
H = \frac{\BL \cdot \BL}{2 I} - \mu \calE L_z,
\end{equation}

where \(I\) is the moment of inertia, and we have assumed an electric field \(\calE = \calE \zcap\).  This Hamiltonian has eigenstates described by spherical harmonics \(Y_{l, m}(\theta, \phi)\), with \(m\) taking on \(2l+1\) possible integral values, \(m = -l, -l + 1, \cdots, l -1, l\).  The corresponding eigenvalues are
\begin{equation}\label{eqn:basicStatMechProblemSet5Problem3:40}
\lambda_{l, m} = \frac{l(l+1) \Hbar^2}{2I} - \mu \calE m \Hbar.
\end{equation}

(Recall that \(l\) is the total angular momentum eigenvalue, while \(m\) is the eigenvalue corresponding to \(L_z\).)

\makesubproblem{}{pr:basicStatMechProblemSet5Problem3:a}
Schematically sketch these eigenvalues as a function of \(\calE\) for \(l = 0,1,2\).

\makesubproblem{}{pr:basicStatMechProblemSet5Problem3:b}
Find the quantum partition function, assuming only \(l = 0\) and \(l = 1\) contribute to the sum.

\makesubproblem{}{pr:basicStatMechProblemSet5Problem3:c}
Using this partition function, find the average dipole moment \(\mu \expectation{L_z}\) as a function of the electric field and temperature for small electric fields, commenting on its behaviour at very high temperature and very low temperature.

\makesubproblem{}{pr:basicStatMechProblemSet5Problem3:d}
Estimate the temperature above which discarding higher angular momentum states, with \(l \ge 2\), is not a good approximation.
} % makeproblem

\makeanswer{basicStatMech:problemSet5:3}{
\makeSubAnswer{Sketch the energy eigenvalues}{pr:basicStatMechProblemSet5Problem3:a}

Let's summarize the values of the energy eigenvalues \(\lambda_{l,m}\) for \(l = 0, 1, 2\) before attempting to plot them.
%
\paragraph{\(l = 0\)}
%
For \(l = 0\), the azimuthal quantum number can only take the value \(m = 0\), so we have
\begin{equation}\label{eqn:basicStatMechProblemSet5Problem3:60}
\lambda_{0,0} = 0.
\end{equation}
%
\paragraph{\(l = 1\)}
%
For \(l = 1\) we have
\begin{equation}\label{eqn:basicStatMechProblemSet5Problem3:80}
\frac{l(l+1)}{2} = 1(2)/2 = 1,
\end{equation}
so we have
\begin{subequations}
\begin{equation}\label{eqn:basicStatMechProblemSet5Problem3:100}
\lambda_{1,0} = \frac{\Hbar^2}{I}
\end{equation}
\begin{equation}\label{eqn:basicStatMechProblemSet5Problem3:120}
\lambda_{1,\pm 1} = \frac{\Hbar^2}{I} \mp \mu \calE \Hbar.
\end{equation}
\end{subequations}
%
\paragraph{\(l = 2\)}
%
For \(l = 2\) we have
\begin{equation}\label{eqn:basicStatMechProblemSet5Problem3:140}
\frac{l(l+1)}{2} = 2(3)/2 = 3,
\end{equation}
so we have
\begin{subequations}
\begin{equation}\label{eqn:basicStatMechProblemSet5Problem3:160}
\lambda_{2,0} = \frac{3 \Hbar^2}{I}
\end{equation}
\begin{equation}\label{eqn:basicStatMechProblemSet5Problem3:180}
\lambda_{2,\pm 1} = \frac{3 \Hbar^2}{I} \mp \mu \calE \Hbar
\end{equation}
\begin{equation}\label{eqn:basicStatMechProblemSet5Problem3:200}
\lambda_{2,\pm 2} = \frac{3 \Hbar^2}{I} \mp 2 \mu \calE \Hbar.
\end{equation}
\end{subequations}

These are sketched as a function of \(\calE\) in \cref{fig:basicStatMechProblemSet5Problem3:basicStatMechProblemSet5Problem3Fig1}.

\imageFigure{../figures/phy452-basicstatmech/basicStatMechProblemSet5Problem3Fig1}{Energy eigenvalues for \(l = 0,1, 2\).}{fig:basicStatMechProblemSet5Problem3:basicStatMechProblemSet5Problem3Fig1}{0.4}

\makeSubAnswer{Partition function}{pr:basicStatMechProblemSet5Problem3:b}

Our partition function, in general, is
\begin{dmath}\label{eqn:basicStatMechProblemSet5Problem3:220}
Z
= \sum_{l = 0}^\infty \sum_{m = -l}^l e^{-\lambda_{l,m} \beta}
= \sum_{l = 0}^\infty \exp\lr{ -\frac{l (l+1) \Hbar^2 \beta}{2 I}} \sum_{m = -l}^l e^{ m \mu \Hbar \calE \beta}.
\end{dmath}

Dropping all but \(l = 0, 1\) terms this is
\begin{equation}\label{eqn:basicStatMechProblemSet5Problem3:380}
Z
\approx
1 + e^{-\Hbar^2 \beta/I}
\lr{ 1 + e^{- \mu \Hbar \calE \beta } + e^{ \mu \Hbar \calE \beta}},
\end{equation}

or
\boxedEquation{eqn:basicStatMechProblemSet5Problem3:400}{
Z
\approx
1 + e^{-\Hbar^2 \beta/I} (
1 + 2 \cosh
\lr{ \mu \Hbar \calE \beta}
).
}

\makeSubAnswer{Average dipole moment}{pr:basicStatMechProblemSet5Problem3:c}

For the average dipole moment, averaging over both the states and the partitions, we have
\begin{dmath}\label{eqn:basicStatMechProblemSet5Problem3:260}
Z \expectation{ \mu L_z }
= \sum_{l = 0}^\infty \sum_{m = -l}^l
\bra{l m} \mu L_z \ket{l m} e^{-\beta \lambda_{l, m}}
= \sum_{l = 0}^\infty \sum_{m = -l}^l
\mu \bra{l m} m \Hbar \ket{l m} e^{-\beta \lambda_{l, m}}
= \mu \Hbar \sum_{l = 0}^\infty \exp\lr{ -\frac{l (l+1) \Hbar^2 \beta}{2 I}}
\sum_{m = -l}^l
m e^{ \mu m \Hbar \calE \beta}
= \mu \Hbar \sum_{l = 0}^\infty \exp\lr{ -\frac{l (l+1) \Hbar^2 \beta}{2 I}}
\sum_{m = 1}^l
m
\lr{ e^{ \mu m \Hbar \calE \beta} -e^{-\mu m \Hbar \calE \beta} }
= 2 \mu \Hbar \sum_{l = 0}^\infty \exp\lr{ -\frac{l (l+1) \Hbar^2 \beta}{2 I}}
\sum_{m = 1}^l m \sinh (\mu m \Hbar \calE \beta).
\end{dmath}

For the cap of \(l = 1\) we have
\begin{dmath}\label{eqn:basicStatMechProblemSet5Problem3:280}
\expectation{ \mu L_z } \approx
\frac{2 \mu \Hbar }{Z}
\lr{ 1 (0) + e^{-\Hbar^2 \beta/ I} \sinh (\mu \Hbar \calE \beta) }
\approx
2 \mu \Hbar
\frac{
e^{-\Hbar^2 \beta/ I} \sinh (\mu \Hbar \calE \beta)
}
{
1 + e^{-\Hbar^2 \beta/I}
\lr{1 + 2 \cosh( \mu \Hbar \calE \beta)}
},
\end{dmath}

or
\boxedEquation{eqn:basicStatMechProblemSet5Problem3:300}{
\expectation{ \mu L_z } \approx
\frac{
2 \mu \Hbar
\sinh (\mu \Hbar \calE \beta)
}
{
e^{\Hbar^2 \beta/I}
+ 1 + 2 \cosh( \mu \Hbar \calE \beta)
}.
}

This is plotted in \cref{fig:basicStatMechProblemSet5Problem3:basicStatMechProblemSet5Problem3Fig3}.

\imageFigure{../figures/phy452-basicstatmech/basicStatMechProblemSet5Problem3Fig3A}{Dipole moment.}{fig:basicStatMechProblemSet5Problem3:basicStatMechProblemSet5Problem3Fig3}{0.2}

For high temperatures \(\mu \Hbar \calE \beta \ll 1\) or \(\kB T \gg \mu \Hbar \calE\), expanding the hyperbolic sine and cosines to first and second order respectively and the exponential to first order we have
\begin{dmath}\label{eqn:basicStatMechProblemSet5Problem3:720}
\expectation{ \mu L_z } \approx
2 \mu \Hbar
\frac
{
   \frac{\mu \Hbar \calE}{\kB T}
}
{
   4 + \frac{h^2}{I \kB T} +
   \lr{\frac{\mu \Hbar \calE}{\kB T} }^2
}
=
\frac{2 (\mu \Hbar)^2 \calE \kB T}{4 (\kB T)^2 + \Hbar^2 \kB T/I + (\mu \Hbar \calE)^2 }
\approx
\frac{(\mu \Hbar)^2 \calE}{4 \kB T}.
\end{dmath}

Our dipole moment tends to zero approximately inversely proportional to temperature.   These last two respective approximations are plotted along with the all temperature range result in \cref{fig:basicStatMechProblemSet5Problem3:basicStatMechProblemSet5Problem3Fig4A}.

\imageFigure{../figures/phy452-basicstatmech/basicStatMechProblemSet5Problem3Fig4A}{High temperature approximations to dipole moments.}{fig:basicStatMechProblemSet5Problem3:basicStatMechProblemSet5Problem3Fig4A}{0.2}
%\begin{equation}\label{eqn:basicStatMechProblemSet5Problem3:360}
%\expectation{ \mu L_z } \approx 2 \mu \Hbar \frac{0}{1 + 1 + 2} = 0.
%\end{equation}

For low temperatures \(\kB T \ll \mu \Hbar \calE\), where \(\mu \Hbar \calE \beta \gg 1\) we have
\begin{dmath}\label{eqn:basicStatMechProblemSet5Problem3:740}
\expectation{ \mu L_z } \approx
\frac
{ 2 \mu \Hbar e^{\mu \Hbar \calE \beta} }
{ e^{\Hbar^2 \beta/I} + e^{\mu \Hbar \calE \beta} }
=
\frac
{ 2 \mu \Hbar }
{ 1 + e^{ (\Hbar^2 \beta/I - \mu \Hbar \calE)/{\kB T} } }.
\end{dmath}

Provided the electric field is small enough (which means here that \(\calE < \Hbar/\mu I\)) this will look something like \cref{fig:basicStatMechProblemSet5Problem3:basicStatMechProblemSet5Problem3Fig5A}.

\imageFigure{../figures/phy452-basicstatmech/basicStatMechProblemSet5Problem3Fig5A}{Low temperature dipole moment behaviour.}{fig:basicStatMechProblemSet5Problem3:basicStatMechProblemSet5Problem3Fig5A}{0.2}

\makeSubAnswer{Approximation validation}{pr:basicStatMechProblemSet5Problem3:d}

In order to validate the approximation, let's first put the partition function and the numerator of the dipole moment into a tidier closed form, evaluating the sums over the radial indices \(l\).  First let's sum the exponentials for the partition function, making an \(n = m + l\)
\begin{dmath}\label{eqn:basicStatMechProblemSet5Problem3:420}
\sum_{m = -l}^l a^m
=
a^{-l} \sum_{n=0}^{2l} a^n
=
a^{-l} \frac{a^{2l + 1} - 1}{a - 1}
=
\frac{a^{l + 1} - a^{-l}}{a - 1}
=
\frac{a^{l + 1/2} - a^{-(l+1/2)}}{a^{1/2} - a^{-1/2}}.
\end{dmath}

With a substitution of \(a = e^b\), we have
\boxedEquation{eqn:basicStatMechProblemSet5Problem3:440}{
\sum_{m = -l}^l e^{b m}
=
\frac{
\sinh(b(l + 1/2))
}
{
\sinh(b/2)
}.
}

Now we can sum the azimuthal exponentials for the dipole moment.  This sum is of the form
\begin{dmath}\label{eqn:basicStatMechProblemSet5Problem3:460}
\sum_{m = -l}^l m a^m
= a
\lr{ \sum_{m = 1}^l + \sum_{m = -l}^{-1} }
m a^{m-1}
= a
\frac{d}{da}
\sum_{m = 1}^l
\lr{ a^{m} + a^{-m} }
=
a
\frac{d}{da}
\lr{ \sum_{m = -l}^l a^m - \cancel{1} }
=
a
\frac{d}{da}
\lr{ \frac{a^{l + 1/2} - a^{-(l+1/2)}}{a^{1/2} - a^{-1/2}} }.
\end{dmath}

With \(a = e^{b}\), and \(1 = a db/da\), we have
\begin{equation}\label{eqn:basicStatMechProblemSet5Problem3:480}
a \frac{d}{da} = a \frac{db}{da} \frac{d}{db} = \frac{d}{db},
\end{equation}
we have
\begin{equation}\label{eqn:basicStatMechProblemSet5Problem3:500}
\sum_{m = -l}^l m e^{b m}
= \frac{d}{db}
\lr{
\frac{
\sinh(b(l + 1/2))
}{
\sinh(b/2)
}
}.
\end{equation}

With a little help from Mathematica to simplify that result we have
\boxedEquation{eqn:basicStatMechProblemSet5Problem3:520}{
\sum_{m = -l}^l m e^{b m}
=
\frac
{
l \sinh(b (l+1)) - (l+1) \sinh(b l)
}
{
2 \sinh^2(b/2)
}
.
}

We can now express the average dipole moment with only sums over radial indices \(l\)
\begin{dmath}\label{eqn:basicStatMechProblemSet5Problem3:540}
\expectation{ \mu L_z }
=
\mu \Hbar
\frac{
   \sum_{l = 0}^\infty \exp\lr{ -\frac{l (l+1) \Hbar^2 \beta}{2 I}}
   \sum_{m = -l}^l m e^{ \mu m \Hbar \calE \beta}
}
{
   \sum_{l = 0}^\infty \exp\lr{ -\frac{l (l+1) \Hbar^2 \beta}{2 I}}
   \sum_{m = -l}^l e^{ m \mu \Hbar \calE \beta}
}
=
\mu \Hbar
\frac
{
   \sum_{l = 0}^\infty \exp\lr{ -\frac{l (l+1) \Hbar^2 \beta}{2 I}}
   \frac
   {
      l \sinh(\mu \Hbar \calE \beta (l+1)) - (l+1) \sinh(\mu \Hbar \calE \beta l)
   }
   {
      2 \sinh^2(\mu \Hbar \calE \beta/2)
   }
}
{
\sum_{l = 0}^\infty \exp\lr{ -\frac{l (l+1) \Hbar^2 \beta}{2 I}}
   \frac
   {
      \sinh(\mu \Hbar \calE \beta(l + 1/2))
   }
   {
      \sinh(\mu \Hbar \calE \beta/2)
   }
}.
\end{dmath}

So our average dipole moment, with \( x = \mu \Hbar \calE \beta \), is
\boxedEquation{eqn:basicStatMechProblemSet5Problem3:560}{
\expectation{ \mu L_z }
=
\frac
{
{\mu \Hbar }
   \sum_{l = 0}^\infty \exp\lr{ -\frac{l (l+1) \Hbar^2 \beta}{2 I}}
   \lr{ l \sinh(x (l+1)) - (l+1) \sinh(x l) }
}
{
2 \sinh(x/2)
   \sum_{l = 0}^\infty \exp\lr{ -\frac{l (l+1) \Hbar^2 \beta}{2 I}}
   \sinh(x(l + 1/2))
}.
}

The hyperbolic sine in the denominator from the partition function and the difference of hyperbolic sines in the numerator both grow fast.  This is illustrated in \cref{fig:basicStatMechProblemSet5Problem3:basicStatMechProblemSet5Problem3Fig2}.

\imageFigure{../figures/phy452-basicstatmech/basicStatMechProblemSet5Problem3Fig2A}{Hyperbolic sine plots for dipole moment.}{fig:basicStatMechProblemSet5Problem3:basicStatMechProblemSet5Problem3Fig2}{0.2}

Let's look at the order of these hyperbolic sines for large arguments.  For the numerator we have a difference of the form
\begin{dmath}\label{eqn:basicStatMechProblemSet5Problem3:580}
x \sinh( x + 1 ) - (x + 1) \sinh ( x )
=
\inv{2}
\lr{
x \lr{ e^{x + 1} - e^{-x - 1} }
-(x +1 )\lr{ e^{x } - e^{-x } }
}
\approx
\inv{2}
\lr{
x e^{x + 1}
-(x +1 ) e^{x }
}
=
\inv{2}
\lr{
x e^{x}  ( e - 1 ) - e^x
}
= O(x e^x).
\end{dmath}

For the hyperbolic sine from the partition function we have for large \(x\)
\begin{equation}\label{eqn:basicStatMechProblemSet5Problem3:600}
\sinh( x + 1/2)
= \inv{2} \lr{e^{x + 1/2} - e^{-x - 1/2}}
\approx \frac{\sqrt{e}}{2} e^{x}
= O(e^x).
\end{equation}

%In particular
%To be able to disregard the \(l \ge 2\) energy eigenstates we require
%
%\begin{equation}\label{eqn:basicStatMechProblemSet5Problem3:620}
%\mu \Hbar \calE \beta \ll 1,
%\end{equation}

While these hyperbolic sines increase without bound as \(l\) increases, we have a negative quadratic dependence on \(l\) in the \(\BL^2\) contribution to these sums, provided that is small enough we can neglect the linear growth of the hyperbolic sines.  We wish for that factor to be large enough that it dominates for all \(l\).  That is
\begin{equation}\label{eqn:basicStatMechProblemSet5Problem3:640}
\frac{l(l+1) \Hbar^2}{2 I \kB T} \gg 1,
\end{equation}
or
\begin{equation}\label{eqn:basicStatMechProblemSet5Problem3:660}
T \ll \frac{l(l+1) \Hbar^2}{2 I \kB T}.
\end{equation}

Observe that the RHS of this inequality, for \(l = 1, 2, 3, 4, \cdots\) satisfies
\begin{equation}\label{eqn:basicStatMechProblemSet5Problem3:680}
\frac{\Hbar^2 }{I \kB}
<
\frac{3 \Hbar^2 }{I \kB}
<
\frac{6 \Hbar^2 }{I \kB}
<
\frac{10 \Hbar^2 }{I \kB}
< \cdots
\end{equation}

So, for small electric fields, our approximation should be valid provided our temperature is constrained by
\boxedEquation{eqn:basicStatMechProblemSet5Problem3:700}{
T \ll \frac{\Hbar^2 }{I \kB}.
}
}
