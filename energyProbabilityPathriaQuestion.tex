%
% Copyright � 2013 Peeter Joot.  All Rights Reserved.
% Licenced as described in the file LICENSE under the root directory of this GIT repository.
%
%\input{../blogpost.tex}
%\renewcommand{\basename}{energyProbabilityPathriaQuestion}
%\renewcommand{\dirname}{notes/phy452/}
%\newcommand{\keywords}{Statistics mechanics, PHY452H1S, Boltzmann factor, mean energy, specific heat}
%
%\input{../peeter_prologue_print2.tex}
%
%\beginArtNoToc
%
%\generatetitle{Unresolved question about energy distribution around mean energy}
\label{chap:energyProbabilityPathriaQuestion}
%
In \citep{pathriastatistical} is an expansion of
\begin{equation}\label{eqn:energyProbabilityPathriaQuestion:20}
P(E) \propto e^{-\beta E} g(E),
\end{equation}
around the mean energy \(E^\conj = U\).  The first derivative part of the expansion is simple enough
\begin{dmath}\label{eqn:energyProbabilityPathriaQuestion:40}
\PD{E}{}
\lr{ e^{-\beta E} g(E) }
=
\lr{ -\beta g(E) + g'(E)}
e^{-\beta E}
=
g(E) e^{-\beta E}
\lr{ -\beta + (\ln g(E))' }.
\end{dmath}
%
The peak energy \(E^\conj\) will be where this derivative equals zero.  That is
\begin{equation}\label{eqn:energyProbabilityPathriaQuestion:60}
0 = g(E^\conj) e^{-\beta E^\conj}
\lr{ -\beta + \evalbar{(\ln g(E))'}{E = E^\conj} },
\end{equation}
or
\begin{equation}\label{eqn:energyProbabilityPathriaQuestion:80}
\evalbar{
\PD{E}{}
\lr{\ln g(E)}
}{E = E^\conj} = \beta.
\end{equation}
%
With
\begin{subequations}
\begin{equation}\label{eqn:energyProbabilityPathriaQuestion:100}
S = \kB \ln g
\end{equation}
\begin{equation}\label{eqn:energyProbabilityPathriaQuestion:120}
\inv{\kB} \lr{\PD{E}{S}}_{E = U} = \inv{\kB T} = \beta.
\end{equation}
\end{subequations}
%
We have
\begin{equation}\label{eqn:energyProbabilityPathriaQuestion:140}
\lr{\PD{E}{ \ln g(E) }}_{E = U} = \beta,
\end{equation}
so that
\begin{equation}\label{eqn:energyProbabilityPathriaQuestion:160}
E^\conj = U.
\end{equation}
%
So far so good.  Reading the text, the expansion of the logarithm of \(P(E)\) around \(E = E^\conj = U\) wasn't clear.  Let's write that out in full.  To two terms that is
\begin{dmath}\label{eqn:energyProbabilityPathriaQuestion:180}
\ln e^{-\beta E} g(E)
=
\mathLabelBox{
\ln e^{-\beta U} g(U)
}{
\(- \beta U + \inv{\kB} S\)
}
+
\evalbar{
\PD{E}{}
\lr{\ln e^{-\beta E} g(E)}
}{E = U}
+
\inv{2}
\evalbar{
\PDSq{E}{}
\lr{\ln e^{-\beta E} g(E)}
}{E = U}
(E - U)^2.
\end{dmath}
%
The first order term has the derivative of the logarithm of \(e^{-\beta E}g(E)\).  Since the logarithm is monotonic and the derivative of \(e^{-\beta E}g(E)\) has been shown to be zero at \(E = U\), this must be zero.  We can also see this explicitly by computation
\begin{dmath}\label{eqn:energyProbabilityPathriaQuestion:200}
\evalbar{
\PD{E}{} \ln e^{-\beta E} g(E)
}{E = U}
=
\evalbar{
\frac{
-\beta e^{-\beta E} g(E) + e^{-\beta E} g'(E)
}{
e^{-\beta E} g(E)
}
}{E = U}
=
\evalbar{
\frac{-\beta g + g'}{g}
}{E = U}
=
-\beta
+
\evalbar{
(\ln g)'
}{E = U}
=
-\beta + \inv{\kB} \evalbar{\PD{E}{S}}{E = U}
= -\beta + \inv{\kB T}
= -\beta + \beta
= 0.
\end{dmath}
%
For the second derivative we have
\begin{dmath}\label{eqn:energyProbabilityPathriaQuestion:220}
\evalbar{
\PD{E}{} \ln e^{-\beta E} g(E)
}{E = U}
=
\evalbar{
\PD{E}{}
\lr{
-\beta + (\ln g)'
}
}{E = U}
=
\evalbar{
\PD{E}{} \frac{g'}{g}
}{E = U}
=
\evalbar{
\frac{g''}{g} - \frac{(g')^2}{g^2}
}{E = U}
=
\evalbar{
\frac{g''}{g}
}{E = U}
 - ((\ln g)')^2
=
\evalbar{
\frac{g''}{g}
}{E = U}
 - \beta^2.
\end{dmath}
%
Somehow this is supposed to come out to \(\kB T^2 \CV\)?  Backing up, we have
\begin{dmath}\label{eqn:energyProbabilityPathriaQuestion:240}
\evalbar{
\PD{E}{} \ln e^{-\beta E} g(E)
}{E = U}
=
\evalbar{
\PDSq{E}{} \ln g
}{E = U}
=
\inv{\kB}
\evalbar{
\PDSq{E}{S}
}{E = U}.
\end{dmath}
%
I still don't see how to get \(\CV = \PDi{T}{U}\) out of this?  \(\CV\) is a derivative with respect to temperature, but here we have derivatives with respect to energy (keeping \(\beta = 1/\kB T\) fixed)?
%
%\EndArticle
