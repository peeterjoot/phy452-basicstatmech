%
% Copyright � 2013 Peeter Joot.  All Rights Reserved.
% Licenced as described in the file LICENSE under the root directory of this GIT repository.
%
\makeoproblem{State counting - spins}{basicStatMech:problemSet3:3}{2013 problem set III, p3}{
Consider a toy model of a magnet where the net magnetization arises from electronic spins on each atom which can point in one of only two possible directions - Up/North or Down/South.  If we have a system with \(N\) spins, and if the magnetization can only take on values \(\pm 1\) (Up = \(+1\), Down = \(-1\)), how many configurations are there which have a total magnetization \(m\), where \(m = (N_\uparrow - N_\downarrow)/N\) (note that \(N_\uparrow + N_\downarrow = N\))?  Simplify this result assuming \(N \gg 1\) and a generic \(m\) (assume we are not interested in the extreme case of a fully magnetized system where \(m = \pm 1\)).  Find the value of the magnetization \(m\) for which the number of such microscopic states is a maximum.  For \(N = 20\), make a numerical plot of the number of states as a function of the magnetization (note: \(-1 \le m \le 1\)) without making the \(N \gg 1\) assumption.
} % makeoproblem

\makeanswer{basicStatMech:problemSet3:3}{

For the first couple values of \(N\), lets enumerate the spin sample spaces, their magnetization.

\paragraph{\(N = 1\)}

\begin{itemize}
\item \(\uparrow\) : \(m = 1\)
\item \(\downarrow\), \(m = -1\)
\end{itemize}

\paragraph{\(N = 2\)}

\begin{itemize}
\item \(\uparrow \uparrow\) : \(m = 1\)
\item \(\uparrow \downarrow\) : \(m = 0\)
\item \(\downarrow \uparrow\), \(m = 0\)
\item \(\downarrow \downarrow\), \(m = -1\)
\end{itemize}

\paragraph{\(N = 3\)}

\begin{itemize}
\item \(\uparrow \uparrow \uparrow\) : \(m = 1\)
\item \(\uparrow \uparrow \downarrow\) : \(m = 1/3\)
\item \(\uparrow \downarrow \uparrow\) : \(m = 1/3\)
\item \(\uparrow \downarrow \downarrow\) : \(m = -1/3\)
\item \(\downarrow \uparrow \uparrow\) : \(m = 1/3\)
\item \(\downarrow \uparrow \downarrow\) : \(m = -1/3\)
\item \(\downarrow \downarrow \uparrow\) : \(m = -1/3\)
\item \(\downarrow \downarrow \downarrow\) : \(m = 1\)
\end{itemize}

The respective multiplicities for \(N = 1,2,3\) are \(\{1\}\), \(\{1, 2, 1\}\), \(\{1, 3, 3, 1\}\).  It's clear that these are just the binomial coefficients.  Let's write for the multiplicities
\begin{equation}\label{eqn:basicStatMechProblemSet3Problem3:20}
g(N, m) = \binom{N}{i(m)},
\end{equation}
where \(i(m)\) is a function that maps from the magnetization values \(m\) to the integers \([0, N]\).  Assuming
\begin{equation}\label{eqn:basicStatMechProblemSet3Problem3:40}
i(m) = a m + b,
\end{equation}
where \(i(-1) = 0\) and \(i(1) = N\), we solve
\begin{equation}\label{eqn:basicStatMechProblemSet3Problem3:60}
\begin{aligned}
a (-1) + b &= 0 \\
a (1) + b &= N,
\end{aligned}
\end{equation}
so
\begin{equation}\label{eqn:basicStatMechProblemSet3Problem3:80}
i(m) = \frac{N}{2}(m + 1),
\end{equation}
and
\begin{equation}\label{eqn:basicStatMechProblemSet3Problem3:100}
g(N, m) = \binom{N}{\frac{N}{2}(1 + m)} = \frac{N!}{
\left(\frac{N}{2}(1 + m)\right)!
\left(\frac{N}{2}(1 - m)\right)!
}.
\end{equation}

From
\begin{equation}\label{eqn:basicStatMechProblemSet3Problem3:120}
\begin{aligned}
2 m &= N_{\uparrow} - N_{\downarrow} \\
N &= N_{\uparrow} + N_{\downarrow},
\end{aligned}
\end{equation}

we see that this can also be written
\boxedEquation{eqn:basicStatMechProblemSet3Problem3:140}{
g(N, m) =
\frac{N!}{
(N_{\uparrow})!
(N_{\downarrow})!
}
}

\paragraph{Simplification for large \(N\)}

%We can think of \(g(N, m)\) as an unnormalized probability density function (with a \(2^N\) normalization factor), so to find a large \(N\) approximation we can apply the central limit theorem.  We need the mean and variance for the \(N = 1\) case
%
%\begin{equation}\label{eqn:basicStatMechProblemSet3Problem3:160}
%\expectation{m} = \inv{2}(-1) + \inv{2}(+1) = 0
%\end{equation}
%\begin{equation}\label{eqn:basicStatMechProblemSet3Problem3:180}
%\expectation{m^2} = \inv{2}(1) + \inv{2}(1) = 1
%\end{equation}
%
%So, by the central limit theorem, a pdf of \(g(N, m)/2^N\) would have a limiting value of approximately
%
%\begin{equation}\label{eqn:basicStatMechProblemSet3Problem3:200}
%g(N, m) = 2^N \inv{\sqrt{2 \pi N}} \exp\left( - \frac{ m^2 }{2 N } \right)
%\end{equation}

%I think this ought to be something that we can use the central limit theorem for, but I'm having trouble getting an answer consistent with the discrete plot.
Using Stirlings approximation
\begin{equation}\label{eqn:basicStatMechProblemSet3Problem3:220}
\begin{aligned}
\ln g(N, m)
&= \ln N! - \ln N_{\downarrow}! - \ln N_{\uparrow}! \\
&\approx
\left( N + \inv{2} \right) \ln N - \cancel{N} + \cancel{ \inv{2} \ln 2 \pi }  \\
&\quad-\left( N_{\uparrow} + \inv{2} \right) \ln N_{\uparrow} + \cancel{N_{\uparrow}} - \cancel{\inv{2} \ln 2 \pi}  \\
&\quad-\left( N_{\downarrow} + \inv{2} \right) \ln N_{\downarrow} + \cancel{N_{\downarrow}} - \inv{2} \ln 2 \pi \\
&=
\left( N_{\uparrow} + N_{\downarrow} + \inv{2} + \mathLabelBox{\inv{2}}{add} \right) \ln N
-\inv{2} \ln 2 \pi
-
\mathLabelBox{
\inv{2} \ln N
}{then subtract}  \\
&\quad -\left( N_{\uparrow} + \inv{2} \right) \ln N_{\uparrow}
\quad -\left( N_{\downarrow} + \inv{2} \right) \ln N_{\downarrow}
\\
&=
-\left( N_{\uparrow} + \inv{2} \right) \ln N_{\uparrow}/N
-\left( N_{\downarrow} + \inv{2} \right) \ln N_{\downarrow}/N
- \inv{2} \ln 2 \pi N \\
&=
-\left( N_{\uparrow} + \inv{2} \right) \ln \inv{2}( 1 + m )
-\left( N_{\downarrow} + \inv{2} \right) \ln \inv{2} ( 1 - m )
- \inv{2} \ln 2 \pi N \\
&\approx
\left( N_{\uparrow} + \inv{2} \right) \ln 2
+\left( N_{\downarrow} + \inv{2} \right) \ln 2
- \inv{2} \ln 2 \pi N \\
&\quad -\left( N_{\uparrow} + \inv{2} \right) \left(m - \inv{2} m^2 \right) \\
&\quad -\left( N_{\downarrow} + \inv{2} \right) \left( -m - \inv{2} m^2 \right) \\
&=
(N + 1) \ln 2 - \inv{2} \ln 2 \pi N
+ m ( N_{\downarrow} - N_{\uparrow} )
+ \inv{2} m^2 ( N + 1 ) \\
&=
(N + 1) \ln 2 - \inv{2} \ln 2 \pi N
- N m^2
+ \inv{2} m^2 ( N + 1 ) \\
&\approx
(N + 1) \ln 2 - \inv{2} \ln 2 \pi N
- \inv{2} m^2 N
\end{aligned}
\end{equation}

This gives us for large \(N\)
\begin{equation}\label{eqn:basicStatMechProblemSet3Problem3:240}
g(N, m) \approx 2^N \sqrt{\frac{2}{ \pi N}} e^{ -m^2 N/2 }
\end{equation}

For \(N = 20\) this approximation and the exact expression are plotted in \cref{fig:statMechProblemSet3Problem3:statMechProblemSet3Problem3Fig1}.

\imageFigure{../figures/phy452-basicstatmech/statMechProblemSet3Problem3Fig1}{Distribution of number of configurations for \(N = 20\) magnets as a function of magnetization}{fig:statMechProblemSet3Problem3:statMechProblemSet3Problem3Fig1}{0.3}

With the large scales of this extremely peaked function, no visible difference can be seen.  That difference does exist, and is plotted in \cref{fig:statMechProblemSet3Problem3:statMechProblemSet3Problem3Fig2}.

\imageFigure{../figures/phy452-basicstatmech/statMechProblemSet3Problem3Fig2}{\(N = 20\) differences between the exact binomial expression and the Gaussian approximation}{fig:statMechProblemSet3Problem3:statMechProblemSet3Problem3Fig2}{0.3}
}
