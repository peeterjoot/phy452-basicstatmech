%
% Copyright © 2016 Peeter Joot.  All Rights Reserved.
% Licenced as described in the file LICENSE under the root directory of this GIT repository.
%
\makeoproblem{Central limit theorem applied to a product?}{pr:midtermReview:1}{2013 midterm pr 1}{
What can we say about a product of independent identical random variables using the Central limit theorem?
} % makeoproblem
\makeanswer{pr:midtermReview:1}{ 
Writing

\begin{equation}\label{eqn:midterm1review:20}
X = x_1 x_2 \cdots x_N.
\end{equation}

We can't apply CLT directly, but we can apply it to \(\ln X\)
\begin{equation}\label{eqn:midterm1review:40}
Y = \ln X = \sum_i \ln x_i = \sum_i y_i,
\end{equation}
so
\begin{equation}\label{eqn:midterm1review:60}
\tilde{P}(Y) = \inv{\sqrt{ 2 \pi N \sigma_y^2}} \exp
\lr{
-\frac{(Y - N \mu_y)^2}{N \sigma_y^2}
}.
\end{equation}

Of interest here is how probabilities change under change of variables.  If \(Y = \ln X\), this works like
\begin{equation}\label{eqn:midterm1review:80}
P(X) dX = \tilde{P}(Y) dY
\end{equation}
\begin{equation}\label{eqn:midterm1review:100}
P(X) = \tilde{P}(Y = \ln X) \frac{dY}{dX}.
\end{equation}

This is illustrated roughly in \cref{fig:lecture8:lecture8Fig1}

\imageFigure{../figures/phy452-basicstatmech/lecture8Fig1}{Change of variables for a probability distribution.}{fig:lecture8:lecture8Fig1}{0.2}
} % makeanswer
