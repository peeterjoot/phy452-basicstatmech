%
% Copyright � 2013 Peeter Joot.  All Rights Reserved.
% Licenced as described in the file LICENSE under the root directory of this GIT repository.
%
\section{Phase space}

Now let's switch directions a bit and look at how to examine a more general system described by the phase space of generalized coordinates

\begin{equation}\label{eqn:basicStatMechLecture4:440}
\{ 
x_{i_\alpha}(t),
p_{i_\alpha}(t)
 \}
\end{equation}

Here
\begin{itemize}
\item \(i = \mbox{molecule or particle number}\)
\item \(\alpha \in \{x, y, z\}\)
\item \(\mbox{Dimension} = N_{\text{particles}} \times 2 \times d\), where \(d\) is the physical space dimension.
\end{itemize}

The motion in phase space will be governed by the knowledge of how each of these coordinates change for all the particles.  Assuming a Hamiltonian \(H\) and recalling that \(H = \xdot p - \LL\), gives us

\begin{subequations}
\begin{equation}\label{eqn:basicStatMechLecture4:600}
\PD{p}{H} = \xdot
\end{equation}
\begin{equation}\label{eqn:basicStatMechLecture4:620}
\PD{x}{H} = -\PD{x}{\LL} = - \ddt{p},
\end{equation}
\end{subequations}
we have for the following set of equations describing the entire system
\begin{subequations}
\begin{equation}\label{eqn:basicStatMechLecture4:460}
\ddt{} x_{i_\alpha}(t) 
= \PD{p_{i_\alpha}}{H}
\end{equation}
\begin{equation}\label{eqn:basicStatMechLecture4:480}
\ddt{} p_{i_\alpha}(t)
= -\PD{x_{i_\alpha}}{H}.
\end{equation}
\end{subequations}

Example, 1D SHO
\begin{equation}\label{eqn:basicStatMechLecture4:500}
H = \inv{2} \frac{p^2}{m} + \inv{2} k x^2.
\end{equation}

This has phase space trajectories as in \cref{fig:basicStatMechLecture4:basicStatMechLecture4Fig4}.  An exploratory purely phase space review of this 1D harmonic oscillator system can be found in \cref{chap:hamiltonianSHOphaseSpace}.

\imageFigure{../figures/phy452-basicstatmech/basicStatMechLecture4Fig4}{Classical SHO phase space trajectories.}{fig:basicStatMechLecture4:basicStatMechLecture4Fig4}{0.3}

%\EndNoBibArticle
