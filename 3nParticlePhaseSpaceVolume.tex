%
% Copyright � 2013 Peeter Joot.  All Rights Reserved.
% Licenced as described in the file LICENSE under the root directory of this GIT repository.
%
%\input{../blogpost.tex}
%\renewcommand{\basename}{3nParticlePhaseSpaceVolume}
%\renewcommand{\dirname}{notes/phy452/}
%\newcommand{\keywords}{}
%
%\input{../peeter_prologue_print2.tex}
%
%\beginArtNoToc
%
%\generatetitle{\(3N\) SHO particle phase space volume}
%\chapter{\(3n\) SHO particle phase space volume}
%\label{chap:3nParticlePhaseSpaceVolume}
%\section{Motivation}
%\section{Guts}

\makeproblem{\(3N\) SHO particle phase space volume}{pr:3nParticlePhaseSpaceVolume:1}{
Calculate the phase space volume calculate for \(3N\) classical SHO particles.
} % makeproblem

\makeanswer{pr:3nParticlePhaseSpaceVolume:1}{
Our Hamiltonian is
\begin{dmath}\label{eqn:3nParticlePhaseSpaceVolume:20}
H
= \sum_{i = 1}^N \frac{\Bp_i^2}{2 m} + \inv{2} m \omega^2 \Br_i^2
= \sum_{i \in [1, N], \alpha \in \{1, 2, 3\}}
\frac{p_{i_\alpha}^2}{2 m} + \inv{2} m \omega^2 x_{i_\alpha}^2.
\end{dmath}

Let's calculate the phase space volume for the shell in \([E - \Delta/2, E + \Delta/2]\) (as in the midterm for the 1D SHO).  That is
\begin{dmath}\label{eqn:3nParticlePhaseSpaceVolume:40}
\int_{
E - \frac{\Delta}{2} \le H \le E + \frac{\Delta}{2} \le H
}
 d^{3N} x d^{3N} p
=
\int_{
E - \frac{\Delta}{2} \le
\sum_{i_\alpha}
\lr{\frac{p_{i_\alpha}^2}{2 m} + \inv{2} m \omega^2 x_{i_\alpha}^2}
 \le E + \frac{\Delta}{2} \le H
}
 d^{3N} x d^{3N} p
=
\int_{
2 m \lr{E - \frac{\Delta}{2} }
\le
\sum_{i_\alpha}
\lr{ p_{i_\alpha}^2 + m^2 \omega^2 x_{i_\alpha}^2 }
 \le
2 m \lr{E + \frac{\Delta}{2}}
}
 d^{3N} x d^{3N} p.
\end{dmath}
Now change variables \(\tilde{x}_{i_\alpha} = m \omega x_{i_\alpha}\).
This gives us
\begin{dmath}\label{eqn:3nParticlePhaseSpaceVolume:60}
\begin{aligned}
&\int_{
E - \frac{\Delta}{2} \le H \le E + \frac{\Delta}{2} \le H
}
 d^{3N} x d^{3N} p \\
&=
\inv{(m \omega)^{3N}}
\int_{
2 m \lr{E - \frac{\Delta}{2} }
\le
\sum_{i_\alpha}
\lr{ p_{i_\alpha}^2 + \tilde{x}_{i_\alpha}^2 }
 \le
2 m \lr{E + \frac{\Delta}{2}}
}
 d^{3N} \tilde{x} d^{3N} p.
\end{aligned}
\end{dmath}
This integral is now the volume of a \(6N\) dimensional spherical shell, giving
\begin{dmath}\label{eqn:3nParticlePhaseSpaceVolume:80}
\int_{
E - \frac{\Delta}{2} \le H \le E + \frac{\Delta}{2} \le H
}
 d^{3N} x d^{3N} p
=
\inv{(m \omega)^{3N}}
\frac{\pi^{6N/2}}{(6N/2)!}
\evalrange{r^{6N}}{\sqrt{2 m \lr{E - \frac{\Delta}{2}}}}{\sqrt{2 m \lr{E + \frac{\Delta}{2}}}}.
\end{dmath}

This is
\boxedEquation{eqn:3nParticlePhaseSpaceVolume:100}{
\int_{
E - \frac{\Delta}{2} \le H \le E + \frac{\Delta}{2} \le H
}
 d^{3N} x d^{3N} p
=
\lr{\frac{2 \pi \Delta}{\omega}}^{3N}
}

} % makeanswer

%\EndNoBibArticle
