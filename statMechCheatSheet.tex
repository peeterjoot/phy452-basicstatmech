%
% Copyright � 2013 Peeter Joot.  All Rights Reserved.
% Licenced as described in the file LICENSE under the root directory of this GIT repository.
%
%\input{../blogpost.tex}
%\renewcommand{\basename}{statMechCheatSheet}
%\renewcommand{\dirname}{notes/phy452/}
%\newcommand{\keywords}{Central limit theorem, Binomial distribution, Generating function, Handy mathematics, Radius of gyration of a 3D polymer, Velocity random walk, Random walk, Maxwell distribution, Hamilton's equations, Liouville's theorem, Ergodic, Thermodynamics, Microstates, Ideal gas, Quantum free particle in a box, Spin, Canonical ensemble, Grand Canonical ensemble, Fermions, Bosons, Density of states, cheat sheet, Statistical mechanics, PHY452H1S}
%
%\input{../peeter_prologue_print2.tex}
%
%\beginArtNoToc
%
%\generatetitle{Summary of statistical mechanics relations and helpful formulas}
\label{chap:statMechCheatSheet}

\paragraph{Central limit theorem}

If \(\expectation{x} = \mu\) and \(\sigma^2 = \expectation{x^2} - \expectation{x}^2\), and \(X = \sum x\), then in the limit
\begin{subequations}
\begin{equation}\label{eqn:statMechCheatSheet:20}
\lim_{N \rightarrow \infty} P(X)
= \inv{\sigma \sqrt{2 \pi N}} \exp\left( - \frac{ (x - N \mu)^2}{2 N \sigma^2} \right)
\end{equation}
\begin{equation}\label{eqn:statMechCheatSheet:40}
\expectation{X} = N \mu
\end{equation}
\begin{equation}\label{eqn:statMechCheatSheet:60}
\expectation{X^2} - \expectation{X}^2 = N \sigma^2.
\end{equation}
\end{subequations}

\paragraph{Binomial distribution}
\begin{equation}\label{eqn:statMechCheatSheet:80}
P_N(X)
=
\left\{
\begin{array}{l l}
\left(\inv{2}\right)^N
\frac{N!}{
\left(\frac{N-X}{2}\right)!
\left(\frac{N+X}{2}\right)!
}
& \quad \mbox{if \(X\) and \(N\) have same parity} \\
0& \quad \mbox{otherwise}
\end{array},
\right.
\end{equation}

where \(X\) was something like number of Heads minus number of Tails.

\paragraph{Generating function}

Given the Fourier transform of a probability distribution \(\tilde{P}(k)\) we have
\begin{dmath}\label{eqn:statMechCheatSheet:100}
\evalbar{
\frac{\partial^n}{\partial k^n}
   \tilde{P}(k)
}{k = 0}
= (-i)^n \expectation{x^n}
\end{dmath}
\paragraph{Handy mathematics}
\begin{dmath}\label{eqn:statMechCheatSheet:120}
\ln( 1 + x ) = x - \frac{x^2}{2} + \frac{x^3}{3} - \frac{x^4}{4}
\end{dmath}
\begin{equation}\label{eqn:statMechCheatSheet:140}
N! \approx \sqrt{ 2 \pi N} N^N e^{-N}
\end{equation}
\begin{equation}\label{eqn:statMechCheatSheet:160}
\ln N! \approx \inv{2} \ln 2 \pi -N +
\lr{ N + \inv{2} }
\ln N \approx N \ln N - N
\end{equation}
\begin{equation}\label{eqn:statMechCheatSheet:180}
\erf(z) = \frac{2}{\sqrt{\pi}} \int_0^z e^{-t^2} dt
\end{equation}
\begin{equation}\label{eqn:statMechCheatSheet:200}
\Gamma(\alpha) = \int_0^\infty dy e^{-y} y^{\alpha - 1}
\end{equation}
\begin{equation}\label{eqn:statMechCheatSheet:220}
\Gamma(\alpha + 1) = \alpha \Gamma(\alpha)
\end{equation}
\begin{equation}\label{eqn:statMechCheatSheet:240}
\Gamma
\lr{1/2}
 = \sqrt{\pi}
\end{equation}
\begin{dmath}\label{eqn:statMechCheatSheet:260}
\zeta(s) = \sum_{k=1}^{\infty} k^{-s}
\end{dmath}
\begin{equation}\label{eqn:statMechCheatSheet:280}
\begin{aligned}
\zeta(3/2) &\approx 2.61238 \\
\zeta(2) &\approx 1.64493 \\
\zeta(5/2) &\approx 1.34149 \\
\zeta(3) &\approx 1.20206
\end{aligned}
\end{equation}
\begin{dmath}\label{eqn:statMechCheatSheet:300}
\Gamma(z) \Gamma(1-z) = \frac{\pi}{\sin(\pi z)}
\end{dmath}

\begin{subequations}
\begin{equation}\label{eqn:statMechCheatSheet:320}
P(x, t) = \int_{-\infty}^\infty \frac{dk}{2 \pi} \tilde{P}(k, t) \exp\left( i k x \right)
\end{equation}
\begin{equation}\label{eqn:statMechCheatSheet:340}
\tilde{P}(k, t) = \int_{-\infty}^\infty dx P(x, t) \exp\left( -i k x \right)
\end{equation}
\end{subequations}
Heavyside theta
\begin{subequations}
\begin{equation}\label{eqn:statMechCheatSheet:360}
\Theta(x) =
\left\{
\begin{array}{l l}
1 & \quad x \ge 0 \\
0 & \quad x < 0
\end{array}
\right.
\end{equation}
\begin{equation}\label{eqn:statMechCheatSheet:380}
\frac{d\Theta}{dx} = \delta(x)
\end{equation}
\end{subequations}
\begin{subequations}
\begin{dmath}\label{eqn:statMechCheatSheet:400}
\sum_{m = -l}^l a^m
=
\frac{a^{l + 1/2} - a^{-(l+1/2)}}{a^{1/2} - a^{-1/2}}
\end{dmath}
\begin{equation}\label{eqn:statMechCheatSheet:420}
\sum_{m = -l}^l e^{b m}
=
\frac{
\sinh(b(l + 1/2))
}
{
\sinh(b/2)
}
\end{equation}
\end{subequations}
\begin{subequations}
\begin{dmath}\label{eqn:statMechCheatSheet:440}
\int_{-\infty}^\infty q^{2 N} e^{-a q^2} dq
=
\frac{(2 N - 1)!!}{(2a)^N} \sqrt{\frac{\pi}{a}}
\end{dmath}
\begin{dmath}\label{eqn:statMechCheatSheet:460}
\int_{-\infty}^\infty e^{-a q^2} dq
=
\sqrt{\frac{\pi}{a}}
\end{dmath}
\end{subequations}
\begin{equation}\label{eqn:statMechCheatSheet:480}
\binom{-\Abs{m}}{k} = (-1)^k \frac{\Abs{m}}{\Abs{m} + k} \binom{\Abs{m}+k}{\Abs{m}}
\end{equation}
\begin{dmath}\label{eqn:statMechCheatSheet:500}
\int_0^\infty d\epsilon \frac{\epsilon^3}{e^{\beta \epsilon} - 1} =
\frac{\pi ^4}{15 \beta ^4},
\end{dmath}
volume in mD
\begin{equation}\label{eqn:statMechCheatSheet:520}
V_m
=
\frac{ \pi^{m/2} R^{m} }
{
   \Gamma\left( m/2 + 1 \right)
}
\end{equation}
area of ellipse
\begin{equation}\label{eqn:statMechCheatSheet:540}
A = \pi a b
\end{equation}
\paragraph{Radius of gyration of a 3D polymer}
With radius \(a\), we have
\begin{dmath}\label{eqn:statMechCheatSheet:560}
r_N \approx a \sqrt{N}
\end{dmath}
\paragraph{Velocity random walk}
Find
\begin{equation}\label{eqn:statMechCheatSheet:580}
\calP_{N_{\txtc}}(\Bv) \propto e^{-
\frac{(\Bv - \Bv_0)^2}{2 N_{\txtc}}
}
\end{equation}
\paragraph{Random walk}
1D Random walk
\begin{dmath}\label{eqn:statMechCheatSheet:600}
\calP( x, t )
=
\inv{2} \calP(x + \delta x, t - \delta t)
+
\inv{2} \calP(x - \delta x, t - \delta t)
\end{dmath}
leads to
\begin{equation}\label{eqn:statMechCheatSheet:620}
\PD{t}{\calP}(x, t) =
\inv{2}
\frac{(\delta x)^2}{\delta t}
\PDSq{x}{\calP}(x, t)
= D \PDSq{x}{\calP}(x, t)
=
-\PD{x}{J},
\end{equation}
The diffusion constant relation to the probability current is referred to as Fick's law
\begin{dmath}\label{eqn:statMechCheatSheet:640}
D = -\PD{x}{J}
\end{dmath}
with which we can cast the probability diffusion identity into a continuity equation form
\begin{dmath}\label{eqn:statMechCheatSheet:660}
\PD{t}{\calP} + \PD{x}{J} = 0
\end{dmath}
In 3D (with the Maxwell distribution frictional term), this takes the form
\begin{subequations}
\begin{equation}\label{eqn:statMechCheatSheet:680}
\Bj = -D \spacegrad_\Bv c(\Bv, t) - \eta \Bv c(\Bv, t)
\end{equation}
\begin{equation}\label{eqn:statMechCheatSheet:700}
\PD{t}{} c(\Bv, t) + \spacegrad_\Bv \cdot \Bj(\Bv, t) = 0
\end{equation}
\end{subequations}
\paragraph{Maxwell distribution}
Add a frictional term to the velocity space diffusion current
\begin{equation}\label{eqn:statMechCheatSheet:720}
j_v = -D \PD{v}{c}(v, t) - \eta v c(v).
\end{equation}
For steady state the continuity equation \(0 = \frac{dc}{dt} = -\PD{v}{j_v}\) leads to
\begin{equation}\label{eqn:statMechCheatSheet:740}
c(v) \propto \exp
\left(
- \frac{\eta v^2}{2 D}
\right).
\end{equation}
We also find
\begin{dmath}\label{eqn:statMechCheatSheet:760}
\expectation{v^2} = \frac{D}{\eta},
\end{dmath}
and identify
\begin{equation}\label{eqn:statMechCheatSheet:780}
\inv{2} m \expectation{\Bv^2} = \inv{2} m \left( \frac{D}{\eta} \right) = \inv{2} \kB T
\end{equation}
\paragraph{Hamilton's equations}
\begin{subequations}
\begin{equation}\label{eqn:statMechCheatSheet:800}
\PD{p}{H} = \xdot
\end{equation}
\begin{equation}\label{eqn:statMechCheatSheet:820}
\PD{x}{H} = -\pdot
\end{equation}
\end{subequations}
SHO
\begin{subequations}
\begin{equation}\label{eqn:statMechCheatSheet:840}
H = \frac{p^2}{2m} + \inv{2} k x^2
\end{equation}
\begin{equation}\label{eqn:statMechCheatSheet:860}
\omega^2 = \frac{k}{m}
\end{equation}
\end{subequations}
Quantum energy eigenvalues
\begin{equation}\label{eqn:statMechCheatSheet:880}
E_n =
\lr{ n + \inv{2} }
\Hbar \omega
\end{equation}
\paragraph{Liouville's theorem}
\begin{dmath}\label{eqn:statMechCheatSheet:900}
\ddt{\rho}
%= \PD{t}{\rho} + \PD{t}{x} \PD{x}{\rho} + \PD{t}{p} \PD{p}{\rho}
= \PD{t}{\rho} + \xdot \PD{x}{\rho} + \pdot \PD{p}{\rho}
= \cdots
= \PD{t}{\rho} + \PD{x}{
\lr{\xdot \rho}
} + \PD{p}{
\lr{\xdot \rho}
}
= \PD{t}{\rho} + \spacegrad_{x,p} \cdot (\rho \xdot, \rho \pdot)
= \PD{t}{\rho} + \spacegrad \cdot \BJ
= 0,
\end{dmath}
Regardless of whether we have a steady state system, if we sit on a region of phase space volume, the probability density in that neighborhood will be constant.
\paragraph{Ergodic}
A system for which all accessible phase space is swept out by the trajectories.  This and Liouville's theorem allows us to assume that we can treat any given small phase space volume as if it is equally probable to the same time evolved phase space region, and switch to ensemble averaging instead of time averaging.
\paragraph{Thermodynamics}
\begin{subequations}
\begin{dmath}\label{eqn:statMechCheatSheet:920}
dE = T dS - P dV + \mu dN
\end{dmath}
\begin{dmath}\label{eqn:statMechCheatSheet:940}
\inv{T} = \PDc{E}{S}{N,V}
\end{dmath}
\begin{dmath}\label{eqn:statMechCheatSheet:960}
\frac{P}{T} = \PDc{V}{S}{N,E}
\end{dmath}
\begin{dmath}\label{eqn:statMechCheatSheet:980}
-\frac{\mu}{T} = \PDc{N}{S}{V,E}
\end{dmath}
\begin{equation}\label{eqn:statMechCheatSheet:1000}
P
= - \PDc{V}{E}{N,S}
= - \PDc{V}{F}{N,T}
\end{equation}
\begin{dmath}\label{eqn:statMechCheatSheet:1020}
\mu = \PDc{N}{E}{V,S} = \PDc{N}{F}{V,T}
\end{dmath}
\begin{dmath}\label{eqn:statMechCheatSheet:1040}
T = \PDc{S}{E}{N,V}
\end{dmath}
\begin{dmath}\label{eqn:statMechCheatSheet:1060}
F = E - TS
\end{dmath}
\begin{equation}\label{eqn:statMechCheatSheet:1080}
G = F + P V = E - T S + P V = \mu N
\end{equation}
\begin{equation}\label{eqn:statMechCheatSheet:1100}
H = E + P V = G + T S
\end{equation}
\begin{equation}\label{eqn:statMechCheatSheet:1120}
\CV = T \PDc{T}{S}{N,V} = \PDc{T}{E}{N,V} = - T
\lr{ \PDSq{T}{F} }
_{N,V}
\end{equation}
\begin{equation}\label{eqn:statMechCheatSheet:1140}
\CP = T \PDc{T}{S}{N,P} = \PDc{T}{H}{N,P}
\end{equation}
\end{subequations}
\begin{equation}\label{eqn:statMechCheatSheet:1160}
\mathLabelBox
[
   labelstyle={xshift=-2cm},
   linestyle={out=270,in=90, latex-}
]
{dE}{Change in energy}
=
\mathLabelBox
[
   labelstyle={below of=m\themathLableNode, below of=m\themathLableNode}
]
{\dbar W}{work done on the system}
+
\mathLabelBox
[
   labelstyle={xshift=2cm},
   linestyle={out=270,in=90, latex-}
]
{\dbar Q}{Heat supplied to the system}
\end{equation}
Example (work on gas): \(\dbar W = -P dV\).  Adiabatic: \(\dbar Q = 0\).  Cyclic: \(dE = 0\).
\paragraph{Microstates}
\begin{dmath}\label{eqn:statMechCheatSheet:1180}
\beta = \inv{\kB T}
\end{dmath}
\begin{equation}\label{eqn:statMechCheatSheet:1200}
S = \kB \ln \Omega
\end{equation}
\begin{dmath}\label{eqn:statMechCheatSheet:1220}
\Omega(N, V, E) =
\inv{h^{3N} N!}
\int_V
d\Bx_1
\cdots
d\Bx_N
\int
d\Bp_1
\cdots
d\Bp_N
\delta \left(
E
- \frac{\Bp_1^2}{2 m}
\cdots
- \frac{\Bp_N^2}{2 m}
\right)
=
\frac{V^N}{h^{3N} N!}
\int
d\Bp_1
\cdots
d\Bp_N
\delta \left(
E
- \frac{\Bp_1^2}{2m}
\cdots
- \frac{\Bp_N^2}{2m}
\right)
\end{dmath}
\begin{equation}\label{eqn:statMechCheatSheet:1240}
\Omega = \frac{d\gamma}{dE}
\end{equation}
\begin{equation}\label{eqn:statMechCheatSheet:1260}
\gamma
=
\frac{V^N}{h^{3N} N!}
\int
d\Bp_1
\cdots
d\Bp_N
\Theta \left(
E
- \frac{\Bp_1^2}{2m}
\cdots
- \frac{\Bp_N^2}{2m}
\right)
\end{equation}
quantum
\begin{equation}\label{eqn:statMechCheatSheet:1280}
\gamma = \sum_i \Theta(E - \epsilon_i)
\end{equation}
\paragraph{Ideal gas}
\begin{equation}\label{eqn:statMechCheatSheet:1300}
\Omega = \frac{V^N}{N!} \inv{h^{3N}} \frac{( 2 \pi m E)^{3 N/2 }}{E} \frac{1}{\Gamma( 3N/2 ) }
\end{equation}
\begin{equation}\label{eqn:statMechCheatSheet:1320}
S_{\mathrm{ideal}} =
\kB
\left(
N \ln \frac{V}{N} + \frac{3 N}{2} \ln
\lr{ \frac{4 \pi m E }{3 N h^2} }
 + \frac{5 N}{2}
\right)
\end{equation}
\paragraph{Quantum free particle in a box}
\begin{subequations}
\begin{equation}\label{eqn:statMechCheatSheet:1340}
\Psi_{n_1, n_2, n_3}(x, y, z) =
\lr{\frac{2}{L}}
^{3/2}
\sin
\lr{ \frac{ n_1 \pi x}{L} }
\sin
\lr{ \frac{ n_2 \pi x}{L} }
\sin
\lr{ \frac{ n_3 \pi x}{L} }
\end{equation}
\begin{equation}\label{eqn:statMechCheatSheet:1360}
\epsilon_{n_1, n_2, n_3} = \frac{h^2}{8 m L^2}
\lr{ n_1^2 + n_2^2 + n_3^2 }
\end{equation}
\begin{dmath}\label{eqn:statMechCheatSheet:1380}
\epsilon_k = \frac{\Hbar^2 k^2}{2m},
\end{dmath}
\end{subequations}
\paragraph{Spin}
magnetization
\begin{equation}\label{eqn:statMechCheatSheet:1400}
\mu = \PD{B}{F}
\end{equation}
moment per particle
\begin{equation}\label{eqn:statMechCheatSheet:1420}
m = \mu/N
\end{equation}
spin matrices
\begin{subequations}
\begin{equation}\label{eqn:statMechCheatSheet:1440}
\sigma_x = \PauliX
\end{equation}
\begin{equation}\label{eqn:statMechCheatSheet:1460}
\sigma_y = \PauliY
\end{equation}
\begin{equation}\label{eqn:statMechCheatSheet:1480}
\sigma_z = \PauliZ
\end{equation}
\end{subequations}
\(l \ge 0, -l \le m \le l\)
\begin{subequations}
\begin{equation}\label{eqn:statMechCheatSheet:1500}
\BL^2 \ket{lm} = l(l+1)\Hbar^2 \ket{lm}
\end{equation}
\begin{equation}\label{eqn:statMechCheatSheet:1520}
L_z \ket{l m} = \Hbar m \ket{l m}
\end{equation}
\end{subequations}
spin addition
\begin{dmath}\label{eqn:statMechCheatSheet:1540}
S(S + 1) \Hbar^2
\end{dmath}
\paragraph{Canonical ensemble}
classical
\begin{equation}\label{eqn:statMechCheatSheet:1560}
\Omega(N, E) =
\frac{ V }{ h^3 N} \int d\Bp_1
e^{\frac{S}{\kB}(N, E)}
e^{-\inv{\kB}
\lr{\PD{N}{S}}
_{E, V} }
e^{-
\frac{\Bp_1^2}{2m \kB}
\lr{\PD{E}{S}}
_{N, V}
}
\end{equation}
quantum
\begin{subequations}
\begin{dmath}\label{eqn:statMechCheatSheet:1580}
\Omega(E)
\approx
\sum_{m \in \text{subsystem}}
e^{\inv{\kB} S(E)}
e^{-\beta \calE_m}
\end{dmath}
\begin{equation}\label{eqn:statMechCheatSheet:1600}
Z
= \sum_m e^{-\beta \calE_m}
= \tr
\lr{ e^{-\beta \hat{H}_{\text{subsystem}}} }
\end{equation}
\end{subequations}
\begin{subequations}
\begin{equation}\label{eqn:statMechCheatSheet:1620}
\expectation{E}
=
\frac{
\int
H
e^{- \beta H }
}{
\int
e^{- \beta H }
}
\end{equation}
\begin{equation}\label{eqn:statMechCheatSheet:1640}
\expectation{E^2}
=
\frac{
\int
H^2
e^{- \beta H }
}{
\int
e^{- \beta H }
}
\end{equation}
\begin{equation}\label{eqn:statMechCheatSheet:1660}
Z \equiv
\inv{h^{3N} N!}
\int
e^{- \beta H }
\end{equation}
\begin{equation}\label{eqn:statMechCheatSheet:1680}
\expectation{E} = -\inv{Z} \PD{\beta}{Z} = - \PD{\beta}{\ln Z}
=
\PD{\beta}{(\beta F)}
\end{equation}
\begin{equation}\label{eqn:statMechCheatSheet:1700}
\sigma_{\txtE}^2
= \expectation{E^2} - \expectation{E}^2
=
\PDSq{\beta}{\ln Z}
= \kB T^2 \PD{T}{\expectation{E}}
= \kB T^2 \CV \propto N
\end{equation}
\begin{equation}\label{eqn:statMechCheatSheet:1720}
Z
= e^{-\beta (\expectation{E} - T S) }
= e^{-\beta F}
\end{equation}
\begin{equation}\label{eqn:statMechCheatSheet:1740}
F = \expectation{E} - T S = -\kB T \ln Z
\end{equation}
\end{subequations}
\paragraph{Grand Canonical ensemble}
\begin{equation}\label{eqn:statMechCheatSheet:1760}
S = - \kB \sum_{r,s} P_{r,s} \ln P_{r,s}
\end{equation}
\begin{subequations}
\begin{equation}\label{eqn:statMechCheatSheet:1780}
P_{r, s} = \frac{e^{-\alpha N_r - \beta E_s}}{\ZG}
\end{equation}
\begin{equation}\label{eqn:statMechCheatSheet:1800}
\ZG = \sum_{r,s} e^{-\alpha N_r - \beta E_s} = \sum_{r,s} z^{N_r} e^{-\beta E_s} = \sum_{N_r} z^{N_r} Z_{N_r}
\end{equation}
\begin{equation}\label{eqn:statMechCheatSheet:1820}
z = e^{-\alpha} = e^{\mu \beta}
\end{equation}
\begin{equation}\label{eqn:statMechCheatSheet:1840}
q = \ln \ZG = P V \beta
\end{equation}
\begin{equation}\label{eqn:statMechCheatSheet:1860}
\expectation{H} = -\PDc{\beta}{q}{z,V} = \kB T^2 \PDc{\mu}{q}{z,V} = \sum_\epsilon \frac{\epsilon}{z^{-1} e^{\beta \epsilon} \pm 1}
\end{equation}
\begin{equation}\label{eqn:statMechCheatSheet:1880}
\expectation{N} = z \PDc{z}{q}{V,T} = \sum_\epsilon \inv{z^{-1} e^{\beta\epsilon} \pm 1}
\end{equation}
\begin{equation}\label{eqn:statMechCheatSheet:1900}
F = - \kB T \ln \frac{ \ZG }{z^N}
\end{equation}
\begin{equation}\label{eqn:statMechCheatSheet:1920}
\expectation{n_\epsilon} = -\inv{\beta} \PDc{\epsilon}{q}{z, T, \text{other} \epsilon} = \inv{z^{-1} e^{\beta \epsilon} \pm 1}
\end{equation}
\end{subequations}
\begin{dmath}\label{eqn:statMechCheatSheet:1940}
\var(N) = \inv{\beta} \PDc{\mu}{\expectation{N}}{V, T} = - \inv{\beta} \PDc{\epsilon}{\expectation{n_\epsilon}}{z,T} = z^{-1} e^{\beta \epsilon}
\end{dmath}
\begin{subequations}
\begin{dmath}\label{eqn:statMechCheatSheet:1960}
\calP \propto
e^{\frac{\mu}{\kB T} N_S}
e^{-\frac{E_S}{\kB T} }
\end{dmath}
\begin{equation}\label{eqn:statMechCheatSheet:1980}
\ZG
= \sum_{N=0}^\infty e^{\beta \mu N}
\sum_{n_k, \sum n_m = N} e^{-\beta \sum_m n_m \epsilon_m}
=
\prod_{k}
\lr{ \sum_{n_k} e^{-\beta(\epsilon_k - \mu) n_k}}
\end{equation}
\begin{dmath}\label{eqn:statMechCheatSheet:2000}
\ZG^{\mathrm{QM}} = {\tr}_{\{\text{energy}, N\}}
\lr{ e^{ -\beta (\hat{H} - \mu \hat{N} ) } }
\end{dmath}
\end{subequations}
\begin{subequations}
\begin{equation}\label{eqn:statMechCheatSheet:2020}
P V = \frac{2}{3} U
\end{equation}
\begin{dmath}\label{eqn:statMechCheatSheet:2040}
f_\nu^\pm(z) = \inv{\Gamma(\nu)} \int_0^\infty dx \frac{x^{\nu - 1}}{z^{-1} e^x \pm 1}
\end{dmath}
\begin{dmath}\label{eqn:statMechCheatSheet:2060}
f_\nu^\pm(z \approx 0) =
z
\mp
\frac{z^{2}}{2^\nu}
+
\frac{z^{3}}{3^\nu}
\mp
\frac{z^{4}}{4^\nu}
+
\cdots
\end{dmath}
\end{subequations}
\begin{equation}\label{eqn:statMechCheatSheet:2080}
z \frac{d f_\nu^{\pm}(z) }{dz} =
f_{\nu-1}^{\pm}(z)
\end{equation}
\begin{equation}\label{eqn:statMechCheatSheet:2100}
\frac{d f_{3/2}^{\pm}(z) }{dT} = -\frac{3}{2T} f_{3/2}^{\pm}(z)
f_{\nu-1}^{\pm}(z)
\end{equation}
\paragraph{Fermions}
\begin{dmath}\label{eqn:statMechCheatSheet:2120}
\sum_{n_k = 0}^1 e^{-\beta(\epsilon_k - \mu) n_k}
=
1 + e^{-\beta(\epsilon_k - \mu)}
\end{dmath}
\begin{equation}\label{eqn:statMechCheatSheet:2140}
N = (2 S + 1) V \int_0^{\kF} \frac{4 \pi k^2 dk}{(2 \pi)^3}
\end{equation}
\begin{subequations}
\begin{dmath}\label{eqn:statMechCheatSheet:2160}
k_{\txtF} =
\lr{\frac{ 6 \pi^2 \rho }{2 S + 1}}
^{1/3}
\end{dmath}
\begin{dmath}\label{eqn:statMechCheatSheet:2180}
\epsilon_{\txtF} = \frac{\Hbar^2}{2m}
\lr{ \frac{6 \pi \rho}{2 S + 1}}
^{2/3}
\end{dmath}
\end{subequations}
%\begin{dmath}\label{eqn:statMechCheatSheet:2200}
%\mu(T = 0) = \epsilon_{\txtF}
%\end{dmath}
\begin{dmath}\label{eqn:statMechCheatSheet:2220}
\mu = \epsilon_{\txtF} - \frac{\pi^2}{12} \frac{(\kB T)^2}{\epsilon_{\txtF}} + \cdots
\end{dmath}
\begin{dmath}\label{eqn:statMechCheatSheet:2240}
\lambda \equiv \frac{h}{\sqrt{2 \pi m \kB T}}
\end{dmath}
\begin{equation}\label{eqn:statMechCheatSheet:2260}
\frac{N}{V}
=
\frac{g}{\lambda^3} f_{3/2}(z)
=
\frac{g}{\lambda^3}
\lr{ e^{\beta \mu} - \frac{e^{2 \beta \mu}}{2^{3/2}} + \cdots }
\end{equation}
(so \(n = \frac{g}{\lambda^3} e^{\beta \mu}\) for large temperatures)
\begin{subequations}
\begin{equation}\label{eqn:statMechCheatSheet:2280}
P \beta = \frac{g}{\lambda^3} f_{5/2}(z)
\end{equation}
\begin{dmath}\label{eqn:statMechCheatSheet:2300}
U
= \frac{3}{2} N \kB T
\frac{f_{5/2}(z)}{
f_{3/2}(z)
}.
\end{dmath}
\end{subequations}
\begin{dmath}\label{eqn:statMechCheatSheet:2320}
f_\nu^+(e^y) \approx
\frac{y^\nu}{\Gamma(\nu + 1)}
\lr{ 1 + 2 \nu \sum_{j = 1, 3, 5, \cdots} (\nu-1) \cdots(\nu - j) \left( 1 - 2^{-j} \right) \frac{\zeta(j+1)}{ y^{j + 1} } }
\end{dmath}
\begin{subequations}
\begin{dmath}\label{eqn:statMechCheatSheet:2340}
\frac{C}{N} =
\frac{\pi^2}{2} \kB \frac{ \kB T}{\epsilon_{\txtF}}
\end{dmath}
\begin{dmath}\label{eqn:statMechCheatSheet:2360}
A = N \kB T
\lr{ \ln z - \frac{f_{5/2}(z)}{f_{3/2}(z)} }
\end{dmath}
\end{subequations}
\paragraph{Bosons}
\begin{equation}\label{eqn:statMechCheatSheet:2380}
\ZG = \prod_\epsilon \inv{ 1 - z e^{-\beta \epsilon} }
\end{equation}
\begin{equation}\label{eqn:statMechCheatSheet:2400}
P \beta = \inv{\lambda^3} g_{5/2}(z)
\end{equation}
\begin{equation}\label{eqn:statMechCheatSheet:2420}
U = \frac{3}{2} \kB T \frac{V}{\lambda^3} g_{5/2}(z)
\end{equation}
\begin{equation}\label{eqn:statMechCheatSheet:2440}
N_e = N - N_0 = N
\lr{ \frac{T}{T_c} }
^{3/2}
\end{equation}
For \(T < T_c\), \(z = 1\).
\begin{equation}\label{eqn:statMechCheatSheet:2460}
g_\nu(1) = \zeta(\nu).
\end{equation}
\begin{dmath}\label{eqn:statMechCheatSheet:2480}
\sum_{n_k = 0}^\infty e^{-\beta(\epsilon_k - \mu) n_k} =
\inv{
1 - e^{-\beta(\epsilon_k - \mu)}
}
\end{dmath}
\begin{dmath}\label{eqn:statMechCheatSheet:2500}
f_\nu^-( e^{-\alpha} ) = \frac{ \Gamma(1 - \nu)}{ \alpha^{1 - \nu} } + \cdots
\end{dmath}
\begin{subequations}
\begin{dmath}\label{eqn:statMechCheatSheet:2520}
\rho \lambda^3 = g_{3/2}(z) \le \zeta(3/2) \approx 2.612
\end{dmath}
\begin{dmath}\label{eqn:statMechCheatSheet:2540}
\kB T_{\txtc} =
\lr{ \frac{\rho}{\zeta(3/2)} }
^{2/3} \frac{ 2 \pi \Hbar^2}{m}
\end{dmath}
\end{subequations}
BEC
\begin{subequations}
\begin{dmath}\label{eqn:statMechCheatSheet:2560}
\rho
= \rho_{\Bk = 0}
+ \inv{\lambda^3}
g_{3/2}(z)
\end{dmath}
\begin{equation}\label{eqn:statMechCheatSheet:2580}
\rho_0 = \rho
\left(
1 -
\lr{ \frac{T}{T_{\txtc}} }
^{3/2}
\right)
\end{equation}
\end{subequations}
\begin{subequations}
\begin{dmath}\label{eqn:statMechCheatSheet:2600}
\frac{E}{V} \propto
\lr{ \kB T}
^{5/2}
\end{dmath}
\begin{dmath}\label{eqn:statMechCheatSheet:2620}
\frac{C}{V} \propto
\lr{\kB T}
^{3/2}
\end{dmath}
\begin{dmath}\label{eqn:statMechCheatSheet:2640}
\frac{S}{N \kB} = \frac{5}{2} \frac{g_{5/2}}{g_{3/2}} - \ln z \Theta(T - T_c)
\end{dmath}
\end{subequations}
\paragraph{Density of states}
Low velocities
\begin{subequations}
\begin{equation}\label{eqn:statMechCheatSheet:2660}
N_1(\epsilon)
=
V
\frac{m \Hbar}{\Hbar^2 \sqrt{ 2 m \epsilon}}
\end{equation}
\begin{equation}\label{eqn:statMechCheatSheet:2680}
N_2(\epsilon)
=
V
\frac{m}{\Hbar^2}
\end{equation}
\begin{equation}\label{eqn:statMechCheatSheet:2700}
N_3(\epsilon)
=
V
\lr{\frac{2 m}{\Hbar^2}}
^{3/2} \inv{4 \pi^2} \sqrt{\epsilon}
\end{equation}
\end{subequations}
relativistic
\begin{subequations}
\begin{dmath}\label{eqn:statMechCheatSheet:2720}
\calD_1(\epsilon)
=
\frac{2 L}{ c h }
\frac{ \sqrt{ \epsilon^2 -
\lr{ m c^2 }^2
} }{\epsilon}
\end{dmath}
\begin{dmath}\label{eqn:statMechCheatSheet:2740}
\calD_2(\epsilon)
=
\frac{2 \pi A}{ (c h)^2 }
\frac{ \epsilon^2 -
\lr{ m c^2 }
^2 }{ \epsilon }
\end{dmath}
\begin{dmath}\label{eqn:statMechCheatSheet:2760}
\calD_3(\epsilon)
=
\frac{4 \pi V}{ (c h)^3 }
\frac{
\left(
	\epsilon^2 - \lr{ m c^2 }^2
\right)^{3/2}
}{\epsilon}
\end{dmath}
\end{subequations}
%\EndNoBibArticle
