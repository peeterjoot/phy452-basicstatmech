%
% Copyright � 2013 Peeter Joot.  All Rights Reserved.
% Licenced as described in the file LICENSE under the root directory of this GIT repository.
%

%
%
%\input{../peeter_prologue_print.tex}
%\input{../peeter_prologue_widescreen.tex}
%
%\label{chap:conditionalProbStatMech}
%\useCCL
%\blogpage{http://sites.google.com/site/peeterjoot2/math2011/conditionalProbStatMech.pdf}
%\date{Dec 27, 2011}
%\revisionInfo{conditionalProbStatMech.tex}
%
%\beginArtWithToc
%\beginArtNoToc
%
%\paragraph{A bit of confusion}
%
In \citep{jackson2000equilibrium} \S 1.5 while discussing statistical uncertainty is a mention of conditional probability.  Once told that a die only rolls numbers up to four, we have a conditional probability for the die of
\begin{equation}\label{eqn:conditionalProbStatMech:10}
P(i | i \le 4) =
\frac{P[i U(i \le 4)]}{P(i \le 4)}
= \frac{\inv{6}}{\frac{4}{6}} = \inv{4}.
\end{equation}

I was having trouble understanding the numerator in this expression.  An initial confusion was, ``what is this U''.  I came to the conclusion that this is just a typo, and was meant to be set intersection, as in
\begin{equation}\label{eqn:conditionalProbStatMech:30}
P(i | i \le 4) =
\frac{P[i \cap (i \le 4)]}{P(i \le 4)}
= \frac{\inv{6}}{\frac{4}{6}} = \inv{4}.
\end{equation}

The denominator makes sense.  I picture a sample space with 6 points as in \cref{fig:conditionalProbStatMechalProbStatMechFig1}
\imageFigure{../figures/phy452-basicstatmech/conditionalProbStatMechFig1}{Sample space for a die.}{fig:conditionalProbStatMechalProbStatMechFig1}{0.15}

The points 1,2,3,4 represent the \(i \le 4\) subspace, as in \cref{fig:conditionalProbStatMechalProbStatMechFig2}
\imageFigure{../figures/phy452-basicstatmech/conditionalProbStatMechFig2}{\(i \le 4\) compound event.}{fig:conditionalProbStatMechalProbStatMechFig2}{0.15}

So we have a \(4/6\) probability for that compound event.

I found it easy to get mixed up considering the numerator.  I was envisioning, as in \cref{fig:conditionalProbStatMech:conditionalProbStatMechFig3}
\imageFigure{../figures/phy452-basicstatmech/conditionalProbStatMechFig3}{Incorrect depiction of \(i \cap (i \le 4)\) compound event.}{fig:conditionalProbStatMech:conditionalProbStatMechFig3}{0.15}

a set of six intersecting with the set of points \(1,2,3,4\), which is just that set of four.  To clear up the confusion, imagine instead that we are asking about the probability of finding the \(i = 2\) face in the dice roll, given the fact that the die only rolls \(i = 1, 2, 3, 4\).  Our intersection sample space for that event is then shown in \cref{fig:conditionalProbStatMech:conditionalProbStatMechFig4}
\imageFigure{../figures/phy452-basicstatmech/conditionalProbStatMechFig4}{Intersection of the \texorpdfstring{\(i = 2\)}{i equal 2} and \texorpdfstring{\(i \le 4\)}{i less than 4} events.}{fig:conditionalProbStatMech:conditionalProbStatMechFig4}{0.1}

The intersection is just the single point, so we have a \(1/6\) probability for that compound event.  The final result makes sense, since if we are looking for the probability for any of the \(i = 1, 2, 3, 4\) events given that the die will only roll one of these values, we should have a value of \(1/4\) as found through the compound probability formula.

%\EndArticle
