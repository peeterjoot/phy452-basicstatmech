%
% Copyright � 2013 Peeter Joot.  All Rights Reserved.
% Licenced as described in the file LICENSE under the root directory of this GIT repository.
%
\makeoproblem{State counting - polymer}{basicStatMech:problemSet3:2}{2013 ps3, p2}{
A typical protein is a long chain molecule made of very many elementary units called amino acids - it is an example of a class of such macromolecules called polymers. Consider a protein made \(N\) amino acids, and assume each amino acid is like a sphere of radius a. As a toy model assume that the protein configuration is like a random walk with each amino acid being one ``step'', i.e., the center-to-center vector from one amino acid to the next is a random vector of length \(2a\) and ignore any issues with overlapping spheres (so-called ``excluded volume'' constraints). Estimate the spatial extent of this protein in space. Typical proteins assume a compact form in order to be functional. In this case, taking the constraint of nonoverlapping spheres, estimate the radius of such a compact protein assuming it has an overall spherical shape with fully packed amino acids (ignore holes in the packing, and use only volume ratios to make this estimate). With \(N = 300\) and \(a \approx 5 \angstrom\), estimate these sizes for the random walk case as well as the compact globular case.
} % makeoproblem

\makeanswer{basicStatMech:problemSet3:2}{

We are considering a geometry like that of \cref{fig:problemSet3problem2touchingCircles:problemSet3problem2touchingCirclesFig3}, depicted in two dimensions for ease of illustration.

\imageFigure{../figures/phy452-basicstatmech/problemSet3problem2touchingCirclesFig3}{Touching ``spheres''.}{fig:problemSet3problem2touchingCircles:problemSet3problem2touchingCirclesFig3}{0.3}

From the geometry, if \(\Bc_k\) is the vector to the center of the \(k\)th sphere, we have for some random unit vector \(\rcap_k\)
\begin{equation}\label{eqn:basicStatMechProblemSet3Problem2:20}
\Bc_{k+1} = \Bc_k + 2 a \rcap_k.
\end{equation}

Proceeding recursively, writing \(d_N\), and \(n = N -1 > 0\), we have for the difference of the positions of the first and last centers of the chain
\begin{dmath}\label{eqn:basicStatMechProblemSet3Problem2:40}
d_N
= \Abs{\Bc_N - \Bc_1}
= 2 a \Abs{ \sum_{k = 1}^{n} \rcap_k}
= 2 a \lr{ \sum_{k,m = 1}^{n} \rcap_k \cdot \rcap_m }^{1/2}
= 2 a \lr{ n + 2 \sum_{1 \le k < m \le n} \rcap_k \cdot \rcap_m }^{1/2}
= 2 a \sqrt{n} \lr{ 1 + \frac{2}{n} \sum_{1 \le k < m \le n} \rcap_k \cdot \rcap_m }^{1/2}.
\end{dmath}

The \(\rcap_k\)'s clearly cannot be completely random since we have a constraint that \(\rcap_k \cdot \rcap_{k+1} > -\cos\pi/3\), or else two adjacent spheres will overlap.  There will also be overlapping constraints for longer portions of the chain that are harder to express.  We are ignoring both such constraints, and seek the ensemble average of all systems of the form \eqnref{eqn:basicStatMechProblemSet3Problem2:40}.

Employing random azimuthal and polar angular variables \(\theta_k\), and \(\phi_k\), we have
\begin{equation}\label{eqn:basicStatMechProblemSet3Problem2:60}
\rcap_k =
\begin{bmatrix}
\sin\theta_k \cos\phi_k \\
\sin\theta_k \sin\phi_k \\
\cos\theta_k,
\end{bmatrix}
\end{equation}
so that the average polymer length is
\begin{equation}\label{eqn:basicStatMechProblemSet3Problem2:80}
\begin{aligned}
\expectation{
d_N
}
&=
2 a \sqrt{n}
\lr{\inv{(2 \pi)(\pi)}}^{n}
\int_{\theta_j \in [0, \pi]}
d\theta_1 d\theta_2 \cdots d\theta_n
\int_{\phi_j \in [0, 2 \pi]}
d\phi_1 d\phi_2 \cdots d\phi_n \times \\
&\quad \lr{ 1 + \frac{2}{n} \sum_{1 \le k < m \le n}
\sin\theta_k \cos\phi_k \sin\theta_m \cos\phi_m
+
\sin\theta_k \sin\phi_k \sin\theta_m \sin\phi_m
+
\cos\theta_k \cos\theta_m
}^{1/2}.
\end{aligned}
\end{equation}

Observing that even \(\int \sqrt{1 + a \cos\theta} d\theta\) is an elliptic integral, we don't have any hope of evaluating this in closed form.  However, to first order, we have
\begin{equation}\label{eqn:basicStatMechProblemSet3Problem2:100}
\begin{aligned}
d_N
&\approx
2 a \sqrt{n}
\lr{\inv{(2 \pi)(\pi)}}^{n}
\int_{\theta_j \in [0, \pi]}
d\theta_1 d\theta_2 \cdots d\theta_n
\int_{\phi_j \in [0, 2 \pi]}
d\phi_1 d\phi_2 \cdots d\phi_n \times \\
&\quad \lr{ 1 + \frac{1}{n} \sum_{1 \le k < m \le n}
\sin\theta_k \cos\phi_k \sin\theta_m \cos\phi_m
+
\sin\theta_k \sin\phi_k \sin\theta_m \sin\phi_m
+
\cos\theta_k \cos\theta_m
}
 \\
&=
2 a \sqrt{n}
\lr{\inv{(2 \pi)(\pi)}}^{n}
\lr{2 \pi^2}^n \\
&\quad +
2 a \sqrt{n}
\lr{\inv{(2 \pi)(\pi)}}^{n}
\lr{2 \pi^2}^{n - 2} \lr{\inv{2} n(n+1)} \inv{n} \times \\
&\quad \int_0^{\pi}
d\theta
\int_0^{\pi}
d\theta'
\int_0^{2 \pi}
d\phi
\int_0^{2 \pi}
d\phi'
\lr{
\sin\theta \cos\phi \sin\theta' \cos\phi'
+
\sin\theta \sin\phi \sin\theta' \sin\phi'
+
\cos\theta \cos\theta'
}
\end{aligned}
\end{equation}

The \(\int_0^{2 \pi} \cos\phi\) integrals kill off the first term, the \(\int_0^{2 \pi} \sin\phi\) integrals kill of the second, and the \(\int_0^\pi \cos\theta\) integral kill of the last term, and we are left with just
\begin{equation}\label{eqn:basicStatMechProblemSet3Problem2:120}
d_N \approx 2 a \sqrt{N-1}.
\end{equation}

Ignoring the extra \(2 \times a\) of the end points, and assuming that for large \(N\) we have \(\sqrt{N-1} \approx \sqrt{N}\), the spatial extent of the polymer chain is
\boxedEquation{eqn:basicStatMechProblemSet3Problem2:140}{
d_N \approx 2 a \sqrt{N}.
}

\paragraph{Grading remark:} You have computed the end to end length.  The radius of gyration is a better measure.

%The total end to end length is roughly the center to center length plus twice the radius for the ends, or
%
%\begin{equation}\label{eqn:basicStatMechProblemSet3Problem2:140}
%2 a (\sqrt{N -1} + 1).
%\end{equation}

\paragraph{Spherical packing}

Assuming the densest possible spherical packing, a face centered cubic \citep{wiki:spherePacking}, as in \cref{fig:problemSet2Problem2faceCenteredCubic:problemSet2Problem2faceCenteredCubicFig4}, we see that the density of such a spherical packing is
% this one frame grabbed from \href{http://y2u.be/Rm-i1c7zr6Q}{http://y2u.be/Rm-i1c7zr6Q}
%\imageFigure{../figures/phy452-basicstatmech/problemSet2Problem2faceCenteredCubicFig4}{Element of a face centered cubic}{fig:problemSet2Problem2faceCenteredCubic:problemSet2Problem2faceCenteredCubicFig4}{0.3}
% This one from mathematica courtosy of Jens
% http://mathematica.stackexchange.com/questions/18789/can-a-latticedata-image-be-displayed-in-a-space-filled-fashion
\imageFigure{../figures/phy452-basicstatmech/problemSet2Problem2faceCenteredCubicFig4b}{Element of a face centered cubic.}{fig:problemSet2Problem2faceCenteredCubic:problemSet2Problem2faceCenteredCubicFig4}{0.3}
\begin{equation}\label{eqn:basicStatMechProblemSet3Problem2:160}
\eta_{\mathrm{FCC}} =
\frac{\lr{8 \times \inv{8} + 6 \times \inv{2} } \frac{4}{3} \pi r^3}{ \lr{ \sqrt{8 r^2} }^3 }
=
\frac{\pi}{\sqrt{18}}.
\end{equation}

With a globular radius of \(R\) and an atomic radius of \(a\), and density \(\eta\) we have
\begin{equation}\label{eqn:basicStatMechProblemSet3Problem2:200}
\eta \frac{4}{3} \pi R^3 = N \frac{4}{3} \pi a^3,
\end{equation}
so that the globular radius \(R\) is
\boxedEquation{eqn:basicStatMechProblemSet3Problem2:180}{
R_N(\eta) = a \sqrt[3]{\frac{N}{\eta}}.
}

\paragraph{Some numbers}

With \(N = 300\) and \(a \approx 5 \angstrom\), and ignoring spaces (i.e. \(\eta = 1\), for a non-physical infinite packing), our globular diameter is approximately
\begin{equation}\label{eqn:basicStatMechProblemSet3Problem2:240}
2 \times 5 \angstrom \sqrt[3]{300} \approx 67 \angstrom.
\end{equation}

This is actually not much different than the maximum spherical packing of an FCC lattice, which results a slightly larger globular cluster diameter
\begin{equation}\label{eqn:basicStatMechProblemSet3Problem2:260}
2 \times 5 \angstrom \sqrt[3]{300 \sqrt{18}/\pi} \approx 74 \angstrom.
\end{equation}

Both however, are much less than the end to end length of the random walk polymer chain
\begin{equation}\label{eqn:basicStatMechProblemSet3Problem2:220}
2 (5 \angstrom) \sqrt{300} \approx 173 \angstrom.
\end{equation}
}
