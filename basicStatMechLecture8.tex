%
% Copyright � 2013 Peeter Joot.  All Rights Reserved.
% Licenced as described in the file LICENSE under the root directory of this GIT repository.
%
%\input{../blogpost.tex}
%\renewcommand{\basename}{basicStatMechLecture8}
%\renewcommand{\dirname}{notes/phy452/}
%\newcommand{\keywords}{Statistical mechanics, PHY452H1S, thermodynamics, adiabatic}
%\input{../peeter_prologue_print2.tex}
%
%\beginArtNoToc
%\generatetitle{PHY452H1S Basic Statistical Mechanics.  Lecture 8: Midterm review, thermodynamics.  Taught by Prof.\ Arun Paramekanti}
\label{chap:basicStatMechLecture8}

%\section{Disclaimer}
%
%Peeter's lecture notes from class.  May not be entirely coherent.

%\section{Gibbs paradox}
%
%We'd been discussing Gibbs paradox, considering the doubled volume as in \cref{fig:lecture8:lecture8Fig5}.
%
%\imageFigure{../figures/phy452-basicstatmech/lecture8Fig5}{Gibbs paradox two volumes allowed to mix}{fig:lecture8:lecture8Fig5}{0.2}
%
%and defined a \underlineAndIndex{statistical entropy} that requires division of \(\Omega\) by \(N!\).  We have
%
%\begin{equation}\label{eqn:basicStatMechLecture8:180}
%S_{\mathrm{statistical}} = \kB \ln
%\mathLabelBox
%[
%   labelstyle={xshift=2cm},
%   linestyle={out=270,in=90, latex-}
%]
%{
%\Omega
%}{Number of states}
%\end{equation}
%
%\section{Phenomenology}

Let's move on to discuss some concepts of thermodynamics

\section{First law of thermodynamics}%\index{First law of thermodynamics}

This is the law of \underlineAndIndex{energy conservation}!  \footnote{This requires some corrections if relativistic systems are studied.}
\begin{equation}\label{eqn:basicStatMechLecture8:200}
\mathLabelBox
[
   labelstyle={xshift=-2cm},
   linestyle={out=270,in=90, latex-}
]
{dE}{Change in energy}
=
\mathLabelBox
[
   labelstyle={below of=m\themathLableNode, below of=m\themathLableNode}
]
{\dbar W}{work done on the system}
+
\mathLabelBox
[
   labelstyle={xshift=2cm},
   linestyle={out=270,in=90, latex-}
]
{\dbar Q}{Heat supplied to the system}.
\end{equation}

Here the \(\dbar\) differentials indicate that the change depends is path dependent.  For example the number of times a piston is moved back and forth can be significant, even given a specific change in the total differential change in the piston position.  The total change of energy in such a change is not path dependent.

Consider a piston setup compressing some gas taking the position of the piston from \(X \rightarrow X + dX\) as in \cref{fig:lecture8:lecture8Fig6}.
\imageFigure{../figures/phy452-basicstatmech/lecture8Fig6}{Piston and gas.}{fig:lecture8:lecture8Fig6}{0.2}
\begin{equation}\label{eqn:basicStatMechLecture8:220}
\dbar W = \sum_i
\mathLabelBox
[
   labelstyle={below of=m\themathLableNode, below of=m\themathLableNode}
]
{f_i}{
Generated force
}
\mathLabelBox
[
   labelstyle={xshift=2cm},
   linestyle={out=270,in=90, latex-}
]
{dX_i}{
Controllable macroscopic coordinate
}.
\end{equation}

What is \(\dbar Q\)?  These are the microscopic, and uncontrollable changes in energy.

\section{Adiabatic and other processes}

\begin{itemize}
\item Change \(X_i \rightarrow X_i + dX_i\)
\item Change \(\dbar Q\)
\end{itemize}

\paragraph{Adiabatic process}\index{adiabatic}

These are those that are thermally isolated, or
\begin{equation}\label{eqn:basicStatMechLecture8:240}
\dbar Q = 0.
\end{equation}

The only way to change energy is to change the macroscopic \(X_i\), so the change in energy is given by
\begin{equation}\label{eqn:basicStatMechLecture8:260}
dE = \dbar W = \sum_i f_i dX_i.
\end{equation}

These are what we are used to thinking about as conservative systems.  Imagine, for example, a force applied to a spring, slowly so that there is no heat loss, we can figure out the total change in energy by summing differential changes to the spring.

We can imagine that there is some way to change the position that results in a change of energy.  This is of course just the force as in \cref{fig:lecture8:lecture8Fig7}.
\begin{equation}\label{eqn:basicStatMechLecture8:280}
-\frac{dE}{dx_i} \sim \text{force}.
\end{equation}

\imageFigure{../figures/phy452-basicstatmech/lecture8Fig7}{A possible energy position relationship.}{fig:lecture8:lecture8Fig7}{0.3}

We can imagine that we started with an initially thermally isolated system, with the macroscopic parameter unchanged, but somebody sneaks in and heats the system, before re-thermally isolating it.  We'd then have a system that perhaps looks like \cref{fig:lecture8:lecture8Fig8}.

\imageFigure{../figures/phy452-basicstatmech/lecture8Fig8}{Level curves for adiabatic and non-adiabatic changes to the system.}{fig:lecture8:lecture8Fig8}{0.3}

The jumps from one curve from another is the product of the sneaky heat suppliers.

Now, if we think about this in a higher dimensional space, we'd instead have level surfaces.

%\EndNoBibArticle
