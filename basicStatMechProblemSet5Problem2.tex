%
% Copyright � 2013 Peeter Joot.  All Rights Reserved.
% Licenced as described in the file LICENSE under the root directory of this GIT repository.
%
\makeoproblem{Independent 1D harmonic oscillators}{basicStatMech:problemSet5:2}{2013 ps5, p2}{
%(4 points)
Consider a set of \(N\) independent classical harmonic oscillators, each having a frequency \(\omega\).
%
\makesubproblem{}{pr:basicStatMechProblemSet5Problem2:a}
Find the canonical partition at a temperature \(T\) for this system of oscillators keeping track of correction factors of Planck constant. (Note that the oscillators are distinguishable, and we do not need \(1/N!\) correction factor.)
%
\makesubproblem{}{pr:basicStatMechProblemSet5Problem2:b}
Using this, derive the mean energy and the specific heat at temperature \(T\).
%
\makesubproblem{}{pr:basicStatMechProblemSet5Problem2:c}
For quantum oscillators, the partition function of each oscillator is simply \(\sum_n e^{-\beta E_n}\) where \(E_n\) are the (discrete) energy levels given by \((n + 1/2)\Hbar \omega\), with \(n = 0,1,2,\cdots\).  Hence, find the canonical partition function for \(N\) independent distinguishable quantum oscillators, and find the mean energy and specific heat at temperature \(T\).
%
\makesubproblem{}{pr:basicStatMechProblemSet5Problem2:d}
Show that the quantum results go over into the classical results at high temperature \(\kB T \gg \Hbar \omega\), and comment on why this makes sense.
%
\makesubproblem{}{pr:basicStatMechProblemSet5Problem2:e}
Also find the low temperature behaviour of the specific heat in both classical and quantum cases when \(\kB T \ll \Hbar \omega\).
} % makeoproblem
%
\makeanswer{basicStatMech:problemSet5:2}{
\makeSubAnswer{Classical partition function}{pr:basicStatMechProblemSet5Problem2:a}
%
For a single particle in one dimension our partition function is
\begin{equation}\label{eqn:basicStatMechProblemSet5Problem2:20}
Z_1 = \inv{h} \int dp dq e^{-\beta
\lr{ \inv{2 m} p^2 + \inv{2} m \omega^2 q^2 }
},
\end{equation}
with
\begin{subequations}
\begin{equation}\label{eqn:basicStatMechProblemSet5Problem2:40}
a = \sqrt{\frac{\beta}{2 m}} p
\end{equation}
\begin{equation}\label{eqn:basicStatMechProblemSet5Problem2:60}
b = \sqrt{\frac{\beta m}{2}} \omega q,
\end{equation}
\end{subequations}
we have
\begin{dmath}\label{eqn:basicStatMechProblemSet5Problem2:80}
Z_1
=
\inv{h \omega}
\sqrt{\frac{2 m}{\beta}}
\sqrt{\frac{2}{\beta m}}
\int da db e^{-a^2 - b^2}
=
\frac{2}{\beta h \omega}
2 \pi \int_0^\infty r e^{-r^2}
=
\frac{2 \pi}{\beta h \omega}
=
\frac{1}{\beta \Hbar \omega}.
\end{dmath}
%
So for \(N\) distinguishable classical one dimensional harmonic oscillators we have
\boxedEquation{eqn:basicStatMechProblemSet5Problem2:100}{
Z_N(T) = Z_1^N =
\lr{\frac{\kB T}{\Hbar \omega}}^N.
}
%
\makeSubAnswer{Classical mean energy and heat capacity}{pr:basicStatMechProblemSet5Problem2:b}
%
From the free energy
\begin{equation}\label{eqn:basicStatMechProblemSet5Problem2:120}
F = -\kB T \ln Z_N = N \kB T \ln (\beta \Hbar \omega),
\end{equation}
we can compute the mean energy
\begin{dmath}\label{eqn:basicStatMechProblemSet5Problem2:140}
U
= \inv{\kB} \PD{\beta}{}
\lr{ \frac{F}{T} }
= N \PD{\beta}{} \ln (\beta \Hbar \omega)
= \frac{N }{\beta},
\end{dmath}
or
\boxedEquation{eqn:basicStatMechProblemSet5Problem2:160}{
U = N \kB T.
}
%
The specific heat follows immediately
\boxedEquation{eqn:basicStatMechProblemSet5Problem2:180}{
\CV = \PD{T}{U} = N \kB.
}
%
\makeSubAnswer{Quantum partition function, mean energy and heat capacity}{pr:basicStatMechProblemSet5Problem2:c}
%
For a single one dimensional quantum oscillator, our partition function is
\begin{dmath}\label{eqn:basicStatMechProblemSet5Problem2:200}
Z_1
=
\sum_{n = 0}^\infty e^{-\beta \Hbar \omega
\lr{n + \inv{2} }
}
=
e^{-\beta \Hbar \omega/2}
\sum_{n = 0}^\infty e^{-\beta \Hbar \omega n}
=
\frac{
e^{-\beta \Hbar \omega/2}
}{1 - e^{-\beta \Hbar \omega}}
=
\frac{1}{e^{\beta \Hbar \omega/2} - e^{-\beta \Hbar \omega/2}}
= \inv{\sinh(\beta \Hbar \omega/2)}.
\end{dmath}
%
Assuming distinguishable quantum oscillators, our \(N\) particle partition function is
\boxedEquation{eqn:basicStatMechProblemSet5Problem2:220}{
Z_N(\beta)
= \inv{\sinh^N(\beta \Hbar \omega/2)}.
}
%
This time we don't add the \(1/\Hbar\) correction factor, nor the \(N!\) indistinguishability correction factor.
%
Our free energy is
\begin{equation}\label{eqn:basicStatMechProblemSet5Problem2:240}
F = N \kB T
\ln \sinh(\beta \Hbar \omega/2)
,
\end{equation}
our mean energy is
\begin{dmath}\label{eqn:basicStatMechProblemSet5Problem2:260}
U
= \inv{\kB} \PD{\beta}{} \frac{F}{T}
= N \PD{\beta}{}
\ln \sinh(\beta \Hbar \omega/2)
= N \frac{\cosh( \beta \Hbar \omega/2 )}{
\sinh(\beta \Hbar \omega/2)
} \frac{\Hbar \omega}{2},
\end{dmath}
or
\boxedEquation{eqn:basicStatMechProblemSet5Problem2:280}{
U(T)
= \frac{N \Hbar \omega}{2} \coth
\lr{ \frac{\Hbar \omega}{2 \kB T}}.
}
%
This is plotted in \cref{fig:basicStatMechProblemSet5Problem2:basicStatMechProblemSet5Problem2Fig2}.
%
\imageFigure{../figures/phy452-basicstatmech/basicStatMechProblemSet5Problem2Fig2A}{Mean energy for \(N\) one dimensional quantum harmonic oscillators.}{fig:basicStatMechProblemSet5Problem2:basicStatMechProblemSet5Problem2Fig2}{0.2}
%
With \(\coth'(x) = -1/\sinh^2(x)\), our specific heat is
\begin{dmath}\label{eqn:basicStatMechProblemSet5Problem2:300}
\CV
= \PD{T}{U}
= \frac{N \Hbar \omega}{2} \frac{-1}{\sinh^2
\lr{ \frac{\Hbar \omega}{2 \kB T}}
} \frac{\Hbar \omega}{2 \kB}
\lr{\frac{-1}{T^2}},
\end{dmath}
%
or
\boxedEquation{eqn:basicStatMechProblemSet5Problem2:320}{
\CV
=
N \kB
\lr{\frac{\Hbar \omega}{2 \kB T
\sinh
\lr{ \frac{\Hbar \omega}{2 \kB T}}
} }^2.
}
%
%This is plotted in \cref{fig:basicStatMechProblemSet5Problem2:basicStatMechProblemSet5Problem2Fig1}.
%
\makeSubAnswer{Classical limits}{pr:basicStatMechProblemSet5Problem2:d}
%
In the high temperature limit \(1 \gg \Hbar \omega/\kB T\), we have
\begin{equation}\label{eqn:basicStatMechProblemSet5Problem2:340}
\cosh
\lr{\frac{\Hbar \omega}{2 \kB T}}
\approx 1
\end{equation}
\begin{equation}\label{eqn:basicStatMechProblemSet5Problem2:360}
\sinh
\lr{\frac{\Hbar \omega}{2 \kB T}}
\approx
\frac{\Hbar \omega}{2 \kB T},
\end{equation}
so
\begin{equation}\label{eqn:basicStatMechProblemSet5Problem2:380}
U \approx N \frac{\cancel{\Hbar \omega}}{\cancel{2}} \frac{\cancel{2} \kB T}{\cancel{\Hbar \omega}},
\end{equation}
or
\begin{equation}\label{eqn:basicStatMechProblemSet5Problem2:400}
U(T) \approx N \kB T,
\end{equation}
matching the classical result of \eqnref{eqn:basicStatMechProblemSet5Problem2:160}.  Similarly from the quantum specific heat result of \eqnref{eqn:basicStatMechProblemSet5Problem2:320}, we have
\begin{equation}\label{eqn:basicStatMechProblemSet5Problem2:420}
\CV(T) \approx
N \kB
\lr{\frac{\Hbar \omega}{2 \kB T
\lr{ \frac{\Hbar \omega}{2 \kB T}}
} }^2
= N \kB.
\end{equation}
%
This matches our classical result from \eqnref{eqn:basicStatMechProblemSet5Problem2:180}.  We expect this equivalence at high temperatures since our quantum harmonic partition function \eqnref{eqn:basicStatMechProblemSet5Problem2:220} is approximately
\begin{equation}\label{eqn:basicStatMechProblemSet5Problem2:440}
Z_N \approx \frac{2}{\beta \Hbar \omega},
\end{equation}
%
This differs from the classical partition function only by this factor of \(2\).  While this alters the free energy by \(\kB T \ln 2\), it doesn't change the mean energy since \(\PDi{\beta}{(\kB \ln 2)} = 0\).  At high temperatures the mean energy are large enough that the quantum nature of the system has no significant effect.
%
\makeSubAnswer{Low temperature limits}{pr:basicStatMechProblemSet5Problem2:e}
%
For the classical case the heat capacity was constant (\(\CV = N \kB\)), all the way down to zero.  For the quantum case the heat capacity drops to zero for low temperatures.  We can see that via L'hopitals rule.  With \(x = \Hbar \omega \beta/2\) the low temperature limit is
\begin{dmath}\label{eqn:basicStatMechProblemSet5Problem2:460}
\lim_{T \rightarrow 0} \CV
= N \kB \lim_{x \rightarrow \infty}
\frac{x^2}{\sinh^2 x}
= N \kB \lim_{x \rightarrow \infty}
\frac{2x }{2 \sinh x \cosh x}
= N \kB \lim_{x \rightarrow \infty}
\frac{1 }{\cosh^2 x + \sinh^2 x}
= N \kB \lim_{x \rightarrow \infty}
\frac{1 }{\cosh (2 x) }
= 0.
\end{dmath}
%
We also see this in the plot of \cref{fig:basicStatMechProblemSet5Problem2:basicStatMechProblemSet5Problem2Fig1}.
%
\imageFigure{../figures/phy452-basicstatmech/basicStatMechProblemSet5Problem2Fig1A}{Specific heat for \(N\) quantum oscillators.}{fig:basicStatMechProblemSet5Problem2:basicStatMechProblemSet5Problem2Fig1}{0.2}
}
