%
% Copyright � 2013 Peeter Joot.  All Rights Reserved.
% Licenced as described in the file LICENSE under the root directory of this GIT repository.
%\input{../blogpost.tex}
%\renewcommand{\basename}{thermodynamicIdentity}
%\renewcommand{\dirname}{notes/phy452/}
%\newcommand{\keywords}{Statistical mechanics, PHY452H1S, thermodynamic identity, one form, two form, differential form, pressure, volume, temperature, entropy, energy, partial derivative}
%
%\input{../peeter_prologue_print2.tex}
%
%\beginArtNoToc
%
%\generatetitle{Thermodynamic identities}
\label{chap:thermodynamicIdentity}

Impressed with the clarity of Baez's entropic force discussion on differential forms \citep{baez:entropicForce}, let's use that methodology to find all the possible identities that we can get from the thermodynamic identity (for now assuming \(N\) is fixed, ignoring the \textAndIndex{chemical potential}.)

This isn't actually that much work to do, since a bit of editor regular expression magic can do most of the work.

Our starting point is the thermodynamic identity
\begin{equation}\label{eqn:thermodynamicIdentity:20}
dU = \dbar Q + \dbar W = T dS - P dV,
\end{equation}
or
\begin{equation}\label{eqn:thermodynamicIdentity:40}
0 = dU - T dS + P dV.
\end{equation}

It's quite likely that many of the identities that can be obtained will be useful, but this should at least provide a handy reference of possible conversions.

\paragraph{Differentials in \(P, V\)}

This first case illustrates the method.
\begin{dmath}\label{eqn:thermodynamicIdentity:60}
\begin{aligned}
0
&= dU - T dS + P dV
&=
\PDc{P}{U}{V} dP +\PDc{V}{U}{P} dV \\
&- T
\lr{
\PDc{P}{S}{V} dP + \PDc{V}{S}{P} dV
}
+ P dV \\
&=
dP \lr{
\PDc{P}{U}{V}
- T \PDc{P}{S}{V}
} \\
&+dV \lr{
\PDc{V}{U}{P}
- T \PDc{V}{S}{P}
+ P
}.
\end{aligned}
\end{dmath}

Taking wedge products \index{wedge product} with \(dV\) and \(dP\) respectively, we form two 2-forms
\begin{subequations}
\begin{equation}\label{eqn:thermodynamicIdentity:80}
0 = dP \wedge dV \lr{
\PDc{P}{U}{V}
- T \PDc{P}{S}{V}
}
\end{equation}
\begin{equation}\label{eqn:thermodynamicIdentity:100}
0 = dV \wedge dP \lr{
\PDc{V}{U}{P}
- T \PDc{V}{S}{P}
+ P
}.
\end{equation}
\end{subequations}

Since these must both be zero we find
\begin{subequations}
\begin{equation}\label{eqn:thermodynamicIdentity:120}
\PDc{P}{U}{V} = T \PDc{P}{S}{V}
\end{equation}
\begin{equation}\label{eqn:thermodynamicIdentity:140}
P =
-\PDc{V}{U}{P}
- T \PDc{V}{S}{P}.
\end{equation}
\end{subequations}

\paragraph{Differentials in \(P, T\)}
\begin{dmath}\label{eqn:thermodynamicIdentity:160}
\begin{aligned}
0
&= dU - T dS + P dV \\
&=
\PDc{P}{U}{T} dP + \PDc{T}{U}{P} dT \\
&-T \lr{
\PDc{P}{S}{T} dP + \PDc{T}{S}{P} dT
} \\
&+
\PDc{P}{V}{T} dP + \PDc{T}{V}{P} dT,
\end{aligned}
\end{dmath}
or
\begin{subequations}
\begin{equation}\label{eqn:thermodynamicIdentity:180}
0 = \PDc{P}{U}{T} -T \PDc{P}{S}{T} + \PDc{P}{V}{T}
\end{equation}
\begin{equation}\label{eqn:thermodynamicIdentity:200}
0 = \PDc{T}{U}{P} -T \PDc{T}{S}{P} + \PDc{T}{V}{P}.
\end{equation}
\end{subequations}

\paragraph{Differentials in \(P, S\)}
\begin{dmath}\label{eqn:thermodynamicIdentity:220}
\begin{aligned}
0
&= dU - T dS + P dV \\
&=
\PDc{P}{U}{S} dP + \PDc{S}{U}{P} dS
- T dS \\
&+ P \lr{
\PDc{P}{V}{S} dP + \PDc{S}{V}{P} dS
},
\end{aligned}
\end{dmath}
or
\begin{subequations}
\begin{equation}\label{eqn:thermodynamicIdentity:240}
\PDc{P}{U}{S} = -P \PDc{P}{V}{S}
\end{equation}
\begin{equation}\label{eqn:thermodynamicIdentity:260}
T = \PDc{S}{U}{P} + P \PDc{S}{V}{P}.
\end{equation}
\end{subequations}

\paragraph{Differentials in \(P, U\)}
\begin{dmath}\label{eqn:thermodynamicIdentity:280}
\begin{aligned}
0
&= dU - T dS + P dV
&= dU - T \lr{
\PDc{P}{S}{U} dP + \PDc{U}{S}{P} dU
} \\
&+ P
\lr{ \PDc{P}{V}{U} dP + \PDc{U}{V}{P} dU },
\end{aligned}
\end{dmath}
or
\begin{subequations}
\begin{equation}\label{eqn:thermodynamicIdentity:300}
0 = 1 - T \PDc{U}{S}{P} + P \PDc{U}{V}{P}
\end{equation}
\begin{equation}\label{eqn:thermodynamicIdentity:320}
T \PDc{P}{S}{U} = P \PDc{P}{V}{U}.
\end{equation}
\end{subequations}

\paragraph{Differentials in \(V, T\)}
\begin{dmath}\label{eqn:thermodynamicIdentity:340}
\begin{aligned}
0
&= dU - T dS + P dV \\
&=
\PDc{V}{U}{T} dV + \PDc{T}{U}{V} dT \\
&- T \lr{ \PDc{V}{S}{T} dV + \PDc{T}{S}{V} dT }
+ P dV,
\end{aligned}
\end{dmath}
or
\begin{subequations}
\begin{equation}\label{eqn:thermodynamicIdentity:360}
0 = \PDc{V}{U}{T} - T \PDc{V}{S}{T} + P
\end{equation}
\begin{equation}\label{eqn:thermodynamicIdentity:380}
\PDc{T}{U}{V} = T \PDc{T}{S}{V}.
\end{equation}
\end{subequations}

\paragraph{Differentials in \(V, S\)}
\begin{dmath}\label{eqn:thermodynamicIdentity:400}
0
= dU - T dS + P dV
=
\PDc{V}{U}{S} dV + \PDc{S}{U}{V} dS
- T dS
+ P dV,
\end{dmath}

or
\begin{subequations}
\begin{equation}\label{eqn:thermodynamicIdentity:420}
P = -\PDc{V}{U}{S}
\end{equation}
\begin{equation}\label{eqn:thermodynamicIdentity:440}
T = \PDc{S}{U}{V} .
\end{equation}
\end{subequations}

\paragraph{Differentials in \(V, U\)}
\begin{dmath}\label{eqn:thermodynamicIdentity:460}
\begin{aligned}
0
&= dU - T dS + P dV \\
&=
dU
- T \lr{ \PDc{V}{S}{U} dV + \PDc{U}{S}{V} dU } \\
&+ P \lr{ \PDc{V}{V}{U} dV + \PDc{U}{V}{V} dU },
\end{aligned}
\end{dmath}
or
\begin{subequations}
\begin{equation}\label{eqn:thermodynamicIdentity:480}
0 = 1 - T \PDc{U}{S}{V} + P \PDc{U}{V}{V}
\end{equation}
\begin{equation}\label{eqn:thermodynamicIdentity:500}
T \PDc{V}{S}{U} = P \PDc{V}{V}{U}.
\end{equation}
\end{subequations}

\paragraph{Differentials in \(S, T\)}
\begin{dmath}\label{eqn:thermodynamicIdentity:520}
\begin{aligned}
0
&= dU - T dS + P dV \\
&= \lr{ \PDc{S}{U}{T} dS + \PDc{T}{U}{S} dT }
- T dS \\
&+ P \lr{ \PDc{S}{V}{T} dS + \PDc{T}{V}{S} dT },
\end{aligned}
\end{dmath}
or
\begin{subequations}
\begin{equation}\label{eqn:thermodynamicIdentity:540}
0 = \PDc{S}{U}{T} - T + P \PDc{S}{V}{T}
\end{equation}
\begin{equation}\label{eqn:thermodynamicIdentity:560}
0 = \PDc{T}{U}{S} + P \PDc{T}{V}{S}.
\end{equation}
\end{subequations}

\paragraph{Differentials in \(S, U\)}
\begin{dmath}\label{eqn:thermodynamicIdentity:580}
0
= dU - T dS + P dV
= dU
- T dS
+ P \lr{ \PDc{S}{V}{U} dS + \PDc{U}{V}{S} dU }
\end{dmath}

or
\begin{subequations}
\begin{equation}\label{eqn:thermodynamicIdentity:600}
\inv{P} = - \PDc{U}{V}{S}
\end{equation}
\begin{equation}\label{eqn:thermodynamicIdentity:620}
T = P \PDc{S}{V}{U}.
\end{equation}
\end{subequations}

\paragraph{Differentials in \(T, U\)}
\begin{dmath}\label{eqn:thermodynamicIdentity:640}
\begin{aligned}
0
&= dU - T dS + P dV \\
&= dU - T
\lr{ \PDc{T}{S}{U} dT + \PDc{U}{S}{T} dU } \\
&+ P
\lr{ \PDc{T}{V}{U} dT + \PDc{U}{V}{T} dU },
\end{aligned}
\end{dmath}
or
\begin{subequations}
\begin{equation}\label{eqn:thermodynamicIdentity:660}
0 = 1 - T \PDc{U}{S}{T} + P \PDc{U}{V}{T}
\end{equation}
\begin{equation}\label{eqn:thermodynamicIdentity:680}
T \PDc{T}{S}{U} = P \PDc{T}{V}{U}.
\end{equation}
\end{subequations}

%\EndArticle
