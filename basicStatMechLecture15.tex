%
% Copyright � 2013 Peeter Joot.  All Rights Reserved.
% Licenced as described in the file LICENSE under the root directory of this GIT repository.
%
%\input{../blogpost.tex}
%\renewcommand{\basename}{basicStatMechLecture15}
%\renewcommand{\dirname}{notes/phy452/}
%\newcommand{\keywords}{Statistical mechanics, PHY452H1S, grand canonical ensemble, grand partition function, Fermion, Boson, particle in a box, occupation number, Hamiltonian, number operator, chemical potential, fugacity, density of states, Boltzmann distribution, thermal de Broglie wavelength}
%\input{../peeter_prologue_print2.tex}
%
%\beginArtNoToc
%\generatetitle{PHY452H1S Basic Statistical Mechanics.  Lecture 15: Grand Canonical/Fermion-Bosons.  Taught by Prof.\ Arun Paramekanti}
%\chapter{Grand Canonical/Fermion-Bosons}
\label{chap:basicStatMechLecture15}

%\section{Disclaimer}
%
%Peeter's lecture notes from class.  May not be entirely coherent.
%

Was mentioned that three dimensions confines us to looking at either Fermions or Bosons, and that two dimensions is a rich subject (interchange of two particles isn't the same as one particle cycling around the other ending up in the same place -- how is that different than a particle cycling around another in a two dimensional space?)

Definitions

\begin{itemize}
\item \textAndIndex{Fermion}.  Antisymmetric under exchange.  \(n_k = 0, 1\).  

Paramekanti: ``Fermions: basically they hate each other''.
\item \textAndIndex{Boson}.  Symmetric under exchange.  \(n_k = 0, 1, 2, \cdots\).  

Unlike Fermions, Bosons ``love each other''.  You can pack as many Bosons as you want into a space.  This allows Bosons to all collectively collapse into the ground state (Bose-Einstein condensation).
\end{itemize}

In either case our energies are
\begin{dmath}\label{eqn:basicStatMechLecture15:20}
\epsilon_k = \frac{\Hbar^2 k^2}{2m}.
\end{dmath}

For Fermions we'll have occupation filling of the form \cref{fig:lecture15:lecture15Fig1a}, where there can be only one particle at any given site (an energy level for that value of momentum).  For Bosonic systems as in \cref{fig:lecture15:lecture15Fig1b}, we don't have a restriction of only one particle for each state, and can have any given number of particles for each value of momentum.

\imageFigure{../figures/phy452-basicstatmech/lecture15Fig1a}{Fermionic energy level filling for free particle in a box.}{fig:lecture15:lecture15Fig1a}{0.2}
\imageFigure{../figures/phy452-basicstatmech/lecture15Fig1b}{Bosonic free particle in a box energy level filling.}{fig:lecture15:lecture15Fig1b}{0.2}

Our Hamiltonian is
\begin{dmath}\label{eqn:basicStatMechLecture15:40}
H = \sum_k \hat{n}_k \epsilon_k,
\end{dmath}

where we have a number operator
\begin{dmath}\label{eqn:basicStatMechLecture15:60}
N = \sum \hat{n}_k,
\end{dmath}
such that
\begin{dmath}\label{eqn:basicStatMechLecture15:80}
\antisymmetric{N}{H} = 0.
\end{dmath}
\begin{dmath}\label{eqn:basicStatMechLecture15:100}
\ZG
= \sum_{N=0}^\infty e^{\beta \mu N}
\sum_{n_k, \sum n_m = N} e^{-\beta \sum_m n_m \epsilon_m}.
\end{dmath}

While the second sum is constrained, because we are summing over all \(n_k\), this is essentially an unconstrained sum, so we can write
\begin{dmath}\label{eqn:basicStatMechLecture15:120}
\ZG
= \sum_{n_k}
e^{\beta \mu \sum_m n_m}
e^{-\beta \sum_n n_n \epsilon_n}
= \sum_{n_k}
e^{-\beta \sum_m (\epsilon_m - \mu) n_m}
=
\sum_{n_k} \lr{ \prod_k e^{-\beta(\epsilon_k - \mu) n_k}}
=
\prod_{k} \lr{ \sum_{n_k} e^{-\beta(\epsilon_k - \mu) n_k}}.
\end{dmath}

%
% Copyright � 2013 Peeter Joot.  All Rights Reserved.
% Licenced as described in the file LICENSE under the root directory of this GIT repository.
%
%\input{../blogpost.tex}
%\renewcommand{\basename}{fdGrandPartition}
%\renewcommand{\dirname}{notes/phy452/}
%\newcommand{\keywords}{Statistical mechanics, PHY452H1S, grand canonical partition function, Fermi-Dirac statistics}
%
%\input{../peeter_prologue_print2.tex}
%
%%\usepackage{ amssymb }
%
%\beginArtNoToc
%
\paragraph{A dumb expansion of the Fermi-Dirac grand partition function}
%\generatetitle{A dumb expansion of the Fermi-Dirac grand partition function}
%%\chapter{A dumb expansion of the Fermi-Dirac grand partition function}
%\label{chap:fdGrandPartition}

Let's make a bit more sense of some of the index manipulation done above.

There are some similarly confusing portions in \S 6.2 \citep{pathriastatistical} where we have the following notation for the sums in the grand partition function \(\ZG\) \index{grand partition function}
% \footnote{On notation: \(\ZG\) is new notation that our final exam introduced, and I'm going to switch to that from the \(\ZGorig\) that we used in class.  The text appears to use a script Q like \(\ZGtext\) but with the loop much more disconnected and hard to interpret}
\begin{subequations}
\begin{equation}\label{eqn:fdGrandPartition:20}
\ZG = \sum_{N = 0}^\infty z^N Q_N(V, T)
\end{equation}
\begin{equation}\label{eqn:fdGrandPartition:40}
Q_N(V, T) = {\sum_{\{n_\epsilon\}}}' e^{-\beta \sum_\epsilon n_\epsilon \epsilon}.
\end{equation}
\end{subequations}

This was shorthand notation for the canonical ensemble, subject to constraints on \(N\) and \(E\)
\begin{subequations}
\begin{equation}\label{eqn:fdGrandPartition:60}
Q_N(V, T) = \sum_E e^{-\beta E}
\end{equation}
\begin{equation}\label{eqn:fdGrandPartition:80}
E = \sum_\epsilon n_\epsilon \epsilon
\end{equation}
\begin{equation}\label{eqn:fdGrandPartition:100}
N = \sum_\epsilon n_\epsilon.
\end{equation}
\end{subequations}

I found this notation pretty confusing, since the normal conventions about what is a dummy index in the various summations do not hold.

The claim of the text (and in class) is that we could write out the grand canonical partition function as
\begin{equation}\label{eqn:fdGrandPartition:120}
\Omega =
\left(
\sum_{n_0}
\lr{ z e^{-\beta \epsilon_0} }
^{n_0}
\right)
\left(
\sum_{n_1}
\lr{ z e^{-\beta \epsilon_1} }
^{n_1}
\right)
\cdots
\end{equation}

Let's verify this for a Fermi-Dirac distribution by dispensing with the notational tricks and writing out the original specification of the grand canonical partition function in long form, and compare that to the first few terms of the expansion of \eqnref{eqn:fdGrandPartition:120}.

Let's consider a specific value of \(E\), namely all those values of \(E\) that apply to \(N = 3\).  Note that we have \(n_\epsilon \in \{0, 1\}\) only for a Fermi-Dirac system, so this means we can have values of \(E\) like
\begin{equation}\label{eqn:fdGrandPartition:140}
E \in \{ \epsilon_0 + \epsilon_1 + \epsilon_2, \epsilon_0 + \epsilon_3 + \epsilon_7, \epsilon_2 + \epsilon_6 + \epsilon_{11}, \cdots\}
\end{equation}

Our grand canonical partition function, when written out explicitly, will have the form
\begin{equation}\label{eqn:fdGrandPartition:160}
\ZG
= z^0 e^{-0}
+ z^1 \sum_{\epsilon_k} e^{-\beta \epsilon_k}
+ z^2 \sum_{\epsilon_k, \epsilon_m} e^{-\beta (\epsilon_k + \epsilon_m) }
+ z^3 \sum_{\epsilon_r, \epsilon_s, \epsilon_t} e^{-\beta (\epsilon_r + \epsilon_s + \epsilon_t) }
+ \cdots
\end{equation}

Okay, that's simple enough and really what the primed notation is getting at.  Now let's verify that after simplification this matches up with \eqnref{eqn:fdGrandPartition:120}.  Expanding this out a bit we have
\begin{dmath}\label{eqn:fdGrandPartition:121}
\Omega
=
\left(
\sum_{n_0 = 0}^1
\lr{ z e^{-\beta \epsilon_0} }
^{n_0}
\right)
\left(
\sum_{n_1 = 0}^1
\lr{ z e^{-\beta \epsilon_1} }
^{n_1}
\right)
\cdots
=
\left(
1 +
z e^{-\beta \epsilon_0}
\right)
\left(
1 +
z e^{-\beta \epsilon_1}
\right)
\left(
1 +
z e^{-\beta \epsilon_2}
\right)
\cdots
=
\left(
1 +
z e^{-\beta \epsilon_0}
+
z e^{-\beta \epsilon_1}
+
z e^{-\beta (\epsilon_0 + \epsilon_1)}
\right)
\left(
1 +
z e^{-\beta \epsilon_2}
+
z e^{-\beta \epsilon_3}
+
z^2 e^{-\beta (\epsilon_2 + \epsilon_3)}
\right)
\left(
1 +
z e^{-\beta \epsilon_4}
\right)
\cdots
=
\left(
1 +
z \left(
e^{-\beta \epsilon_0}
+e^{-\beta \epsilon_1}
+e^{-\beta \epsilon_2}
+e^{-\beta \epsilon_3}
\right)
+
z^2
\left(
e^{-\beta (\epsilon_0 + \epsilon_1)}
+e^{-\beta (\epsilon_0 + \epsilon_2)}
+e^{-\beta (\epsilon_0 + \epsilon_3)}
+e^{-\beta (\epsilon_1 + \epsilon_2)}
+e^{-\beta (\epsilon_1 + \epsilon_3)}
+e^{-\beta (\epsilon_2 + \epsilon_3)}
\right)
+ z^3
\left(
e^{-\beta (\epsilon_0 + \epsilon_1 + \epsilon_2)}
+e^{-\beta (\epsilon_0 + \epsilon_1 + \epsilon_3)}
+e^{-\beta (\epsilon_0 + \epsilon_2 + \epsilon_3)}
+e^{-\beta (\epsilon_1 + \epsilon_2 + \epsilon_3)}
\right)
\right)
\left(
1 +
z e^{-\beta \epsilon_4}
\right)
\cdots
\end{dmath}

This completes the verification of the result as expected.  It is definitely a brute force way of doing so, but easy to understand and I found for myself that it removed some of the notation that obfuscated what is really a simple statement.

Once we are comfortable with this Fermi-Dirac expression of the grand canonical partition function, we can then write it in the product form that leads to the sum that we want after taking logs
\begin{equation}\label{eqn:fdGrandPartition:180}
\ZG =
\left(
1 +
z e^{-\beta \epsilon_0}
\right)
\left(
1 +
z e^{-\beta \epsilon_1}
\right)
\left(
1 +
z e^{-\beta \epsilon_2}
\right)
\cdots
=
\prod_\epsilon
\left(
1 +
z e^{-\beta \epsilon}
\right).
\end{equation}

%\EndArticle


\paragraph{Fermions}
\begin{dmath}\label{eqn:basicStatMechLecture15:140}
\sum_{n_k = 0}^1 e^{-\beta(\epsilon_k - \mu) n_k}
=
1 + e^{-\beta(\epsilon_k - \mu)}.
\end{dmath}

\paragraph{Bosons}
\begin{dmath}\label{eqn:basicStatMechLecture15:160}
\sum_{n_k = 0}^\infty e^{-\beta(\epsilon_k - \mu) n_k} =
\inv{
1 - e^{-\beta(\epsilon_k - \mu)}
}.
\end{dmath}

Observe that we require \(\epsilon_k - \mu \ge 0\).
Our grand partition functions are then
\begin{subequations}
\begin{dmath}\label{eqn:basicStatMechLecture15:180}
\ZG^f = \prod_k
\lr{ 1 + e^{-\beta(\epsilon_k - \mu)} }
\end{dmath}
\begin{dmath}\label{eqn:basicStatMechLecture15:200}
\ZG^b = \prod_k
\inv{ 1 - e^{-\beta(\epsilon_k - \mu)} }.
\end{dmath}
\end{subequations}

We can use these to compute the average number of particles
\begin{dmath}\label{eqn:basicStatMechLecture15:220}
\expectation{n_k^f}
= \frac{1 \times 0 +
e^{-\beta(\epsilon_k - \mu)} \times 1}
{ 1 + e^{-\beta(\epsilon_k - \mu)} }
=
\frac
{ e^{-\beta(\epsilon_k - \mu)} }
{ 1 + e^{-\beta(\epsilon_k - \mu)} },
\end{dmath}
or
\boxedEquation{eqn:basicStatMechLecture15:220b}{
\expectation{n_k^f}
=
\inv
{ e^{\beta(\epsilon_k - \mu)} + 1}
}

For Bosons we have
\begin{dmath}\label{eqn:basicStatMechLecture15:240}
\expectation{n_k^b}
= \frac{
0 \times 1
+
e^{-\beta(\epsilon_k - \mu)} \times 1
+
e^{-2 \beta(\epsilon_k - \mu)} \times 2 + \cdots
}
{
1
+
e^{-\beta(\epsilon_k - \mu)}
+
e^{-2 \beta(\epsilon_k - \mu)}
+ \cdots
}
%=
%\inv{ e^{\beta(\epsilon_k - \mu)} - 1}.
\end{dmath}

This \textAndIndex{chemical potential} over temperature exponential
\begin{dmath}\label{eqn:basicStatMechLecture15:260}
e^{\beta \mu} \equiv z,
\end{dmath}
is called the \underlineAndIndex{fugacity}.  The denominator has the form
\begin{dmath}\label{eqn:basicStatMechLecture15:280}
D =
1
+ z e^{-\beta \epsilon_k}
+ z^2 e^{-2 \beta \epsilon_k} + \cdots,
\end{dmath}
so we see that
\begin{dmath}\label{eqn:basicStatMechLecture15:300}
z \PD{z}{D}
=
  z e^{-\beta \epsilon_k}
+ 2 z^2 e^{-2 \beta \epsilon_k}
+ 3 z^3 e^{-3 \beta \epsilon_k}
+ \cdots
=
  e^{-\beta (\epsilon_k - \mu)}
+ 2 e^{-2 \beta (\epsilon_k - \mu)}
+ 3 e^{-3 \beta (\epsilon_k - \mu)}
+ \cdots,
\end{dmath}
which is exactly the numerator, or
\begin{dmath}\label{eqn:basicStatMechLecture15:320}
N = z \PD{z}{D}.
\end{dmath}

Proceeding
\begin{dmath}\label{eqn:basicStatMechLecture15:340}
\expectation{n_k^b}
= \frac{z \PD{z}{D_k} }{D_k}
= z \PD{z}{} \ln D_k
= -z \PD{z}{} \ln \lr{ 1 - z e^{-\beta \epsilon_k} }
= z
\frac{ e^{-\beta \epsilon_k} }
{ 1 - z e^{-\beta \epsilon_k} }
=
\inv{ z^{-1} e^{\beta \epsilon_k} - 1},
\end{dmath}
or
\boxedEquation{eqn:basicStatMechLecture15:340a}{
\expectation{n_k^b}
=
\inv{ e^{\beta(\epsilon_k - \mu)} - 1}.
}

\paragraph{What is the density \(\rho\)?}

For Fermions
\begin{dmath}\label{eqn:basicStatMechLecture15:360}
\rho = \frac{N}{V} =
\inv{V} \sum_{\Bk}
\inv{ e^{\beta(\epsilon_\Bk - \mu)} + 1}.
\end{dmath}

The simplest such system is the ``free particle in a box'', with Hamiltonian
\begin{dmath}\label{eqn:basicStatMechLecture15:620}
-\frac{\Hbar^2}{2 m} \spacegrad^2 \Psi = \epsilon \Psi.
\end{dmath}

For a box with one corner fixed at the origin, and all sides of length \(L\), we have solutions
\begin{dmath}\label{eqn:basicStatMechLecture15:640}
\Psi =
\sin\lr{ \frac{\pi n_x x}{L} }
\sin\lr{ \frac{\pi n_y y}{L} }
\sin\lr{ \frac{\pi n_z z}{L} },
\end{dmath}

and an energy quantization given by
\begin{dmath}\label{eqn:basicStatMechLecture15:660}
\frac{\Hbar^2 \pi^2}{2 m L^2} \lr{n_x^2 + n_y^2 + n_z^2} = \epsilon.
\end{dmath}

With \(\epsilon_{\Bk} = \Hbar^2 \Bk^2/2m\), we have \(k_\alpha = \pi n_\alpha/L\), and can make an integral approximation of the sum
\begin{dmath}\label{eqn:basicStatMechLecture15:680}
\inv{V} \sum_\Bk f_{\Bk}
\approx
\inv{V} \inv{2^3} \int dn_x dn_y dn_y f_\Bn
=
\inv{V} \inv{2^3} \int dk_x dk_y dk_y \lr{\frac{L}{\pi}}^3 f_\Bk
=
\frac{L^3}{V} \int \frac{d^3 \Bk}{(2 \pi)^3} f_\Bk.
\end{dmath}

Since \(V = L^3\), the leading term cancels out.  Generalizing to other (possibly lower) dimensional spaces, our Fermionic density in a \(d\)-dimensional space is approximated by
\boxedEquation{eqn:basicStatMechLecture15:380}{
\rho =
\int \frac{d^d k}{(2 \pi)^d}
\inv{ e^{\beta(\epsilon_{\Bk} - \mu)} + 1}.
}

This integral is actually difficult to evaluate.  However, for \(T \rightarrow 0\) (\(\beta \rightarrow \infty\)), we can evaluate the limit.
If \(\epsilon_\Bk -\mu(0) < 0\), the exponential term goes as \(e^{-1/0}\) which tends to zero, so we have \(\expectation{n_k^f} = 1\) for \(\epsilon_\Bk < \mu(0)\).  When \(\epsilon_\Bk - \mu(0) > 0\), the exponential term goes as \(e^{1/0}\) which tends to infinity, so \(\expectation{n_k^f} \rightarrow 1\) for \(\epsilon_\Bk > \mu(0)\).  This can be summarized as
\begin{dmath}\label{eqn:basicStatMechLecture15:400}
\evalbar{\expectation{n_k^f}}{T = 0}
= \Theta(\mu(0) - \epsilon_k).
\end{dmath}

This is illustrated in \cref{fig:lecture15:lecture15Fig2}, where the smearing that occurs with increasing temperature is also indicated.  \citep{kittel1980thermal} \S 7 calls \(\EF\) as defined above the \underlineAndIndex{Fermi energy}, the energy at which all the lowest order orbitals are exactly filled.  Without any limiting assumption, a discussion of why this energy is being equated with the \(T = 0\) \textAndIndex{chemical potential} energy can be found in \citep{pathriastatistical} \S 8.1 (where the exact integral is evaluated, or at least expressed).  When that integral approximation of the sum is performed exactly, we see that at \(T = 0\), the illustrated cutoff occurs at this \(\EF = \mu(0)\) point.

\imageFigure{../figures/phy452-basicstatmech/lecture15Fig2}{Occupation numbers for different energies.}{fig:lecture15:lecture15Fig2}{0.2}

With
\begin{equation}\label{eqn:basicStatMechLecture15:420}
E_{\txtF} = \mu(T = 0),
\end{equation}

we want to ask what is the radius of the ball for which
\begin{dmath}\label{eqn:basicStatMechLecture15:560}
\epsilon_k = E_{\txtF},
\end{dmath}
or
\begin{dmath}\label{eqn:basicStatMechLecture15:440}
E_{\txtF} = \frac{\Hbar^2 k_{\txtF}^2}{2m}.
\end{dmath}

With this definition
\begin{dmath}\label{eqn:basicStatMechLecture15:460}
k_{\txtF} = \sqrt{\frac{2 m E_{\txtF}}{\Hbar^2}},
\end{dmath}

so that our density where \(\epsilon_k = \mu\) is
\begin{dmath}\label{eqn:basicStatMechLecture15:480}
\rho
=
\int_{k \le k_{\txtF}} \frac{d^3 k}{(2 \pi)^3} \times 1
=
\inv{(2\pi)^3} 4 \pi \int^{k_{\txtF}} k^2 dk
= \frac{4 \pi}{3} k_{\txtF}^3 \inv{(2 \pi)^3},
\end{dmath}
so that
\begin{dmath}\label{eqn:basicStatMechLecture15:500}
k_{\txtF} = (6 \pi^2 \rho)^{1/3}.
\end{dmath}
Our \textAndIndex{chemical potential} at zero temperature is then
\begin{equation}\label{eqn:basicStatMechLecture15:520}
\mu(T = 0) = \frac{\Hbar^2}{2m} (6 \pi^2 \rho)^{2/3}.
\end{equation}
\begin{dmath}\label{eqn:basicStatMechLecture15:540}
\rho^{-1/3} = \mbox{interparticle spacing}.
\end{dmath}

We can convince ourself that the chemical potential must have the form \cref{fig:lecture15:lecture15Fig3}.
\imageFigure{../figures/phy452-basicstatmech/lecture15Fig3}{Large negative chemical potential at high temperatures.}{fig:lecture15:lecture15Fig3}{0.2}

Given large negative chemical potential at high temperatures our number distribution will have the form
\begin{dmath}\label{eqn:basicStatMechLecture15:580}
\expectation{n_k} = e^{-\beta (\epsilon_k - \mu)} \propto e^{-\beta \epsilon_k}.
\end{dmath}

We see that the classical Boltzmann distribution is recovered for high temperatures.
We can also calculate the chemical potential at high temperatures.  We'll find that this has the form
\begin{dmath}\label{eqn:basicStatMechLecture15:700}
e^{\beta \mu} =
%\frac{4}{3}
\rho \lambda_T^3,
\end{dmath}
%
%FIXME: this isn't the result we found in lectures 16 and 17 (no \(4/3\) factor).

where this quantity \(\lambda_T\) is called the \underlineAndIndex{thermal de Broglie wavelength}.
\begin{dmath}\label{eqn:basicStatMechLecture15:600}
\lambda_T = \sqrt{\frac{ 2 \pi \Hbar^2}{m \kB T}}.
\end{dmath}

%\EndNoBibArticle
