%
% Copyright � 2013 Peeter Joot.  All Rights Reserved.
% Licenced as described in the file LICENSE under the root directory of this GIT repository.
%
%\input{../blogpost.tex}
%\renewcommand{\basename}{basicStatMechLecture10}
%\renewcommand{\dirname}{notes/phy452/}
%\newcommand{\keywords}{Statistical mechanics, PHY452H1S, adiabatic, cyclic, equilibrium, thermal stability, temperature, entropy, irreversible}
%\input{../peeter_prologue_print2.tex}
%
%\beginArtNoToc
%\generatetitle{PHY452H1S Basic Statistical Mechanics.  Lecture 10: Continuing review of thermodynamics.  Taught by Prof.\ Arun Paramekanti}
\label{chap:basicStatMechLecture10}
%
%\section{Disclaimer}
%
%Peeter's lecture notes from class.  May not be entirely coherent.
%
%\section{Continuing review of thermodynamics}

We have energy conservation split into two types of energy
\begin{equation}\label{eqn:basicStatMechLecture10:20}
dE =
\mathLabelBox
[
   labelstyle={below of=m\themathLableNode, below of=m\themathLableNode}
]
{
\dbar W
}{Organized macroscopic variable \(X\)}
+
\mathLabelBox{
\dbar Q
}{disorganized}.
\end{equation}

In \cref{fig:lecture10:lecture10Fig1} we plot changes that are adiabatic processes (\(\dbar Q = 0\)) and heating and cooling processes (with \(\dbar W = 0\)).\index{adiabatic}

\imageFigure{../figures/phy452-basicstatmech/lecture10Fig1}{Adiabatic and heating processes.}{fig:lecture10:lecture10Fig1}{0.2}

Given a dimensionality of \(d_w + 1\), a cyclic change is that for which we have
\begin{subequations}
\begin{equation}\label{eqn:basicStatMechLecture10:40}
\{ X_{\mathrm{initial}} \} \rightarrow \{ X_{\mathrm{final}} \}
\end{equation}
\begin{equation}\label{eqn:basicStatMechLecture10:60}
E_{\mathrm{initial}} \rightarrow E_{\mathrm{final}}
\end{equation}
\begin{equation}\label{eqn:basicStatMechLecture10:80}
\Delta W \ne 0
\end{equation}
\begin{equation}\label{eqn:basicStatMechLecture10:100}
\Delta Q \ne 0
\end{equation}
\begin{equation}\label{eqn:basicStatMechLecture10:120}
\Delta E = 0
\end{equation}
\end{subequations}

Such a cyclic process could be represented as in \cref{fig:lecture10:lecture10Fig2}.

\imageFigure{../figures/phy452-basicstatmech/lecture10Fig2}{Cyclic process.}{fig:lecture10:lecture10Fig2}{0.2}

Here we've labeled the level curves with a parameter \(\sigma\), as yet undefined.  We call \(\sigma\) the \underlineAndIndex{thermodynamic entropy}, and say that
\begin{equation}\label{eqn:basicStatMechLecture10:140}
\lr{\sigma, \{x_i\}},
\end{equation}

specifies the state of the system.

\paragraph{Example:} Pushing a block against a surface with friction.

\section{Equilibrium.}

Considering two systems in contact as in \cref{fig:lecture10:lecture10Fig3}.

\imageFigure{../figures/phy452-basicstatmech/lecture10Fig3}{Two systems in contact.}{fig:lecture10:lecture10Fig3}{0.2}

We require

\begin{enumerate}
\item Mechanical equilibrium.

requires balance of the forces \(f_i\)
\begin{equation}\label{eqn:basicStatMechLecture10:160}
\PD{x_i}{E} = f_i,
\end{equation}
and\footnote{Note the neglect of the sign here, the direction of the force isn't really of interest.}
\begin{equation}\label{eqn:basicStatMechLecture10:180}
\PD{x_i}{E_1} = \PD{x_i}{E_2}.
\end{equation}

\item Thermal stability

\begin{equation}\label{eqn:basicStatMechLecture10:200}
\PD{\sigma}{E_1} = \PD{\sigma}{E_2}.
\end{equation}
We must have some quantity that characterizes the state of the system in a non-macroscopic fashion.  The identity \eqnref{eqn:basicStatMechLecture10:200} is a statement that we have equal temperatures.
\end{enumerate}

We define \underlineAndIndex{temperature} as
\begin{equation}\label{eqn:basicStatMechLecture10:220}
T \equiv \PD{\sigma}{E}.
\end{equation}

We could potentially define different sorts of temperature, for example, perhaps \(T^3 \equiv \PDi{\sigma}{E}\).  Should we do this, we effectively also define \(\sigma\) in a specific way.  The definition \eqnref{eqn:basicStatMechLecture10:220} effectively defines this non-macroscopic parameter \(\sigma\) (the entropy) in the simplest possible way.

\section{Cyclic state variable verses non-state variables.}
\begin{equation}\label{eqn:basicStatMechLecture10:240}
\{x_i\}, \sigma \rightarrow \text{``state variable''}
\end{equation}

A non-cyclic process changes these, whereas a cyclic process takes \(\sigma, \{x_i\}\) back to the initial values.  This is characterized by
\begin{subequations}
\begin{equation}\label{eqn:basicStatMechLecture10:260}
\oint d\sigma = 0
\end{equation}
\begin{equation}\label{eqn:basicStatMechLecture10:280}
\oint dx_i = 0.
\end{equation}
\end{subequations}

This doesn't mean that the closed loop integral of other qualities, such as temperature are necessarily zero
\begin{subequations}
\begin{equation}\label{eqn:basicStatMechLecture10:300}
\oint T d\sigma = \oint \dbar Q \ne 0
\end{equation}
\begin{equation}\label{eqn:basicStatMechLecture10:320}
\oint f_i dx_i = \oint \dbar W \ne 0.
\end{equation}
\end{subequations}

Note that the identification of \(\dbar Q = T d\sigma\) follows from our definition
\begin{equation}\label{eqn:basicStatMechLecture10:340}
\lr{ \PD{\sigma}{E} }_x = T,
\end{equation}
so that with \(\dbar W = 0\) we have
\begin{equation}\label{eqn:basicStatMechLecture10:360}
dE = T \dbar \sigma.
\end{equation}

Graphically we have for a cyclic process \cref{fig:lecture10:lecture10Fig4}.

\imageFigure{../figures/phy452-basicstatmech/lecture10Fig4}{Cyclic process.}{fig:lecture10:lecture10Fig4}{0.2}

We have
\begin{subequations}
\begin{equation}\label{eqn:basicStatMechLecture10:380}
\dbar W_{\rightarrow} = -\dbar W_{\leftarrow}
\end{equation}
\begin{equation}\label{eqn:basicStatMechLecture10:400}
\dbar Q_{\rightarrow} = -\dbar Q_{\leftarrow},
\end{equation}
\end{subequations}
so that
\begin{equation}\label{eqn:basicStatMechLecture10:420}
\Delta Q_{12}^{(\txtA)} + \Delta Q_{21}^{(\txtB)} \ne 0,
\end{equation}

or
\begin{equation}\label{eqn:basicStatMechLecture10:440}
\Delta Q_{12}^{(\txtA)} \ne \Delta Q_{12}^{(\txtB)}.
\end{equation}

\section{Irreversible and reversible processes.}

Reversible means that an undoing of the macroscopic quantities brings us back to the initial state.  A counter example is a block on a spring as illustrated in \cref{fig:lecture10:lecture10Fig5}.

\imageFigure{../figures/phy452-basicstatmech/lecture10Fig5}{Heat loss and irreversibility.}{fig:lecture10:lecture10Fig5}{0.2}

In such a system the block will hit gas atoms as it moves.  It's hard to imagine that such gas particles will somehow spontaneously reorganize itself so that they return to their initial positions and velocities.  This is the jist of the \underlineAndIndex{second law of thermodynamics}.  Real processes introduce a degree of irreversibility with
\begin{subequations}
\begin{equation}\label{eqn:basicStatMechLecture10:460}
\text{Energy}_1 \rightarrow \text{Energy}_2
\end{equation}
\begin{equation}\label{eqn:basicStatMechLecture10:480}
\text{Work} \rightarrow \text{Heat},
\end{equation}
\end{subequations}
but \textunderline{not all}
\begin{equation}\label{eqn:basicStatMechLecture10:500}
\text{Heat} \rightarrow \text{Work}.
\end{equation}

%\EndNoBibArticle
