%
% Copyright � 2013 Peeter Joot.  All Rights Reserved.
% Licenced as described in the file LICENSE under the root directory of this GIT repository.
%
%\input{../blogpost.tex}
%\renewcommand{\basename}{basicStatMechLecture17}
%\renewcommand{\dirname}{notes/phy452/}
%\newcommand{\keywords}{Statistical mechanics, PHY452H1S, Fermi gas, chemical potential, density, Fermi energy, Fermi temperature, occupancy, delta function}
%\input{../peeter_prologue_print2.tex}
%
%\beginArtNoToc
%\generatetitle{PHY452H1S Basic Statistical Mechanics.  Lecture 17: Fermi gas thermodynamics.  Taught by Prof.\ Arun Paramekanti}
\label{chap:basicStatMechLecture17}
%
%\section{Disclaimer}
%
%Peeter's lecture notes from class.  May not be entirely coherent.
%
\paragraph{Review}

\begin{enumerate}
\item Energy was found to be
\begin{equation}\label{eqn:basicStatMechLecture17:20}
\frac{E}{N} = \frac{3}{5} \epsilon_{\txtF}\qquad \mbox{where \(T = 0\)}.
\end{equation}

\item Pressure was found to have the form \cref{fig:lecture17:lecture17Fig1}

\imageFigure{../figures/phy452-basicstatmech/lecture17Fig1}{Pressure in Fermi gas}{fig:lecture17:lecture17Fig1}{0.3}

\item The \textAndIndex{chemical potential} was found to have the form \cref{fig:lecture17:lecture17Fig2}.

\imageFigure{../figures/phy452-basicstatmech/lecture17Fig2}{Chemical potential in Fermi gas}{fig:lecture17:lecture17Fig2}{0.3}

We found that
\begin{subequations}
\begin{dmath}\label{eqn:basicStatMechLecture17:40}
e^{\beta \mu} = \rho \lambda_{\txtT}^3
\end{dmath}
\begin{dmath}\label{eqn:basicStatMechLecture17:60}
\lambda_{\txtT} = \frac{h}{\sqrt{ 2 \pi m \kB T}},
\end{dmath}
\end{subequations}
so that the zero crossing is approximately when
\begin{dmath}\label{eqn:basicStatMechLecture17:420}
e^{\beta \times 0} = 1 = \rho \lambda_{\txtT}^3.
\end{dmath}

That last identification provides the relation \(T \sim T_{\txtF}\). FIXME: that bit wasn't clear to me.
\end{enumerate}

\paragraph{How about at other temperatures?}

\begin{enumerate}
\item \(\mu(T) = ?\)
\item \(E(T) = ?\)
\item \(\CV(T) = ?\)
\end{enumerate}

We had
\begin{dmath}\label{eqn:basicStatMechLecture17:80}
N = \sum_k \inv{e^{\beta (\epsilon_k - \mu)} + 1} = \sum_{\Bk} n_{\txtF}(\epsilon_\Bk)
\end{dmath}

and taking the average of \eqnref{eqn:basicStatMechLecture15:40} we have
\begin{dmath}\label{eqn:basicStatMechLecture17:100}
E(T) =
\sum_k \epsilon_\Bk n_{\txtF}(\epsilon_\Bk).
\end{dmath}

We can define a \textAndIndex{density of states}
\begin{dmath}\label{eqn:basicStatMechLecture17:120}
\sum_\Bk
= \sum_\Bk \int_{-\infty}^\infty d\epsilon \delta(\epsilon - \epsilon_\Bk)
=
\int_{-\infty}^\infty d\epsilon
\sum_\Bk
\delta(\epsilon - \epsilon_\Bk),
\end{dmath}
where the liberty to informally switch the order of differentiation and integration has been used.  This construction allows us to write a more general sum
\begin{dmath}\label{eqn:basicStatMechLecture17:140}
\sum_\Bk f(\epsilon_\Bk)
= \sum_\Bk \int_{-\infty}^\infty d\epsilon \delta(\epsilon - \epsilon_\Bk)
f(\epsilon)
=
\sum_\Bk
\int_{-\infty}^\infty d\epsilon
\delta(\epsilon - \epsilon_\Bk)
f(\epsilon)
=
\int_{-\infty}^\infty d\epsilon
f(\epsilon)
\lr{
\sum_\Bk
\delta(\epsilon - \epsilon_\Bk)
}.
\end{dmath}

This sum, evaluated using a continuum approximation, is
\begin{dmath}\label{eqn:basicStatMechLecture17:160}
N(\epsilon) \equiv
\sum_\Bk
\delta(\epsilon - \epsilon_\Bk)
=
\frac{V}{(2 \pi)^3} \int d^3 \Bk \delta\lr{ \epsilon - \frac{\Hbar^2 k^2}{2 m}}
=
\frac{V}{(2 \pi)^3} 4 \pi \int_0^\infty k^2 dk \delta\lr{ \epsilon - \frac{\Hbar^2 k^2}{2 m}}.
\end{dmath}

Using
\begin{dmath}\label{eqn:basicStatMechLecture17:440}
\delta(g(x)) = \sum_{x_0} \frac{\delta(x - x_0)}{\Abs{g'(x_0)}},
\end{dmath}

where the roots of \(g(x)\) are \(x_0\), we have
\begin{dmath}\label{eqn:basicStatMechLecture17:460}
N(\epsilon) =
\frac{V}{(2 \pi)^3} 4 \pi \int_0^\infty k^2 dk \delta
\lr{ k - \frac{\sqrt{2 m \epsilon}}{\Hbar} }
\frac{1}{ \frac{\Hbar^2}{m} \frac{\sqrt{2 m \epsilon}}{\Hbar}}
%\frac{m \Hbar }{ \Hbar^2 \sqrt{2 m \epsilon}}
=
\frac{V}{(2 \pi)^3} 2 \pi \frac{2 m \epsilon}{\Hbar^2}
\frac{2 m \Hbar }{ \Hbar^2 \sqrt{2 m \epsilon}}
=
V \lr{\frac{2 m}{\Hbar^2}}^{3/2} \inv{4 \pi^2} \sqrt{\epsilon}.
\end{dmath}

%To make that last evaluation a change of vars and \(k \sim \sqrt{\epsilon}\) so that \(\epsilon^{3/2} /\epsilon \sim \sqrt{\epsilon}\).

In \textunderline{2D} this would be
\begin{dmath}\label{eqn:basicStatMechLecture17:180}
N(\epsilon) \sim V \int dk k \delta \lr{ \epsilon - \frac{\Hbar^2 k^2}{2m} }
=
V \frac{\sqrt{2 m \epsilon}}{\Hbar}
\frac{m \Hbar}{\Hbar^2 \sqrt{ 2 m \epsilon}}
\sim
V,
\end{dmath}
and in \textunderline{1D}
\begin{dmath}\label{eqn:basicStatMechLecture17:200}
N(\epsilon) \sim V \int dk \delta \lr{ \epsilon - \frac{\Hbar^2 k^2}{2m} }
=
V
\frac{m \Hbar}{\Hbar^2 \sqrt{ 2 m \epsilon}}
\sim \inv{\sqrt{\epsilon}}.
\end{dmath}

\paragraph{Low temperature density and chemical potential}

\begin{dmath}\label{eqn:basicStatMechLecture17:280}
N = V \int_0^\infty
\mathLabelBox{
n_{\txtF}(\epsilon)
}{\(1/(e^{\beta (\epsilon - \mu)} + 1)\)}
\mathLabelBox
[
   labelstyle={below of=m\themathLableNode, below of=m\themathLableNode}
]
{
N(\epsilon)
}{
\(\epsilon^{1/2}\)
}
\end{dmath}

\begin{dmath}\label{eqn:basicStatMechLecture17:300}
\rho
=
\frac{N}{V}
=
\lr{ \frac{2m}{\Hbar^2 } }
^{3/2} \inv{ 4 \pi^2}
\int_0^\infty d\epsilon \frac{\epsilon^{1/2}}{z^{-1} e^{\beta \epsilon} + 1}
=
\lr{ \frac{2m}{\Hbar^2 } }
^{3/2} \inv{ 4 \pi^2}
\lr{\kB T}
^{3/2}
\int_0^\infty dx \frac{x^{1/2}}{z^{-1} e^{x} + 1}
\end{dmath}

where \(z = e^{\beta \mu}\) as usual, and we write \(x = \beta \epsilon\).   For the low temperature asymptotic behavior see \citep{pathriastatistical} appendix \S E.15, as derived in the homework \eqnref{eqn:basicStatMechProblemSet6Problem2:520}.  For \(z\) large it can be shown that this is
\begin{dmath}\label{eqn:basicStatMechLecture17:320}
\int_0^\infty dx \frac{x^{1/2}}{z^{-1} e^{x} + 1}
\approx
\frac{2}{3}
\lr{\ln z}
^{3/2}
\lr{
1 + \frac{\pi^2}{8} \inv{(\ln z)^2}
},
\end{dmath}
so that
\begin{dmath}\label{eqn:basicStatMechLecture17:340}
\rho \approx
\lr{ \frac{2m}{\Hbar^2 } }
^{3/2} \inv{ 4 \pi^2}
\lr{\kB T}
^{3/2}
\frac{2}{3}
\lr{\ln z}
^{3/2}
\lr{
1 + \frac{\pi^2}{8} \inv{(\ln z)^2}
}
=
\lr{ \frac{2m}{\Hbar^2 } }
^{3/2} \inv{ 4 \pi^2}
\frac{2}{3}
\mu^{3/2}
\lr{
1 + \frac{\pi^2}{8} \inv{(\beta \mu)^2}
}
=
\lr{ \frac{2m}{\Hbar^2 } }
^{3/2} \inv{ 4 \pi^2}
\frac{2}{3}
\mu^{3/2}
\lr{
1 + \frac{\pi^2}{8}
\lr{ \frac{\kB T}{\mu}}^2
}
= \rho_{T = 0}
\lr{ \frac{\mu}{ \epsilon_{\txtF} } }
^{3/2}
\lr{
1 + \frac{\pi^2}{8}
\lr{ \frac{\kB T}{\mu}}^2
}
\end{dmath}

Assuming a quadratic form for the chemical potential at low temperature as in \cref{fig:lecture17:lecture17Fig5}, we have

\imageFigure{../figures/phy452-basicstatmech/lecture17Fig5}{Assumed quadratic form for low temperature chemical potential}{fig:lecture17:lecture17Fig5}{0.3}

\begin{dmath}\label{eqn:basicStatMechLecture17:360}
1 =
\lr{ \frac{\mu}{ \epsilon_{\txtF} } }
^{3/2}
\lr{
1 + \frac{\pi^2}{8}
\lr{ \frac{\kB T}{\mu}}^2
}
=
\lr{ \frac{\epsilon_{\txtF} - a T^2}{ \epsilon_{\txtF} } }
^{3/2}
\lr{
1 + \frac{\pi^2}{8}
\lr{ \frac{\kB T}{\epsilon_{\txtF} - a T^2}}^2
}
\approx
\lr{
1 - \frac{3}{2} a \frac{T^2}{\epsilon_{\txtF}}
}
\lr{
1 + \frac{\pi^2}{8} \frac{(\kB T)^2}{\epsilon_{\txtF}^2}
}
\approx
1 - \frac{3}{2} a \frac{T^2}{\epsilon_{\txtF}} + \frac{\pi^2}{8} \frac{(\kB T)^2}{\epsilon_{\txtF}^2},
\end{dmath}

or
\begin{dmath}\label{eqn:basicStatMechLecture17:380}
a = \frac{\pi^2}{12} \frac{\kB^2}{\epsilon_{\txtF}},
\end{dmath}

We have used a Taylor expansion \((1 + x)^n \approx 1 + n x\) for small \(x\), for an end result of
\begin{dmath}\label{eqn:basicStatMechLecture17:400}
\mu = \epsilon_{\txtF} - \frac{\pi^2}{12} \frac{(\kB T)^2}{\epsilon_{\txtF}}.
\end{dmath}

%\EndArticle
