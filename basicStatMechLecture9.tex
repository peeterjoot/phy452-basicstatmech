%
% Copyright � 2013 Peeter Joot.  All Rights Reserved.
% Licenced as described in the file LICENSE under the root directory of this GIT repository.
%
%\input{../blogpost.tex}
%\renewcommand{\basename}{basicStatMechLecture9}
%\renewcommand{\dirname}{notes/phy452/}
%\newcommand{\keywords}{Statistical mechanics, PHY452H1S, adiabatic, thermodynamic, first law, energy conservation, heat, work, reversible}
%\input{../peeter_prologue_print2.tex}
%
%\beginArtNoToc
%\generatetitle{PHY452H1S Basic Statistical Mechanics.  Lecture 9: Lightning review of thermodynamics.  Taught by Mr.\ (Eric) Kin-Ho Lee}
%\label{chap:basicStatMechLecture9}

%\section{Disclaimer}
%
%Peeter's lecture notes from class.  May not be entirely coherent.
%
%\section{Lightning review of thermodynamics}

\section{First law.}

Energy conservation.

\begin{enumerate}
\item Work.  Macroscopic control
\item heat.  Uncontrollable (microscopically)
\end{enumerate}

This is summarized by the differential relationship
\begin{equation}\label{eqn:basicStatMechLecture9:20}
dE = \dbar W + \dbar Q.
\end{equation}

\section{Examples of work.}

We have many types of work (in contrast to only one type of heat).  Examples

\begin{itemize}
\item \(-P dV = \dbar W\)
\item \(q\BE \cdot dl\)
\item \(k x dx\)
\item \(H dm\)
\end{itemize}

Homework: verify the signs of these.

We put these into a general form, to first order, of
\begin{equation}\label{eqn:basicStatMechLecture9:40}
\dbar W_i = f_i dx_i,
\end{equation}
where we assume that higher order terms are not significant.
\begin{equation}\label{eqn:basicStatMechLecture9:60}
\dbar W = \sum_i \dbar W_i = \sum_i f_i dx_i.
\end{equation}

\section{Heat.}

We have only one type of heat, which we loosely describe as something imbued by contact with a ``hotter'' system, as in \cref{fig:lecture9:lecture9Fig1}.

\imageFigure{../figures/phy452-basicstatmech/lecture9Fig1}{System in contact with heat source.}{fig:lecture9:lecture9Fig1}{0.2}

\section{Adiabatic processes.}\index{adiabatic}

This is defined as the condition where we have no heat exchange with the environment, or
\begin{equation}\label{eqn:basicStatMechLecture9:80}
\dbar Q = 0.
\end{equation}

We contrast this with heating processes for which we have
\begin{equation}\label{eqn:basicStatMechLecture9:100}
\dbar W = 0.
\end{equation}

Since we have \(N\) coordinates (\(\dbar W = \sum_{i = 1}^N f_i dx_i\)).  We can think about an \(n + 1\) dimensional space, where

\begin{itemize}
\item \(N\)-dimensions are \(x_i\)
\item 1 dimension that characterizes heat exchange.
\end{itemize}

\makeexample{\(n = 1\)}{example:basicStatMechLecture9:1}{

Given work on gas
\begin{equation}\label{eqn:basicStatMechLecture9:120}
\dbar W = -P dV.
\end{equation}

We have a coordinate, not yet precisely defined, for which fixed levels indicate that there is no heat exchange occurring, as in \cref{fig:lecture9:lecture9Fig2}.
\imageFigure{../figures/phy452-basicstatmech/lecture9Fig2}{Adiabatic and heat exchange processes.}{fig:lecture9:lecture9Fig2}{0.2}
We'll call this axis \(\sigma\), the \underlineAndIndex{thermodynamic entropy}.
} % makeexample

We've been introduced to statistical entropy
\begin{equation}\label{eqn:basicStatMechLecture9:140}
S = \kB \ln \Omega.
\end{equation}

We'll assume for now that these are not related and will eventually figure out the connection between these two concepts.

\makeexample{\(n = 2\)}{example:basicStatMechLecture9:2}{

A representation of a adiabatic, or constant \(\sigma\)-hypersurface process is given in \cref{fig:lecture9:lecture9Fig3}, a heating/cooling process with transition between \(\sigma\)-hypersurfaces in \cref{fig:lecture9:lecture9Fig4}, and a cyclic process, in \cref{fig:lecture9:lecture9Fig5}.

\imageFigure{../figures/phy452-basicstatmech/lecture9Fig3}{Adiabatic process.}{fig:lecture9:lecture9Fig3}{0.2}

\imageFigure{../figures/phy452-basicstatmech/lecture9Fig4}{Heat exchange process.}{fig:lecture9:lecture9Fig4}{0.2}

\imageFigure{../figures/phy452-basicstatmech/lecture9Fig5}{Cyclic process.}{fig:lecture9:lecture9Fig5}{0.2}

The cyclic process is one for which \(dE = \dbar W + \dbar Q = 0\), however, this does not imply \(\dbar W = 0\) and \(\dbar Q = 0\) since we only require that the sum of the two is zero.  In this whole process, we can have for example a net change in heat.  Example: the engine of a car.  Work is done, and heat is generated, but a car that was initially stopped and returns to its final destination, stops and cools down again, has still had significant internal action in the process.
}

\section{Reversible processes.}

What do we mean by reversible?  We mean that any of the changes in the system have been done so slowly that we could reverse the direction of the processes at any point, and should we do so, both the system and the environment will be returned to its initial state.  This is an idealization that is, most of the time, a good approximation, but gives us an excellent idea of the limits of what we can theoretically describe.

\paragraph{Question}: Why does the speed of the process make a difference?

If we are making changes to the system quickly, imagine that we are compressing a gas as in \cref{fig:lecture9:lecture9Fig6}.

\imageFigure{../figures/phy452-basicstatmech/lecture9Fig6}{Fast gas compression by a piston.}{fig:lecture9:lecture9Fig6}{0.2}

Doing work slowly means that the whole system can react to the change imposed.  If we compressed the gas quickly, then changes to the system start only at the contact point with the piston.  This can't be reversed.  If we pull the piston out at this point, none of the non-front gas particles will be able to react.  The system will not be in \underlineAndIndex{thermal equilibrium} for fast changes.

%\EndNoBibArticle
