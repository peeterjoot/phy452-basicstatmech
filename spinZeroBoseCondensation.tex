%
% Copyright � 2013 Peeter Joot.  All Rights Reserved.
% Licenced as described in the file LICENSE under the root directory of this GIT repository.
%
%\input{../blogpost.tex}
%\renewcommand{\basename}{spinZeroBoseCondensation}
%\renewcommand{\dirname}{notes/phy452/}
%\newcommand{\keywords}{Statistical mechanics, PHY452H1S, Bose condensate, average number density, ground state number density, ultra relativistic gas, zeta function}
%
%\input{../peeter_prologue_print2.tex}
%
%\beginArtNoToc
%
%\generatetitle{Ultra relativistic spin zero condensation temperature}
%\chapter{Ultra relativistic spin zero condensation temperature}
\label{chap:spinZeroBoseCondensation}
%
%Here's a bash at one of the exam questions, where I get the time to think things through properly. I think I did something like this on the exam itself, but may have also made some arithmetic errors.
%
\makeoproblem{Relativistic spin zero BEC temp.}{pr:spinZeroBoseCondensation:2}{2013 final, pr. 2}{
Consider a Bose gas with particles having no spin and obeying an ultra relativistic dispersion \(E_\Bk = c \Abs{\Bk}\).  Unlike photons or phonons, these particles are {\bf conserved}, and hence we must determine the \textAndIndex{chemical potential} \(\mu\) which fixes their density.  Working in three dimensions, show whether or not these particles will exhibit Bose condensation, and find \(T_c\) if it is nonzero.
} % makeoproblem
%
\makeanswer{pr:spinZeroBoseCondensation:2}{
%
For the number of particles in the gas, as with photons, we still have
\begin{dmath}\label{eqn:spinZeroBoseCondensation:20}
\expectation{N}
= \sum_\Bk \inv{z^{-1} e^{\beta \epsilon_\Bk} - 1}
= \inv{z^{-1} - 1}
+ \sum_{\Bk \ne 0} \inv{z^{-1} e^{\beta \epsilon_\Bk} - 1}.
\end{dmath}
%
As in the discussion of low velocity particles in \cite{pathriastatistical} \S 7.1, the ground state term has been split out, before making any continuum approximation of the sum over the energetic states.
Writing
\begin{equation}\label{eqn:spinZeroBoseCondensation:40}
\expectation{N} = N_0 + N_e,
\end{equation}
where the number of particles in the ground state is chemical potential and temperature dependent
\begin{equation}\label{eqn:spinZeroBoseCondensation:60}
N_0 = \frac{z}{1 - z}.
\end{equation}
%
We proceed with the continuum approximation for the number of particles in the energetic states
\begin{dmath}\label{eqn:spinZeroBoseCondensation:80}
N_e
=
\sum_{\Bk \ne 0} \inv{z^{-1} e^{\beta \epsilon_\Bk} - 1}
\sim
V \int \frac{d^3 \Bk}{(2 \pi)^3}
\inv{z^{-1} e^{\beta \epsilon_\Bk} - 1}
=
\frac{4 \pi V}{(2 \pi)^3} \int_0^\infty k^2 dk
\inv{z^{-1} e^{\beta c k} - 1}
=
\frac{V}{2 \pi^2}
\lr{ \inv{\beta c} }^3
\int_0^\infty x^2 dx
\inv{z^{-1} e^{x} - 1}
=
\frac{V}{2 \pi^2}
\lr{ \inv{\beta c} }^3
\Gamma(3) g_3(z).
\end{dmath}
%
So we have
\begin{equation}\label{eqn:spinZeroBoseCondensation:100}
N_e
=
\frac{V}{\pi^2}
\lr{ \frac{\kB T}{c} }^3
g_3(z)
\le
\frac{V}{\pi^2}
\lr{ \frac{\kB T}{c} }^3
\zeta(3).
\end{equation}
%
Note that \(\zeta(3) \approx 1.20206\), a fixed number.  The key feature of Bose condensation remains.  There is a finite limit to the number of particles that can be in the energetic state at a given temperature and volume.  Any remaining particles are forced into the ground state.
%
In general the number of particles in the ground state is
\begin{equation}\label{eqn:spinZeroBoseCondensation:120}
N_0 = N - \frac{V}{\pi^2}
\lr{ \frac{\kB T}{c} }^3
g_3(z),
\end{equation}
and we will necessarily have particles in this state if
\begin{equation}\label{eqn:spinZeroBoseCondensation:140}
N - \frac{V}{\pi^2}
\lr{ \frac{\kB T}{c} }^3
\zeta(3) > 0.
\end{equation}
%
That temperature threshold \(T \le T_c\) is the Bose condensation temperature
\boxedEquation{eqn:spinZeroBoseCondensation:160}{
\kB T_c = c
\lr{ \frac{n \pi^2}{\zeta(3)} }
^{1/3}.
}
%
With \(n = N/V\), \(n_0 = N_0/V\), we have for the ground state average number density
\begin{equation}\label{eqn:spinZeroBoseCondensation:n}
n_0 = n
\left( 1 - \frac{g_3(z)}{\zeta(3)}
\lr{ \frac{T}{T_c} }^3
\right).
\end{equation}
%
This is plotted in \cref{fig:spinZeroBoseCondensation:spinZeroBoseCondensationFig1}.
%
\imageFigure{../figures/phy452-basicstatmech/spinZeroBoseCondensationFig1}{Ratio of ground state number density to total number density.}{fig:spinZeroBoseCondensation:spinZeroBoseCondensationFig1}{0.2}
%
From the figure it appears that the notion of any sort of absolute condensation temperature is an approximation.  We can start having particles go into the ground state at higher temperatures than \(T_c\), but once the chemical potential starts approaching zero, that temperature for which we start having particles in the ground state approaches \(T_c\).  The key takeout idea appears to be, once the temperature does drop below \(T_c\), we necessarily start having a non-zero ground state population, and as the temperature drops more and more, the ratio of the number of particles in the ground state relative to the total approaches unity (all particles are forced into the ground state).
} % makeanswer
%
%\EndArticle
