%
% Copyright � 2013 Peeter Joot.  All Rights Reserved.
% Licenced as described in the file LICENSE under the root directory of this GIT repository.
%
%\input{../blogpost.tex}
%\renewcommand{\basename}{velocityToMomentumChangeOfVariables}
%\renewcommand{\dirname}{notes/phy452/}
%\newcommand{\keywords}{special relativity, velocity space volume element, momentum space volume element, Jacobian}
%
%\input{../peeter_prologue_print2.tex}
%
%\beginArtNoToc
%
%\generatetitle{Velocity volume element to momentum volume element}
%\chapter{Velocity volume element to momentum volume element}
\label{chap:velocityToMomentumChangeOfVariables}
%
%\section{Motivation}
%
One of the problems I attempted had integrals over velocity space with volume element \(d^3\Bu\).  Initially I thought that I'd need a change of variables to momentum space, and calculated the corresponding momentum space volume element.  Here's that calculation.
%
%\section{Guts}
%
We are working with a Hamiltonian
\begin{dmath}\label{eqn:velocityToMomentumChangeOfVariables:200}
\epsilon = \sqrt{ (p c)^2 + \epsilon_0^2 },
\end{dmath}
where the rest energy is
\begin{dmath}\label{eqn:velocityToMomentumChangeOfVariables:220}
\epsilon_0 = m c^2.
\end{dmath}
%
Hamilton's equations give us
\begin{dmath}\label{eqn:velocityToMomentumChangeOfVariables:240}
u_\alpha = \frac{ p_\alpha/c^2 }{\epsilon},
\end{dmath}
or
\begin{dmath}\label{eqn:velocityToMomentumChangeOfVariables:260}
p_\alpha = \frac{ m u_\alpha }{\sqrt{1 - \Bu^2/c^2}}.
\end{dmath}
%
This is enough to calculate the Jacobian for our volume element change of variables
\begin{dmath}\label{eqn:huang93:160}
\begin{aligned}
du_x &\wedge du_y \wedge du_z
=
\frac
{\partial(u_x, u_y, u_z)}
{\partial(p_x, p_y, p_z)}
dp_x \wedge dp_y \wedge dp_z \\
&=
\frac{
dp_x \wedge dp_y \wedge dp_z
}{c^6 \lr{ m^2 + (\Bp/c)^2 }^{9/2}} \times \\
&\begin{vmatrix}
m^2 c^2 + p_y^2 + p_z^2 & - p_y p_x & - p_z p_x \\
-p_x p_y & m^2 c^2 + p_x^2 + p_z^2 & - p_z p_y \\
-p_x p_z & -p_y p_z & m^2 c^2 + p_x^2 + p_y^2
\end{vmatrix} \\
&=
m^2 \lr{ m^2 + \Bp^2/c^2 }^{-5/2}
dp_x \wedge dp_y \wedge dp_z.
\end{aligned}
\end{dmath}
%
That final simplification of the determinant was a little hairy, but yielded nicely to Mathematica \nbref{huang93relativisiticGas.nb}.
%
Our final result for the velocity volume element in momentum space, in terms of the particle energy is
\begin{dmath}\label{eqn:huang93:180}
d^3 \Bu = \frac{c^6 \epsilon_0^2 } {\epsilon^5} d^3 \Bp.
\end{dmath}
%
%\EndNoBibArticle
