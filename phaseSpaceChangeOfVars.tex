%
% Copyright � 2013 Peeter Joot.  All Rights Reserved.
% Licenced as described in the file LICENSE under the root directory of this GIT repository.
%
%\input{../blogpost.tex}
%\renewcommand{\basename}{phaseSpaceChangeOfVars}
%\renewcommand{\dirname}{notes/phy452/}
%\newcommand{\keywords}{Statistical mechanics, PHY452H1S, phase space, point transformation, Jacobian, cylindrical coordinates}
%
%\input{../peeter_prologue_print2.tex}
%
%\beginArtNoToc
%
%\generatetitle{Change of variables in 2d phase space}
%%\chapter{Change of variables in 2d phase space}
\label{chap:phaseSpaceChangeOfVars}
%\section{Motivation}

\makeproblem{Cartesian to cylindrical change of variables in 2d phase space}{pr:phaseSpaceChangeOfVars:1}{
In \citep{pathriastatistical} problem 2.2, it's suggested to try a spherical change of vars to verify explicitly that phase space volume is preserved, and to explore some related ideas.  As a first step let's try a similar, but presumably easier change of variables, going from Cartesian to cylindrical phase spaces.
} % makeproblem

\makeanswer{pr:phaseSpaceChangeOfVars:1}{
%\section{Canonical momenta and Hamiltonian}
Our cylindrical velocity is
\begin{dmath}\label{eqn:phaseSpaceChangeOfVars:20}
\Bv = \rdot \rcap + r \thetadot \thetacap,
\end{dmath}
so a purely kinetic Lagrangian would be
\begin{dmath}\label{eqn:phaseSpaceChangeOfVars:40}
\LL
= \inv{2} m \Bv^2
= \inv{2} m \lr{\rdot^2 + r^2 \thetadot^2}.
\end{dmath}

Our canonical momenta are
\begin{subequations}
\begin{equation}\label{eqn:phaseSpaceChangeOfVars:60}
p_r = \PD{\rdot}{\LL} = m \rdot
\end{equation}
\begin{equation}\label{eqn:phaseSpaceChangeOfVars:80}
p_\theta = \PD{\thetadot}{\LL} = m r^2 \thetadot,
\end{equation}
\end{subequations}
and our kinetic energy is
\begin{equation}\label{eqn:phaseSpaceChangeOfVars:100}
H = \LL = \inv{2m} p_r^2 + \inv{2 m r^2} p_\theta^2.
\end{equation}

Now we need to express our momenta in terms of the Cartesian coordinates.  We have for the radial momentum
\begin{dmath}\label{eqn:phaseSpaceChangeOfVars:120}
p_r
= m \rdot
= m \ddt{} \sqrt{x^2 + y^2}
= \inv{2} \frac{2 m}{r} \lr{ x \xdot + y \ydot },
\end{dmath}
or
\begin{equation}\label{eqn:phaseSpaceChangeOfVars:140}
p_r = \inv{r} \lr{ x p_x + y p_y }.
\end{equation}

\begin{dmath}\label{eqn:phaseSpaceChangeOfVars:160}
p_\theta
= m r^2 \ddt{\theta}
= m r^2 \ddt{} \Atan \lr{\frac{y}{x}}.
\end{dmath}

After some reduction \nbref{cyclindrialMomenta.nb}, we find
\begin{equation}\label{eqn:phaseSpaceChangeOfVars:180}
p_\theta = p_y x - p_x y.
\end{equation}

We can assemble these into a complete set of change of variable equations
\begin{subequations}
\begin{equation}\label{eqn:phaseSpaceChangeOfVars:200}
r = \sqrt{x^2 + y^2}
\end{equation}
\begin{equation}\label{eqn:phaseSpaceChangeOfVars:220}
\theta = \Atan\lr{\frac{y}{x}}
\end{equation}
\begin{equation}\label{eqn:phaseSpaceChangeOfVars:240}
p_r = \inv{\sqrt{x^2 + y^2}} \lr{ x p_x + y p_y }
\end{equation}
\begin{equation}\label{eqn:phaseSpaceChangeOfVars:260}
p_\theta = p_y x - p_x y.
\end{equation}
\end{subequations}

Our phase space volume element change of variables is
\begin{dmath}\label{eqn:phaseSpaceChangeOfVars:280}
dr d\theta dp_r dp_\theta =
\frac{\partial(r, \theta, p_r, p_\theta)}{\partial(x, y, p_x, p_y)}
dx dy dp_x dp_y
=
\begin{vmatrix}
 \frac{x}{\sqrt{x^2+y^2}} & \frac{y}{\sqrt{x^2+y^2}} & 0 & 0 \\
 -\frac{y}{x^2+y^2} & \frac{x}{x^2+y^2} & 0 & 0 \\
 \frac{y \left(y p_x-x p_y\right)}{\left(x^2+y^2\right)^{3/2}} & \frac{x \left(x p_y-y p_x\right)}{\left(x^2+y^2\right)^{3/2}} & \frac{x}{\sqrt{x^2+y^2}} & \frac{y}{\sqrt{x^2+y^2}} \\
 p_y & -p_x & -y & x
\end{vmatrix}
dx dy dp_x dp_y
=
\frac{x^2 + y^2}{\left(x^2 + y^2\right)^{3/2}}
\frac{x^2 + y^2}{\left(x^2 + y^2\right)^{1/2}}
%\frac{x^6+3 x^4 y^2+3 x^2 y^4+y^6}{\left(x^2+y^2\right)^3}
%dx dy dp_x dp_y
=
dx dy dp_x dp_y.
\end{dmath}

We see explicitly that this point transformation has a unit Jacobian, preserving area.
} % makeanswer
%\EndArticle
