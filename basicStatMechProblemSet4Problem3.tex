%
% Copyright � 2013 Peeter Joot.  All Rights Reserved.
% Licenced as described in the file LICENSE under the root directory of this GIT repository.
%
\makeoproblem{Adiabatic process for an Ideal Gas}{basicStatMech:problemSet4:3}{\citep{schroeder2000thermal} 1.40(a)}{
%(3 points)
Show that when an ideal monoatomic gas expands adiabatically, the temperature and pressure are related by
\begin{equation}\label{eqn:basicStatMechProblemSet4Problem3:20}
\frac{dT}{dP
}
=
\frac{2}{5}
\frac{T}{P}.
\end{equation}
} % makeoproblem
%
\makeanswer{basicStatMech:problemSet4:3}{
%
From (3.34b) of \citep{kittel1980thermal}, we find that the Adiabatic condition can be expressed algebraically as
\begin{equation}\label{eqn:basicStatMechProblemSet4Problem3:40}
0 = \dbar Q = T dS = dU + P dV.
\end{equation}
%
With
\begin{equation}\label{eqn:basicStatMechProblemSet4Problem3:60}
U = \frac{3}{2} N \kB T = \frac{3}{2} P V,
\end{equation}
this is
\begin{equation}\label{eqn:basicStatMechProblemSet4Problem3:80}
\begin{aligned}
0
&= \frac{3}{2} V dP + \frac{3}{2} P dV + P dV \\
&= \frac{3}{2} V dP + \frac{5}{2} P dV.
\end{aligned}
\end{equation}
%
Dividing through by \(P V\), this becomes a perfect differential, and we can integrate
\begin{equation}\label{eqn:basicStatMechProblemSet4Problem3:100}
\begin{aligned}
0
&= 3 \int \frac{dP }{P} + 5 \int \frac{dV}{V} \\
&= 3 \ln P + 5 \ln V + \ln C \\
&= 3 \ln PV + 2 \ln V + \ln C \\
&= \ln (N \kB T)^3 + \ln \lr{\frac{N \kB T}{P}}^2 + \ln C.
\end{aligned}
\end{equation}
%
Exponentiating yields
\begin{equation}\label{eqn:basicStatMechProblemSet4Problem3:120}
T^5 = C' P^2.
\end{equation}
%
The desired relation follows by taking derivatives
\begin{equation}\label{eqn:basicStatMechProblemSet4Problem3:140}
\begin{aligned}
2 C' P
&= 5 T^4 \frac{dT}{dP} \\
&= 5 C' \frac{P^2}{T} \frac{dT}{dP},
\end{aligned}
\end{equation}
or
\begin{equation}\label{eqn:basicStatMechProblemSet4Problem3:160}
\frac{dT}{dP} =
\frac{2}{5} \frac{T}{P},
\end{equation}
as desired.
}
