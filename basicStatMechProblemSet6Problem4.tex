%
% Copyright � 2013 Peeter Joot.  All Rights Reserved.
% Licenced as described in the file LICENSE under the root directory of this GIT repository.
%
\makeoproblem{Neutron star}{basicStatMech:problemSet6:4}{\citep{huang2001introduction}, prob 9.5}{
%(3 points)
Model a neutron star as an ideal Fermi gas of neutrons at \(T = 0\) moving in the gravitational field of a heavy point mass \(M\) at the center. Show that the pressure \(P\) obeys the equation
\begin{dmath}\label{eqn:basicStatMechProblemSet6Problem4:20}
\frac{dP}{dr} = - \gamma M \frac{\rho(r)}{r^2},
\end{dmath}

where \(\gamma\) is the gravitational constant, \(r\) is the distance from the center, and \(\rho(r)\) is the density which only depends on distance from the center.
} % makeoproblem

\makeanswer{basicStatMech:problemSet6:4}{

In the grand canonical scheme the pressure for a Fermion system is given by
\begin{dmath}\label{eqn:basicStatMechProblemSet6Problem4:40}
\beta P V
= \ln \ZG
= \ln \prod_\epsilon \sum_{n = 0}^1 \lr{ z e^{-\beta \epsilon} }^n
= \sum_\epsilon \ln \lr{ 1 + z e^{-\beta \epsilon} }.
\end{dmath}

The kinetic energy of the particle is adjusted by the gravitational potential
\begin{dmath}\label{eqn:basicStatMechProblemSet6Problem4:60}
\epsilon = \epsilon_\Bk
- \frac{\gamma m M}{r}
= \frac{\Hbar^2 \Bk^2}{2m}
- \frac{\gamma m M}{r}.
\end{dmath}

Differentiating \eqnref{eqn:basicStatMechProblemSet6Problem4:40} with respect to the radius, we have
\begin{dmath}\label{eqn:basicStatMechProblemSet6Problem4:80}
\beta V \PD{r}{P}
= -\beta \PD{r}{\epsilon}
\sum_\epsilon \frac{z e^{-\beta \epsilon}}{ 1 + z e^{-\beta \epsilon} }
= -\beta
\lr{
\frac{\gamma m M}{r^2}
}
\sum_\epsilon \frac{1}{ z^{-1} e^{\beta \epsilon} + 1}
=
-\beta
\lr{
\frac{\gamma m M}{r^2}
}
\expectation{N}.
\end{dmath}

Noting that \(\expectation{N} m/V\) is the average density of the particles, presumed radial, we have

\boxedEquation{eqn:basicStatMechProblemSet6Problem4:100}{
\PD{r}{P}
=
-
\frac{\gamma M}{r^2} \frac{m \expectation{N}}{V}
=
-
\frac{\gamma M}{r^2} \rho(r).
}
}
