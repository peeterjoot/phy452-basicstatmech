%
% Copyright � 2013 Peeter Joot.  All Rights Reserved.
% Licenced as described in the file LICENSE under the root directory of this GIT repository.
%
%\input{../blogpost.tex}
%\renewcommand{\basename}{basicStatMechLecture18}
%\renewcommand{\dirname}{notes/phy452/}
%\newcommand{\keywords}{Statistical mechanics, PHY452H1S, Fermi gas, specific heat, density of states, graphene, relativistic gas, chemical potential, energy, Fermi distribution, hole, electron}
%\input{../peeter_prologue_print2.tex}
%
%\beginArtNoToc
%\generatetitle{PHY452H1S Basic Statistical Mechanics.  Lecture 18: Fermi gas thermodynamics.  Taught by Prof.\ Arun Paramekanti}
%\chapter{Fermi gas thermodynamics}
\label{chap:basicStatMechLecture18}

%\section{Disclaimer}
%
%Peeter's lecture notes from class.  May not be entirely coherent.
%
\paragraph{Review}

Last time we found that the low temperature behavior or the \textAndIndex{chemical potential} was quadratic as in \cref{fig:lecture18:lecture18Fig1}.
\begin{dmath}\label{eqn:basicStatMechLecture18:20}
\mu =
\mu(0) - a \frac{T^2}{T_{\txtF}}
%^
%\kB T ?
\end{dmath}

\imageFigure{../figures/phy452-basicstatmech/lecture18Fig1}{Fermi gas chemical potential}{fig:lecture18:lecture18Fig1}{0.2}

\paragraph{Specific heat}
\begin{dmath}\label{eqn:basicStatMechLecture18:40}
E = \sum_\Bk n_{\txtF}(\epsilon_\Bk, T) \epsilon_\Bk
\end{dmath}

\begin{dmath}\label{eqn:basicStatMechLecture18:60}
\frac{E}{V}
= \inv{(2\pi)^3} \int d^3 \Bk n_{\txtF}(\epsilon_\Bk, T) \epsilon_\Bk
= \int d\epsilon N(\epsilon) n_{\txtF}(\epsilon, T) \epsilon,
\end{dmath}
where
\begin{dmath}\label{eqn:basicStatMechLecture18:80}
N(\epsilon) = \inv{4 \pi^2}
\lr{\frac{2m}{\Hbar^2}}
^{3/2}
\sqrt{\epsilon}.
\end{dmath}

\paragraph{Low temperature \(\CV\)}

\begin{dmath}\label{eqn:basicStatMechLecture18:100}
\frac{\Delta E(T)}{V}
=
\int_0^\infty d\epsilon N(\epsilon)
\lr{ n_{\txtF}(\epsilon, T) - n_{\txtF}(\epsilon, 0)}
\end{dmath}

The only change in the distribution \cref{fig:lecture18:lecture18Fig2}, that is of interest is over the step portion of the distribution, and over this range of interest \(N(\epsilon)\) is approximately constant as in \cref{fig:lecture18:lecture18Fig3}.
\imageFigure{../figures/phy452-basicstatmech/lecture18Fig2}{Fermi distribution}{fig:lecture18:lecture18Fig2}{0.2}
\imageFigure{../figures/phy452-basicstatmech/lecture18Fig3}{Fermi gas density of states}{fig:lecture18:lecture18Fig3}{0.2}
\begin{subequations}
\begin{dmath}\label{eqn:basicStatMechLecture18:120}
N(\epsilon) \approx N(\mu)
\end{dmath}
\begin{dmath}\label{eqn:basicStatMechLecture18:140}
\mu \approx \epsilon_{\txtF},
\end{dmath}
\end{subequations}
so that
\begin{dmath}\label{eqn:basicStatMechLecture18:160}
\Delta e \equiv
\frac{\Delta E(T)}{V}
\approx
N(\epsilon_{\txtF})
\int_0^\infty d\epsilon
\lr{ n_{\txtF}(\epsilon, T) - n_{\txtF}(\epsilon, 0)}
=
N(\epsilon_{\txtF})
\int_{-\epsilon_{\txtF}}^\infty d x (\epsilon_{\txtF} + x)
\lr{ n_{\txtF}(\epsilon + x, T) - n_{\txtF}(\epsilon_{\txtF} + x, 0)}.
\end{dmath}

Here we've made a change of variables \(\epsilon = \epsilon_{\txtF} + x\), so that we have near cancellation of the \(\epsilon_{\txtF}\) factor
\begin{dmath}\label{eqn:basicStatMechLecture18:180}
\Delta e
=
N(\epsilon_{\txtF})
\epsilon_{\txtF}
\int_{-\epsilon_{\txtF}}^\infty d x
\mathLabelBox{
\lr{ n_{\txtF}(\epsilon + x, T) - n_{\txtF}(\epsilon_{\txtF} + x, 0)}
}{almost equal everywhere}
+
N(\epsilon_{\txtF})
\int_{-\epsilon_{\txtF}}^\infty d x x
\lr{ n_{\txtF}(\epsilon + x, T) - n_{\txtF}(\epsilon_{\txtF} + x, 0)}
\approx
N(\epsilon_{\txtF})
\int_{-\infty}^\infty d x x
\lr{
\inv{ e^{\beta x} +1 }
-
\evalbar{\inv{ e^{\beta x} +1 }}{T \rightarrow 0}
}.
\end{dmath}

Here we've extended the integration range to \(-\infty\) since this doesn't change much.  FIXME: justify this to myself?  Taking derivatives with respect to temperature we have
\begin{dmath}\label{eqn:basicStatMechLecture18:200}
\frac{\delta e}{T}
=
-N(\epsilon_{\txtF})
\int_{-\infty}^\infty d x x
\inv{(e^{\beta x} + 1)^2}
\frac{d}{dT} e^{\beta x}
=
N(\epsilon_{\txtF})
\int_{-\infty}^\infty d x x
\inv{(e^{\beta x} + 1)^2}
e^{\beta x}
\frac{x}{\kB T^2}.
\end{dmath}

With \(\beta x = y\), we have for \(T \ll T_{\txtF}\)
\begin{dmath}\label{eqn:basicStatMechLecture18:220}
\frac{C}{V}
=
N(\epsilon_{\txtF})
\int_{-\infty}^\infty \frac{ dy y^2 e^y }{ (e^y + 1)^2 \kB T^2} (\kB T)^3
=
N(\epsilon_{\txtF}) \kB^2 T
\mathLabelBox
[
   labelstyle={xshift=2cm},
   linestyle={out=270,in=90, latex-}
]
{
\int_{-\infty}^\infty \frac{ dy y^2 e^y }{ (e^y + 1)^2 }
}{\(\pi^2/3\)}
=
\frac{\pi^2}{3} N(\epsilon_{\txtF}) \kB (\kB T).
\end{dmath}

Using \eqnref{eqn:basicStatMechLecture18:80} at the Fermi energy and
\begin{subequations}
\begin{dmath}\label{eqn:basicStatMechLecture18:240}
\frac{N}{V} = \rho
\end{dmath}
\begin{dmath}\label{eqn:basicStatMechLecture18:260}
\epsilon_{\txtF} = \frac{\Hbar^2 \kF^2}{2 m}
\end{dmath}
\begin{dmath}\label{eqn:basicStatMechLecture18:280}
\kF = \lr{6 \pi^2 \rho}
^{1/3},
\end{dmath}
\end{subequations}
we have
\begin{dmath}\label{eqn:basicStatMechLecture18:320}
N(\epsilon_{\txtF})
= \inv{4 \pi^2}
\lr{\frac{2m}{\Hbar^2}}
^{3/2}
\sqrt{\epsilon_{\txtF}}
= \inv{4 \pi^2}
\lr{\frac{2m}{\Hbar^2}}
^{3/2}
\frac{\Hbar \kF}{\sqrt{2m}}
= \inv{4 \pi^2}
\lr{\frac{2m}{\Hbar^2}}
^{3/2}
\frac{\Hbar }{\sqrt{2m}} \lr{6 \pi^2 \rho}^{1/3}
=
\inv{4 \pi^2}
\lr{\frac{2m}{\Hbar^2}}
\lr{6 \pi^2 \frac{N}{V}}^{1/3}.
\end{dmath}

Giving
\begin{dmath}\label{eqn:basicStatMechLecture18:480}
\frac{C}{N}
=
\frac{\pi^2}{3}
\frac{V}{N}
\inv{4 \pi^2}
\lr{\frac{2m}{\Hbar^2}}
\lr{6 \pi^2 \frac{N}{V}}
^{1/3}
\kB (\kB T)
=
\lr{\frac{m}{6 \Hbar^2}}
\lr{\frac{V}{N}}^{2/3}
\lr{6 \pi^2}
^{1/3}
\kB (\kB T)
=
\lr{\frac{ \pi^2 m}{3 \Hbar^2}}
\lr{\frac{V}{\pi^2 N}}^{2/3}
\kB (\kB T)
=
\lr{\frac{ \pi^2 m}{\Hbar^2}}
\frac{\Hbar^2}{2 m \epsilon_{\txtF}}
\kB (\kB T),
\end{dmath}
or
\boxedEquation{eqn:basicStatMechLecture18:300}{
\frac{C}{N} =
\frac{\pi^2}{2} \kB \frac{ \kB T}{\epsilon_{\txtF}}.
}

This is illustrated in \cref{fig:lecture18:lecture18Fig4}.  Note that \citep{pathriastatistical} outlines this derivation in the tail end of \S 8.1, and you can look there to see the high level view without any of the details above obscuring things.

\imageFigure{../figures/phy452-basicstatmech/lecture18Fig4}{Specific heat per Fermion}{fig:lecture18:lecture18Fig4}{0.25}

